\section*{Generalized Munchausen Numbers}

We will follow the conventions used in \cite{dvb}. $b\in\N$ will denote a base.
Therefore $b \ge 2$. For every $n \in \N$ the \emph{base $b$ representation
of $n$} will be denoted by $[c_{m-1}, c_{m-2}, \ldots, c_{0}]_{b}$, so 
$0 \le c_{i} < b$ for all $i \in \{0,1,\ldots,m-1\}$ and 
$n = \sum_{i=0}^{m-1} c_{i}b^{i}$.

For every $f : \R \rightarrow \R$ define a function $\F_{b} : \N \rightarrow
\N : n \mapsto \sum_{i=0}^{m-1} f(c_{i})$.

\begin{definition}
	An integer $n \in \N$ is called a \emph{$f$-Munchausen number in base $b$}
	if and only if $n = \F_{b}(n)$.
\end{definition}


\begin{lemma}
	For all functions $f : \R \rightarrow \R$ there exist an 
	$M \in \R$ such that for all $n \in \N$: $\F_{b}(n) \le (\blog(n) + 1) M$.
\end{lemma}

\begin{proof}
	Let $f : \R \rightarrow \R$ be a function. Because $\{0,\ldots,m-1\}$ has 
	finitely many elements $f$ will attain a maximum on this set. Let $M$ be the
	maximum	attained by $f$ on $\{0,\ldots,m-1\}$. Hence, for all 
	$c \in \{0,\ldots,m-1\}$: $f(c) \le M$.
	
	So $\F_{b}(n) = \sum_{i=0}^{m-1} f(c_{i}) \le 
	\sum_{i=0}^{m-1} M = m \times M$.
	
	Now, the number of digits in the base $b$ represantation of $n$ equals 
	$\floor{\blog(n) + 1}$. In other words $m := \floor{\blog(n) + 1} 
	\le \blog(n) + 1$.
	
	So for every function $f$ there exist an $M$ such that 
	$\F_{b}(n) \le (\blog(n) + 1) M$.
\end{proof}

\begin{lemma}
	For all $n \in \N$: 
	\[
		\lim_{x \rightarrow \infty} \frac{e^{x}}{x^{n}} = \infty
	\]
\end{lemma}

\begin{proof}
	We will proof this lemma with induction. In this proof we will call a number
	$n \in \N$ \emph{strong} if and only if
	$\lim_{x \rightarrow \infty} \frac{e^{x}}{x^{n}} = \infty$.
	
	We will show that $0$ is strong by proving that $x \lt e^{x}$ for all 
	$x \in \R$. First of all, notice that for $x \le 0$: $x \le 0 \lt e^{x}$.
	Secondly, for $x \ge 0$	the derivative of $e^{x}$ is greater then the 
	derivative of $x$ and $e^{0} = 1 \gt 0$. So $e^{x} \ge x$ for all $x \ge 0$.
	Hence $\lim_{x \rightarrow \infty} e^{x} = \infty$. This proves that $0$
	is strong.
	
	Now assume that $n$ is strong. We will show that $n+1$ is strong as well.
\end{proof}
