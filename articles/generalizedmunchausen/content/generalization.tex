\section*{$f$-Digit Invariant Numbers}

We will follow the conventions used in \cite{dvb}. $b\in\N$ will denote a base.
Therefore $b \ge 2$. For every $n \in \N$ the \emph{base $b$ representation
of $n$} will be denoted by $[c_{m-1}, c_{m-2}, \ldots, c_{0}]_{b}$, so 
$0 \le c_{i} < b$ for all $i \in \{0,1,\ldots,m-1\}$ and 
$n = \sum_{i=0}^{m-1} c_{i}b^{i}$.

For every $f : \R \rightarrow \R$ define a function $\F_{b} : \N \rightarrow
\N : n \mapsto \sum_{i=0}^{m-1} f(c_{i})$.

\begin{definition}
	An integer $n \in \N$ is called a \emph{$f$-digit invariant number in base
	$b$} if and only if $n = \F_{b}(n)$.
\end{definition}

\begin{proposition}
	For all functions $f : \R \rightarrow \R$ there exist an 
	$M \in \R$ such that for all $n \in \N$: $\F_{b}(n) \le (\blog(n) + 1) M$.
\end{proposition}

\begin{proof}
	Let $f : \R \rightarrow \R$ be a function. Because $\{0,\ldots,m-1\}$ has 
	finitely many elements $f$ will attain a maximum on this set. Let $M$ be the
	maximum	attained by $f$ on $\{0,\ldots,m-1\}$. Hence, for all 
	$c \in \{0,\ldots,m-1\}$: $f(c) \le M$.
	
	So $\F_{b}(n) = \sum_{i=0}^{m-1} f(c_{i}) \le 
	\sum_{i=0}^{m-1} M = m \times M$.
	
	Now, the number of digits in the base $b$ represantation of $n$ equals 
	$\floor{\blog(n) + 1}$. In other words $m := \floor{\blog(n) + 1} 
	\le \blog(n) + 1$.
	
	So for every function $f$ there exist an $M$ such that
  \[
	  \F_{b}(n) \le (\blog(n) + 1) M
  \]
\end{proof}

\begin{lemma}
	For all $x \in \R$: $x < b^{x}$.
\end{lemma}

\begin{proof}
	for all $x \in \R$ the inequality $b^{x} \gt 0$ holds. So for $x \le 0$ the
	lemma holds trivially.
	
	Now for $x \ge 0$ the deriviative of $b^{x}$ is greater than the derivative of
  $x$. Combining this with the fact that $b^{0} = 1 > 0$ we can conclude that $x
  \lt b^{x}$ for all $x \ge 0$. 
	
	This proves that $x < b^{x}$ holds for all $x \in \R$.
\end{proof}

\begin{corollary}
	for non-negative $x \in \R$: $\frac{b^{x}}{x} \ge 1$.
\end{corollary}

\begin{proof}
	$1 = \frac{x}{x} \le \frac{b^{x}}{x}$ for $x \ge 0$.
\end{proof}

\begin{proposition}
	For all $n \in \N$: 
	\[
		\lim_{x \rightarrow \infty} \frac{b^{x}}{x^{n}} = \infty
	\]
\end{proposition}

\begin{proof}
	We will prove this lemma with induction. In this proof we will call a number
	$n \in \N$ \emph{strong} if and only if
	$\lim_{x \rightarrow \infty} \frac{b^{x}}{x^{n}} = \infty$.
	
	We can conclude from the above lemma that $0$ is strong.
	
	Next we will show that $1$ is strong.
	\[
		\frac{b^{x}}{x} =%
		\frac{1}{2}\frac{\left(b^{\frac{x}{2}}\right)^2}{\frac{x}{2}} =%
		\frac{1}{2}\left(\frac{b^{\frac{x}{2}}}{\frac{x}{2}}\right)b^{\frac{x}{2}}
	\]
	In the above equality we know from the above corollary that 
	$\frac{b^{\frac{x}{2}}}{\frac{x}{2}} > 1$ for all non-negative $x$.
	Because $0$ is strong we can conclude that $1$ is strong as well.
	
	Now assume that $n \in N$ is strong. We will show that $n+1$ is also strong.
	Examine the following equality:
	\[
		\frac{b^{x}}{x^{n+1}} =%
		\frac{(n+1)^{n+1}}{(n+1)^{n+1}}%
		\frac{\left(b^{\frac{x}{n+1}}\right)^{n+1}}{x^{n+1}} =%
		\frac{\left(b^{\frac{n-1}{n+1}}\right)^{x}}{(n+1)^{n+1}}%
		\left(\frac{b^{\frac{x}{n+1}}}{\frac{x}{n+1}}\right)%
		\frac{b^{\frac{x}{n+1}}}{\left(\frac{x}{n+1}\right)^{n}}%	
	\]
	Again, by using the fact the lemma, its corollary and the induction
  hypotheses, it is clear that $n+1$ is strong.
	
	This proves that all natural numbers are strong i.e for all $n \in \N$
	$\lim_{x \rightarrow \infty} \frac{b^{x}}{x^{n}} = \infty$.
\end{proof}

\begin{corollary}
	for all $n \in \N$:
	\[
		\lim_{y \rightarrow \infty} \frac{{y}}{\left(\blog{y}\right)^{n}} = \infty
	\]
\end{corollary}

\begin{proof}
	Substitute $x = \blog{y}$.
\end{proof}

\begin{theorem}
	For every $f : \R \rightarrow \R$ there are finitly many $f$-digit
	invariant numbers.
\end{theorem}

\begin{proof}
	Let $M$ be the bound from the first proposition. By the preceding lemma 
	there exist a $N \in \N$ such that for all $n \in \N$ with $n \gt N$:
	$\frac{x}{\blog{x}} \gt 2M$. Hence
	\[
		\frac{x}{\blog{x} + 1} \ge \frac{x}{2\blog{x}} \gt M
	\]
	
	Now
	\[
		n \gt (\blog{x} + 1) M \gt \mathcal{F}_{b} (n)
	\]
	
	This proves that there are only finitly many possible candidates for the
	equilaty $n = \mathcal{F}_{b}(n)$ i.e. there are finitly many $f$ digit
	invariant numbers.
\end{proof}
