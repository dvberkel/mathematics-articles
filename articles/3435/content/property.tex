\section*{Munchausen Number}
Through out the article we will use the following notation. $b \in \N$ will 
denotate a base and therefore the inequility $b \ge 2$ will hold throughout the
article. For every natural number $n \in \N$, the \emph{base $b$ representation
of $n$} will be denoted by $[c_{m-1}, c_{m-2}, \ldots, c_{0}]_{b}$, 
so $0 \le c_{i} < b$ for all $i \in \{0,1,\ldots,m-1\}$ and 
$n = \sum_{i=0}^{m-1} c_{i}b^{i}$.
Furtheremore, we define a function $\P_{b} : \N \rightarrow \N : n \mapsto 
\sum_{i=0}^{m-1} c_{i}^{c_{i}}$, where $n = [c_{m-1},c_{m-2},\ldots,c_{0}]_{b}$.
We will further adopt the convention that $0^{0} = 1$, in accordance with 
$1^{0} = 1$, $2^{0} = 1$ etcetera.

\begin{definition}
	An integer $n \in \N$ is called a \emph{Munchausen number in base $b$} if 
	and only if $n = \P_{b}(n)$.
\end{definition}

So by the equality in the introduction we know that $3435$ is a Munchausen
number in base $10$. 

\begin{remark}
	A related concept to Munchausen number is that of Narcissistic number. 
	(See for example \cite{pickover}, \cite{wikipedia:narcissistic_number} and
	\cite{wolfram:narcissistic_number}.)
	
	The reason for picking the name Munchausen number stems from the visual of
	raising oneself, a feat demonstrated by the famous Baron von Munchausen 
	(\cite{wikipedia:munchausen}). Andrew Baxter remarked that the Baron is a 
	narcissistic man indeed, so I think the name is aptly chosen.
\end{remark}

The following two lemmas will be used to proof the main 
result of this article: for every base $b \in \N$ there are only finitely many 
Munchausen numbers in base $b$.

\begin{lemma}
	For all $n \in \N$: $\P_{b}(n) \le (\blog(n) + 1)(b-1)^{b-1}$.
\end{lemma}

\begin{proof}
	Notice that the function $x \mapsto x^{x}$ is strictly increasing if 
	$x \ge \frac{1}{e}$. This can be seen from the derivative of $x^{x}$ which 
	is $x^x(\log(x) + 1)$. This last expression is clearly positive for 
	$x > \frac{1}{e}$.
	Together with the definition of $0^{0} = 1$, we see that $x^{x}$ is 
	increasing for all the nonnegative integers.
	
	For all $n \in \N$ with $n = [c_{m-1}, c_{m-2}, \ldots, c_{0}]_{b}$ we have 
	the ineqalities $0 \le c_{i} \le b-1$ for all $i$ within $0 \le i < m$.	\\
	So $\P_{b}(n) = \sum_{i=0}^{m-1} c_{i}^{c_{i}} \le 
	\sum_{i=0}^{m-1} (b-1)^{b-1} = m \times (b-1)^{b-1}$.
	
	Now, the number of digits in the base $b$ represantation of $n$ equals 
	$\floor{\blog(n) + 1}$. In other words $m := \floor{\blog(n) + 1} 
	\le \blog(n) + 1$.
	
	So $\P_{b}(n) \le (\blog(n) + 1)(b-1)^{b-1}$
\end{proof}

\begin{lemma}
	If $n \in \N$ and $n > 2b^{b}$ then $\frac{n}{\blog(n) + 1}	> (b-1)^{b-1}$.
\end{lemma}

\begin{proof}
	Let $n \in \N$ such that $n > 2b^{b}$. Notice that 
	$x \mapsto \frac{x}{\blog(x)}$ is strictly increasing if $x > e$. To see
	this notice that the derivative of $\frac{x}{\blog{x}}$ is 
	$\log(b)\frac{\log(x) - 1}{\log^2(x)}$ which is positive for $x > e$.
	Furthermore $\blog(2) + 1 \le 2 \le b = b\blog(b)$. 
	
	Now, because $n > 2b^{b} > e$, from the following chain of ineqalities:
	\[
		\frac{n}{\blog(n) + 1} > \frac{2b^{b}}{b\blog(b) + \blog(2) + 1} \ge 
		\frac{2b^{b}}{2b\blog(b)} = b^{b-1} > (b-1)^{b-1}
	\]
	we can deduce that $\frac{n}{\blog(n) + 1} > (b-1)^{b-1}$
\end{proof}

With both lemma's in place we can present without further ado the main result of
this article.

\begin{proposition}
	For every base $b \in \N$ with $b \ge 2$: there are only finitely many 
	Munchausen numbers in base $b$.
\end{proposition}

\begin{proof}
	By the preceding lemma's we have, for all $n \in \N$ with $n > 2b^{b}$: 
	$n > (\blog(n) + 1)(b-1)^{b-1} \ge \P_{b}(n)$.
	
	So, in order for $n$ to equal $\P_{b}(n)$, $n$ must be less then or equal to 
	$2b^{b}$. This proves that there are only finitely many Munchausen numbers
	in base $b$.
\end{proof}
