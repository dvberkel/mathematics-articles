Folklore tells us that there are no uninteresting natural numbers. The argument
hinges on the following observation: \emph{Every subset of the natural numbers
is either empty, or has a smallest element}.

The argument usually goes something like this. If there would be any 
uninteresting natural numbers, the set $\mathcal{U}$ of all these uninteresting
natural numbers would have a smallest element, say $u \in \mathcal{U}$. 
But $u$ in it self has a very remarkable property. $u$ is the smallest 
uninteresting natural number, which is very interesting indeed. So 
$\mathcal{U}$, the set of all the uninteresting natural numbers, can not have a
smallest element, therefore $\mathcal{U}$ must be empty. In other words, all 
natural numbers are interesting.

Having established this result, exhibiting an interesting property of a specific
natural number is often left as an excercise for the reader. Take for example 
the integer $3435$. At first it does not seem that remarkable, until one 
stumbles upon the following identity.
\[
	3435 = 3^{3} + 4^{4} + 3^{3} + 5^{5}
\]
This coincidence is even more remarkable when one discovers that there is only 
one other natural number which shares this property with $3435$, namely
$1 = 1^{1}$.

In this article we will establish the claim made and generalize the result.
