\section*{Exhaustive Search}

The proposition in the preceding section tells use that for every base 
$b \in \N$, Munchausen numbers in that base only occur within the interval 
$[1,2b^{b}]$. This makes it possible to exhaustively search for Munchausen 
numbers in each base.

Figure \ref{figure:munchausen} lists all the Munchausen numbers in the bases 2
through 10. So for example in base $4$, $29$ and $55$ are the only non-trivial
Munchausen numbers. Furthermore, the base $4$ representation of $29$ and $55$
have a striking resemblance. For $29 = [1,3,1]_{4} = 1^{1} + 3^{3} + 1^{1}$ and
$55 = [3,1,3]_{4} = 3^{3} + 1^{1} + 3^{3}$.

\begin{figure}[bh]
	\begin{center}
		\caption{Munchausen numbers in base 2 through 10.}
		\label{figure:munchausen}
		\begin{tabular}{|c|l|}
			\hline
			Base & Munchausen Numbers \\
			\hline
			2  & 1, 2 \\
			3  & 1, 5, 8 \\
			4  & 1, 29, 55 \\
			5  & 1 \\
			6  & 1, 3164, 3416 \\
			7  & 1, 3665 \\
			8  & 1 \\
			9  & 1, 28, 96446, 923362 \\
			10 & 1, 3435 \\
			\hline
		\end{tabular}
	\end{center}
\end{figure}

The code in listing \ref{code:munchausen} is used to produce the numbers in
figure \ref{figure:munchausen}. There are two utility functions. These are 
\lstinline!munchausen! and \lstinline!next!. \lstinline!munchausen! calculates 
$\P_{b}(n)$ given a base $b$ representation of $n$. \lstinline!next! returns the
base $b$ representation of $n+1$ given a base $b$ representation of $n$.

I would like to conclude this article with a question my wife asked me while I
was writing this: ``But what about $20082009$?''

\lstinputlisting[
	caption={GAP code finding Munchausen numbers},
	label=code:munchausen,
	basicstyle=\footnotesize
]{code/OEIS.g}

