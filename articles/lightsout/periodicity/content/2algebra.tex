\section{Periodicity}\label{sec:algebra}
In this section we reformulate the second stage of
the chasing operation in terms of matrix multiplication.
This will enable us to prove our observations in the Introduction.

Using the square $c\times c$ matrix $E_c$ from the previous
section, we define the $2c\times 2c$ matrix $W_{c}$ with entries
$0, 1, -1$ from $\Z/n\Z$ by
\[
W_{c} := \left(
\begin{matrix}
  -E_{c} & -I_{c} \\
  I_{c} & O_{c} \\
\end{matrix}
\right)
\]
where $O_c$ and $I_c$ are the $c\times c$ zero and
identity matrix.

Consider the row vector $p\oplus o=(p_1, p_2, \ldots, p_c, 0, 0, \ldots, 0)$
of length $2c$ and the result $r=(p\oplus o)\cdot W_c$: it will be clear that
$$r=(e_1, e_2, \ldots, e_c, -p_1, -p_2, \ldots, -p_c),$$
by the considerations of the previous section:
here, by definition of $W_c$,
$(e_1, e_2, \ldots, e_c)$ is the effect of
the press pattern $p=(p_1, p_2, \ldots, p_c)$. Thus, we can
interpret the result of multiplying $p\oplus o$ by $W_c$
as $e\oplus -p$: the first half is the effect on the first
row of applying the press pattern $p$, the second half is the
effect on the second row.

Suppose we now multiply this result by $W_c$ again:
$$(e\oplus -p)\cdot W_c=(eE_c-pI_c)\oplus (-eI_c-pO_c)=(eE_c-p)\oplus -e.$$
The resulting vector also has an obvious interpretation:
the first half, $eE_c+p$, is precisely the effect on
the second row of chasing the first row, while the
second half, $-e$, is the effect of this on the third row.

Repeating this we find the following result.

\begin{lemma}
For $k\geq 1$ and for any vector
$(p_1, p_2, \ldots, p_c)\in (\Z/n\Z)^c$ it holds that
$$(p_1, p_2, \ldots, p_c)\oplus (0, 0, \ldots,0)\cdot W_c^k=e_k\oplus e_{k+1},$$
where $e_k$ is the $k$-th row of chasing the effect of applying
press pattern $(p_1, p_2, \ldots, p_c)$ to the first row of a
rectangular Lights Out display of $c$ columns.
\end{lemma}
Note that we did not specify the number of rows $r$ in the rectangular
display: %the result only makes sense if $k<r$, but 
one may think of a display with $c$ columns and an arbitrary number
of rows.

%\[
%E_{4} := \left(
%\begin{array}{cccc}
%  1 & 1 & 0 & 0 \\
%  1 & 1 & 1 & 0 \\
%  0 & 1 & 1 & 1 \\
%  0 & 0 & 1 & 1 \\
%\end{array}
%\right)
%\]

The reason we are looking at $W_{c}$ is that its powers tell us something about
the effect of chasing down the lights. In particular, if we have an $(r, c, n)$
Lights Out puzzle, the $c \times c$ upper left sub-matrix of
$W_{c}^{r}$ describes exactly the process of gathering the lights
to the last row.

It is important to relate this result to what we were attempting
to achieve by chasing in the previous section. Suppose we have an
$r\times c$ rectangular board; we concluded that we could solve
a given Lights Out problem $L$ if we could find a press pattern for
the first row that when chased to the bottom row would give
$-\chase(L)=-b=-(b_1, b_2, \ldots, b_c)$, and that if such a solution
exists, the number of different solutions equals the number of
press patterns for the first row that would be chased to the zero
row at the bottom. In terms of the Lemma this means: a solution {\it exists}
if we can find a vector $p$ of length $c$ such that $(p\oplus o)\cdot W_c^r=
(-b)\oplus x$, where $x$ can be any vector in $(\Z/n\Z)^c$, and the
{\it number of solutions} equals the number of different vectors $p$
for which $(p\oplus o)\cdot W_c^r=o\oplus y$, with $y$ arbitrary.

This is summarized as follows. Here
$T_{c, r}$ is the $c\times c$ top left
sub-matrix of the power $W_c^r$ of
the $2c\times 2c$ matrix $W_c$ 
defined above.
\begin{corollary}\label{cor:numbers}
For any $r\times c$ rectangular board for Lights Out with
$n$ colours, the number of solvable initial configurations
equals the number of different vectors in the row space
of $T_{c, r}$ and each of these admits a number of solutions
that equals the number of vectors in the kernel of $T_{c, r}$.
\end{corollary}
%
The first interesting fact we prove is that $W_{c}\in\Mat_{r,c}(\znz)$
is invertible.

\begin{lemma}\label{lem:Winv}
  $W_{c}$ is invertible for all $c\in\N$.
\end{lemma}

\begin{proof}
  \[
  \left(
  \begin{matrix}
    -E_c & -I_c \\
    I_c & O_c \\
  \end{matrix}
  \right)
  \cdot
  \left(
  \begin{matrix}
    O_c & I_c  \\
    -I_c & -E_c \\
  \end{matrix}
  \right)
  =
  \left(
  \begin{matrix}
    I_c & O_c \\
    O_c & I_c \\
  \end{matrix}
  \right)
  \]
\end{proof}
%
A consequence of the invertibility of $W_{c}$ is that the sequence of its powers
is periodic, as we will see. It is also useful to relate this to the
(multiplicative) {\it order} $\ord(W_c)$ of $W_c$,
which is by definition the smallest
positive integer $k$ such that $W_c^k=I_c$.

\begin{theorem}\label{thm:Wperiod}
The sequence %of matrices 
$\left(W_{c}^{r}\right)_{r\in\N}$ is purely periodic.
\end{theorem}

\begin{proof}
  There are only finitely many different square matrices of size $2c$ over
  $\Z/n\Z$. So the sequence $\left(W_{c}^{r}\right)_{r\in\N}$ must
  become periodic. By the preceding lemma $W_{c}$ is invertible so the
  sequence is periodic from the start. If the period starts with $W_c$, it
  must end with $I_c$; on the other hand, if $W_c^j=I_c$ for some $j>0$ 
  but smaller than the period length, then the sequence would repeat 
  already from there on, a contradiction with the definition of period.
\end{proof}
%
As mentioned in \cite{leach17} there is a relation between the images of chasing
the lights and Fibonacci polynomials. We find that relation reflected in our
Structure Lemma. Note that $T_r$ in this Lemma coincides with
$T_{c,r}$ in Corollary \ref{cor:numbers}.

%For this we define, for every positive integer $c$, recursively
%the $c\times c$-matrices $T_j$ over $\Z/n\Z$ for $j\geq -1$ by 
%$$T_{-1}=O_c, T_0=I_c \quad\textrm{and}\quad T_{j+1} := -E_c \cdot T_{j} - T_{j-1}.$$

\begin{lemma}[Structure Lemma]\label{lem_struc}
  For all $c\in\N$ and for all $r\in\N$:
  \[
  W_{c}^{r}
  =
  \left(
  \begin{matrix}
     T_{r} &  -T_{r-1} \\
    T_{r-1} & -T_{r-2}   \\
  \end{matrix}
  \right),
  \]
  where the $c\times c$-matrices $T_j$ over $\Z/n\Z$ are defined for
$j\geq -1$ by the recursion $T_{-1}=O_c$, $T_0=I_c$ and $T_{j+1} := -T_j\cdot E_c - T_{j-1}.$
\end{lemma}

\begin{proof}
By definition,
$W_{c}= \left(\begin{smallmatrix} -E_c & -I_c \\ I_c & O_c \\\end{smallmatrix}\right) = \left(\begin{smallmatrix} T_{1} & -T_{0} \\ T_{0} & -T_{-1} \\\end{smallmatrix}\right)$, as required.

The result now follows by induction on the exponent $j$:
  \[
  \begin{aligned}
  W_{c}^{j+1}
  & = W_{c}^j \cdot W_{c} 
  =
  \left(
  \begin{matrix}
     T_{j} &  -T_{j-1}  \\
    T_{j-1} & -T_{j-2} \\
  \end{matrix}
  \right)
  \cdot
  \left(
  \begin{matrix}
    -E_c & -I_c \\
    I_c & O_c \\
  \end{matrix}
  \right)= \\
  & =
  \left(
  \begin{matrix}
    -T_j \cdot E_c - T_{j-1} & -T_{j}   \\
-T_{j-1}\cdot E_c - T_{j-2}  & -T_{j-1} \\
  \end{matrix}
  \right) 
  \left(
  \begin{matrix}
     T_{j+1} &  -T_{j} \\
    T_{j} & -T_{j-1} \\
  \end{matrix}
  \right).
  \end{aligned}
  \]
\end{proof}
%
