\section{Prime colours}\label{sec:prime}
With the Structure Lemma under our belt we are in a position
to prove the observations we started out with. In fact, we
can prove these observations in a generalized setting, but
only for the case where the number of colours $n$ is a
prime number. The case of a composite number of colours
requires somewhat more care, and is treated in the next section.

So, throughout this section we will assume that the
number of colours $n$ is prime; the importance
of this is that then $\znz$ is a finite field of $n$
elements, which we will denote by $\FF_n$. In this case
we can reformulate Corollary \ref{cor:numbers}.

\begin{corollary}\label{cor:prime}
For a prime number $n$ the number of solvable initial
configurations of $\Lp(r,c,n)$ will be $n^{r\cdot c-d}$, each
allowing $n^d$ different solutions, where $d=d(r,c)$ is
	$\dim\Ker T_{c,r}$.
\end{corollary}
%
The kernel dimensions $d(r,c)$ appeared in Table \ref{tab:n2} for $n=2$.
Table \ref{tab:n3} similarly lists these dimensions for Lights Out
with $n=3$ colours, for small values of $r, c$.

\begin{table}\label{tab:n3}
\begin{center}
\begin{adjustbox}{max width=\textwidth}
\begin{tabular}{|c|cccccccccccccccc|c|}
\hline
$r\ \backslash\ c$:&0 &1 &2 &3 &4 &5 &6 &7 &8 &9 &10 &11 &12 &13 &14 &15 &$\ell$
\\
\hline
\hline
 0 &0 &0 &0 &0 &0 &0 &0 &0 &0 &0 &0 &0 &0 &0 &0 &0 &1\\
 1 &0 &0 &1 &0 &0 &1 &0 &0 &1 &0 &0 &1 &0 &0 &1 &0 &3\\
 2 &0 &1 &1 &1 &0 &2 &0 &1 &1 &1 &0 &2 &0 &1 &1 &1 &6\\
 3 &0 &0 &1 &0 &2 &1 &0 &0 &1 &2 &0 &1 &0 &0 &3 &0 &15\\
 4 &0 &0 &0 &2 &2 &0 &0 &2 &0 &2 &0 &2 &0 &0 &2 &2 &20\\
 5 &0 &1 &2 &1 &0 &3 &0 &1 &4 &1 &0 &3 &0 &1 &2 &1 &18\\
 6 &0 &0 &0 &0 &0 &0 &0 &0 &0 &0 &0 &0 &3 &3 &0 &0 &182\\
 7 &0 &0 &1 &0 &2 &1 &0 &0 &1 &2 &0 &1 &0 &0 &3 &0 &120\\
 8 &0 &1 &1 &1 &0 &4 &0 &1 &4 &1 &0 &4 &0 &1 &1 &1 &18\\
 9 &0 &0 &1 &2 &2 &1 &0 &2 &1 &2 &0 &3 &0 &0 &3 &2 &2460\\
10 &0 &0 &0 &0 &0 &0 &0 &0 &0 &0 &0 &0 &0 &0 &0 &0 &122\\
11 &0 &1 &2 &1 &2 &3 &0 &1 &4 &3 &0 &3 &0 &1 &8 &1 &90\\
12 &0 &0 &0 &0 &0 &0 &3 &0 &0 &0 &0 &0 &6 &6 &0 &0 &182\\
13 &0 &0 &1 &0 &0 &1 &3 &0 &1 &0 &0 &1 &6 &6 &1 &0 &546\\
14 &0 &1 &1 &3 &2 &2 &0 &3 &1 &3 &0 &8 &0 &1 &7 &3 &60\\
15 &0 &0 &1 &0 &2 &1 &0 &0 &1 &2 &0 &1 &0 &0 &3 &0 &9840\\
16 &0 &0 &0 &0 &0 &0 &0 &0 &0 &0 &0 &0 &0 &0 &0 &0 &672605\\
17 &0 &1 &2 &1 &0 &5 &0 &1 &8 &1 &0 &5 &0 &1 &2 &1 &54\\
18 &0 &0 &0 &0 &0 &0 &0 &0 &0 &0 &0 &0 &0 &0 &0 &0 &5097638\\
19 &0 &0 &1 &2 &4 &1 &0 &2 &1 &4 &0 &3 &0 &0 &5 &2 &2460\\
20 &0 &1 &1 &1 &0 &2 &0 &1 &1 &1 &0 &2 &3 &4 &1 &1 &546\\
21 &0 &0 &1 &0 &0 &1 &0 &0 &1 &0 &0 &1 &0 &0 &1 &0 &44286\\
22 &0 &0 &0 &0 &0 &0 &0 &0 &0 &0 &0 &0 &0 &0 &0 &0 &88573\\
23 &0 &1 &2 &1 &2 &3 &0 &1 &4 &3 &0 &3 &0 &1 &8 &1 &360\\
24 &0 &0 &0 &2 &2 &0 &0 &2 &0 &2 &0 &2 &0 &0 &2 &2 &174339220\\
25 &0 &0 &1 &0 &0 &1 &3 &0 &1 &0 &0 &1 &6 &6 &1 &0 &546\\
26 &0 &1 &1 &1 &0 &4 &0 &1 &4 &1 &0 &4 &0 &1 &1 &1 &54\\
27 &0 &0 &1 &0 &2 &1 &3 &0 &1 &2 &0 &1 &6 &6 &3 &0 &199290\\
28 &0 &0 &0 &0 &0 &0 &0 &0 &0 &0 &0 &0 &0 &0 &0 &0 &5719198113740\\
29 &0 &1 &2 &3 &2 &3 &0 &3 &4 &3 &0 &9 &0 &1 &8 &3 &7380\\
30 &0 &0 &0 &0 &0 &0 &0 &0 &0 &0 &0 &0 &0 &0 &0 &0 &51472783023662\\
31 &0 &0 &1 &0 &2 &1 &0 &0 &1 &2 &0 &1 &0 &0 &3 &0 &64570080\\
32 &0 &1 &1 &1 &0 &2 &0 &1 &1 &1 &0 &2 &0 &1 &1 &1 &366\\
\hline
\end{tabular}
\end{adjustbox}
\end{center}
\caption{Dimension of kernels, and length of period $\ell$ for $n=3$}
\end{table}

\begin{proposition}[Observation 1]
Sequence $\left(d_{c}(r)\right)_{r\in\N}$ is purely periodic.
\end{proposition}

\begin{proof}
By the Structure Lemma, Corollary \ref{cor:prime},
the number of different solutions is determined by 
$\dim\Ker T_{c,r}$.
  The periodicity of $(W_{c}^{r})_{r\in\N}$ immediately implies that
  periodicity of $(T_{c,r})_{r\in\N}$.
\end{proof}
Note that the Proposition does not mention the {\it size} of either the period
of the sequence $(W_{c}^{r})_{r\in\N}$ or that of $(T_{r})_{r\in\N}$
and hence of $\left(d_{c}(r)\right)_{r\in\N}$ yet. But it is clear that
the period of $\left(d_{c}(r)\right)_{r\in\N}$
will be a divisor of the period of $(W_{c}^{r})_{r\in\N}$. 
We will clarify the situation shortly.

Next on our agenda is our Observation 2, i.e., for
every number of rows there will be a kernel of maximal dimension.

\begin{lemma}[Observation 2]\label{lem:max}
% There is a kernel of maximal dimension.
For any given number of columns $c$ there will be be a number of rows $r$
such that $d_c(r)=c$.
\end{lemma}

\begin{proof}
  We will be using the notation as defined by the Structure Lemma.

  There exists $p\in\N$ such that $W_{c}^{p} = I$.
  Then $W_{c}^{p-1} = W_{c}^{-1} \cdot W_{c}^{p} = W_{c}^{-1}$.
  Hence,
  \[
  d_{c}(p)
  =
  \dim\Ker T_{-1}
  =
  \dim\Ker O_{c}
  = c.
  \]
\end{proof}
Before we investigate the period lengths further,
we will take a closer look at Observation 4.

\begin{theorem}[Observation 4]\label{thm:consec}
  For all $i\in\N$ the following inequality holds:
  \[d_{c}(i) + d_{c}(i+1) \leq c.\]
\end{theorem}
\begin{proof}
  Assume, to the contrary, that $d_{c}(k) + d_{c}(k+1) > c$ for some $k\in\N$.

  This means that there exist $s=d_{c}(k)$ independent press vectors
  $q^{(1)}, q^{(2)}, \ldots, q^{(m)}$ in the kernel of $T_k$ and
  $t = d_{c}(k+1)$ independent press vectors
  $r^{(1)}, r^{(2)}, \ldots, r^{(n)}$ in the kernel of $T_{k+1}$.
  Since $s+t>c$ there must be a non-trivial press vector $p$ in the intersection
  of both kernels. Then $(p\oplus o)\cdot W_c^{k} = o\oplus w$ for
  some vector $w$, and $(p\oplus o)\cdot W_c^{k+1}=o\oplus w'$ for some $w'$.
  But 
  \[ o\oplus w'=(p\oplus o)\cdot W_c^{k+1}=(p\oplus o)\cdot W_c^{k}\cdot W_c=(o\oplus w)\cdot W_c=
   -w\oplus o\]
  and thus $w=o=w'$ and $p\oplus o$ is a non-trivial vector in the kernel
  of $W_c^k$, which contradicts the invertibility of $W_c$.
\end{proof}

\begin{corollary}\label{cor:max}
  If for some $r\in\N$ we have $d_{c}(r) = c$ then
  \begin{itemize}
    \item $d_{c}(r-1) = d_{c}(r+1) = 0$, and
    \item $W_{c}^{r+1} = \left(\begin{smallmatrix} -T_{r-1} & O \\
                                  O & -T_{r-1} \\\end{smallmatrix}\right)$.
  \end{itemize}
\end{corollary}

\begin{proof}
The first part is immediate by Theorem \ref{thm:consec}.

For the second statement, observe that $d_c(r)=c$ can only happen
when $T_r=O$, so
  $$W_{c}^{r+1} = W_c\cdot W_c^r=
\left(\begin{matrix} -E_{c} & I_c \\ -I_{c} & O_c \\\end{matrix}\right)\cdot
\left(\begin{matrix} O_c & T_{r-1} \\ -T_{r-1} & -T_{r-2} \\\end{matrix}\right)=
\left(\begin{matrix} -T_{r-1} & O_c \\ O_c & -T_{r-1} \\\end{matrix}
\right).$$
\end{proof}
In proving our Corollary \ref{cor:max} we have learned that the
structure of the corresponding matrix power is particularly simple. 
This fact will
be instrumental in the relation between the period of
$\left(W_{c}^{r}\right)_{r\in\N}$ and that of
$\left(d_{c}(r)\right)_{r\in\N}$.

Note that conjugating any $2c\times 2c$ matrix $M=
\left(\begin{smallmatrix} A & B \\ C & D \\\end{smallmatrix}\right)$ by
$Z=\left(\begin{smallmatrix} O_c & I_c \\ I_c & O_c \\\end{smallmatrix}\right)$ 
corresponds to rotating the four main sub-matrices:
\begin{equation}\label{eq:rotate}
Z\cdot M\cdot Z^{-1}=\left(\begin{matrix} O_c & I_c \\ I_c & O_c \\\end{matrix}\right)
\cdot
\left(\begin{matrix} A & B \\ C & D \\\end{matrix}\right)
\cdot
\left(\begin{matrix} O_c & I_c \\ I_c & O_c \\\end{matrix}\right)
=
\left(\begin{matrix} D & C \\ B & A \\\end{matrix}\right).\end{equation}
This fact will be the linchpin in the proof of Observation 5.

But first we will see that $W_{c}$ and its inverse are conjugates.
\begin{lemma}
  $W_{c}^r$ and $W_{c}^{-r}$ are conjugate, for all $r\geq 1$.
\end{lemma}

\begin{proof}
  Omitting the subscripts $c$, we conjugate $W$ by
  $Z := \left(\begin{smallmatrix} O & I \\ I & O \\\end{smallmatrix}\right)$,
  which is its own inverse:
  \[
  \begin{aligned}
  Z \cdot W \cdot Z^{-1}
  & =
  \left(
  \begin{array}{cc}
    O & I \\
    I & O \\
  \end{array}
  \right)
  \cdot
  \left(
  \begin{array}{cc}
    -E & I \\
    -I & O \\
  \end{array}
  \right)
  \cdot
  \left(
  \begin{array}{cc}
    O & I \\
    I & O \\
  \end{array}
  \right)\\
  & =
  \left(
  \begin{array}{cc}
    -I & O \\
    -E & I \\
  \end{array}
  \right)
  \cdot
  \left(
  \begin{array}{cc}
    O & I \\
    I & O \\
  \end{array}
  \right) 
   =
  \left(
  \begin{array}{cc}
    O & -I \\
    I & -E \\
  \end{array}
  \right) 
   =
  W^{-1}
  \end{aligned}
  \]
from which the general result follows immediately by taking $r$-th powers.
\end{proof}

\begin{lemma}\label{mirror}
  If for some $r\in\N$ we have $W_{c}^{r} = \left(\begin{smallmatrix} Q & O \\ O & Q \\\end{smallmatrix}\right)$ then for all $i\in\N$
  \[
  W_{c}^{r+i} = \left(\begin{array}{cc} QT_{i} & QT_{i-1} \\ -QT_{i-1} & -QT_{i-2} \\\end{array}\right)
  \]

  and

  \[
  W_{c}^{r-i} = \left(\begin{array}{cc} -QT_{i-2} & -QT_{i-1} \\ QT_{i-1} & QT_{i} \\\end{array}\right)
  \]
\end{lemma}

\begin{proof}
  By direct calculation and the Structure Lemma we find for $W_{c}^{r+i}$
  \[
% W_{c}^{r+i}
% =
  W_{c}^{r}W_{c}^{i}
  =
  \left(
  \begin{array}{cc}
    Q & O \\
    O & Q \\
  \end{array}
  \right)
  \left(
  \begin{array}{cc}
     T_{i} &  T_{i-1}  \\
    -T_{i-1} & -T_{i-2} \\
  \end{array}
  \right)
  =
  \left(
  \begin{array}{cc}
     QT_{i} &  QT_{i-1}  \\
    -QT_{i-1} & -QT_{i-2} \\
  \end{array}
  \right)
  \]
  Furthermore, with $Z=\left(\begin{smallmatrix} O & I \\ I & O \\\end{smallmatrix}\right)$
  we have, by the preceding lemma,
  \[Z\cdot W_{c}^{r-i}\cdot Z^{-1}=Z\cdot W_{c}^{r}\cdot Z^{-1}\cdot Z\cdot W^{-i}\cdot Z^{-1}=W_{c}^{r}\cdot W_{c}^{i}=W_{c}^{r+i}\]
  which, combined with Equation \ref{eq:rotate}, yields what we set out to prove.
\end{proof}
We now get to the relation between the order $p$ of $W$ and the period of
$d_c$. Let $r_0$ be the smallest positive $r$ for which the maximal
dimension of the kernel occurs (which exists, according to Lemma
\ref{lem:max}); so $d_c(r_0)=c$.
\begin{lemma}\label{lem:half}
  Either $p=r_0+1$ or $p=2(r_0+1)$.
\end{lemma}

\begin{proof}
  Let us write $q=r_0+1$ in this proof.

  If $p=q$ we are finished, so assume it is not. We will show that $p=2q$ in
  that case.

  By Corollary \ref{cor:max} we know that 
  $W^{q}=\left(\begin{smallmatrix} Q & O \\ O & Q \\\end{smallmatrix}\right)$
  for a certain matrix $Q$, and by the preceding lemma, for all $i\geq 1$,
  the lower right $c\times c$ sub-matrix of
  $W_{c}^{q-i}$ equals the upper left $c\times c$ sub-matrix of $W_{c}^{q+i}$
  for all $i$. In particular, with $i=q$ we obtain that
  the upper left sub-matrix of $W_{c}^{q+q}$ is the lower right sub-matrix of
  $W_{c}^{q-q} = W_{c}^{0} = I_{c}$, the $c\times c$ identity matrix.
  Now
  \[
  W_{c}^{2q}
  =
  \left(
  \begin{array}{cc}
    Q & O \\
    O & Q \\
  \end{array}
  \right)^{2}
  =
  \left(
  \begin{array}{cc}
    Q^{2} & O \\
    O & Q^{2} \\
  \end{array}
  \right)
  \]
  so $Q^{2}=I_{c}$ and $W^{2q}=I_{2c}$
  
  Since we assumed thatHence the order of $W$ equals $2q$.
  is the identity matrix, and the period of $(W_{c}^{r})_{r\in\N}$ is $2d$.
\end{proof}

\begin{theorem}[Observation 5]
  The sequence $\left(d_{c}(r)\right)_{r\in\N}$
  is almost palindromic.
\end{theorem}

\begin{proof}
  Let $d$ be such that $W^d$ has a upper left sub-matrix $O$.

  A \emph{press vector} is a $2\times c$ vector with the last $c$ components zero, an \emph{unlit vector} is a $2\times c$ vector with the first $c$ components 0. Notice that for any press vector $v$

  \[
    v\cdot W^d
  \]

  is an unlit vector.

  Choose $m,n\in\N$ such that $m + 1 + n = d$. We will show that for each press vector $v$ for which $v\cdot W^m$ is unlit, there exist a press vector $v'$ such that $v'\cdot W^n$ is unlit.
  This shows that $\dim\Ker T_{m} \leq \dim\Ker T_{n}$. Since the argument is symmetric in $m$ and $n$ we have $\dim\Ker T_{m} = \dim\Ker T_{n}$.

	Let $v$ be a press vector such that $v\cdot W^{m}$ is unlit. In particular $v\cdot W^{m} = u$ with $u := (0,0,\ldots, 0)\oplus (u_{1}, u_{2}, \ldots, u_{c})$.

  Define $p := u\cdot W$. We will show that $p$ is a press vector.

  \[
    u\cdot W = \left(
    \begin{array}{cc}
      -E_{c} & I \\
      -I    & O \\
    \end{array}
    \right)
    \left(
    \begin{array}{c}
      O  \\
      u' \\
    \end{array}
    \right)
    =
    \left(
    \begin{array}{c}
      u' \\
      O  \\
    \end{array}
    \right)
  \]

  Since for any press vector $w$ we have that $w\cdot W^{d} = O$, in particular we have

  \[
  O = v\cdot W^{d} = v\cdot W^{n} W W^{m} = u\cdot W^{n} W = p\cdot W^{n}
  \]

  which shows that for each press vector that $W^{m}$ unlits, there is a press vector that $W^{n}$ unlits.
\end{proof}

Now everything is in play to prove Observation 4.

\begin{theorem}[Observation 4]
	The period of $(d_{c}(r))_{r\in\N}$ starts after the maximal dimension.
\end{theorem}

\begin{proof}
By Lemma \ref{cor:max} the power of $W$ has a block diagonal structure. By Lemma \ref{mirror} the sequence of dimensions
is shifted mirror image. Together with the fact that the sequence is almost palindromic from the previous theorem, we have our proof.
\end{proof}
