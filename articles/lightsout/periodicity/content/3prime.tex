\section{Prime colours}\label{sec:prime}
With the Structure Lemma under our belt we are in a position
to prove the observations we started out with. In fact, we
can prove these observations in a generalized setting, but
only for the case where the number of colours $n$ is a
prime number. The case of a composite number of colours
requires somewhat more care, and is treated in the next section.

So, throughout this section we will assume that the
number of colours $n$ is prime; the importance
of this is that then $\znz$ is a finite field of $n$
elements, which we will denote by $\FF_n$. In this case
we can reformulate Corollary \ref{cor:numbers}.

\begin{corollary}\label{cor:prime}
For a prime number $n$ the number of solvable initial
configurations of $\Lp(r,c,n)$ will be $n^{r\cdot c-d}$, each
allowing $n^d$ different solutions, where $d=d(r,c)$ is
	$\dim\Ker T_{c,r}$.
\end{corollary}
%
The kernel dimensions $d(r,c)$ appeared in Table \ref{tab:n2} for $n=2$.
Table \ref{tab:n3} similarly lists these dimensions for Lights Out
with $n=3$ colours, for small values of $r, c$.

\begin{table}\label{tab:n3}
\begin{center}
\begin{adjustbox}{max width=\textwidth}
\begin{tabular}{|c|cccccccccccccccc|c|}
\hline
$r\ \backslash\ c$:&0 &1 &2 &3 &4 &5 &6 &7 &8 &9 &10 &11 &12 &13 &14 &15 &$\ell$
\\
\hline
\hline
 0 &0 &0 &0 &0 &0 &0 &0 &0 &0 &0 &0 &0 &0 &0 &0 &0 &1\\
 1 &0 &0 &1 &0 &0 &1 &0 &0 &1 &0 &0 &1 &0 &0 &1 &0 &3\\
 2 &0 &1 &1 &1 &0 &2 &0 &1 &1 &1 &0 &2 &0 &1 &1 &1 &6\\
 3 &0 &0 &1 &0 &2 &1 &0 &0 &1 &2 &0 &1 &0 &0 &3 &0 &15\\
 4 &0 &0 &0 &2 &2 &0 &0 &2 &0 &2 &0 &2 &0 &0 &2 &2 &20\\
 5 &0 &1 &2 &1 &0 &3 &0 &1 &4 &1 &0 &3 &0 &1 &2 &1 &18\\
 6 &0 &0 &0 &0 &0 &0 &0 &0 &0 &0 &0 &0 &3 &3 &0 &0 &182\\
 7 &0 &0 &1 &0 &2 &1 &0 &0 &1 &2 &0 &1 &0 &0 &3 &0 &120\\
 8 &0 &1 &1 &1 &0 &4 &0 &1 &4 &1 &0 &4 &0 &1 &1 &1 &18\\
 9 &0 &0 &1 &2 &2 &1 &0 &2 &1 &2 &0 &3 &0 &0 &3 &2 &2460\\
10 &0 &0 &0 &0 &0 &0 &0 &0 &0 &0 &0 &0 &0 &0 &0 &0 &122\\
11 &0 &1 &2 &1 &2 &3 &0 &1 &4 &3 &0 &3 &0 &1 &8 &1 &90\\
12 &0 &0 &0 &0 &0 &0 &3 &0 &0 &0 &0 &0 &6 &6 &0 &0 &182\\
13 &0 &0 &1 &0 &0 &1 &3 &0 &1 &0 &0 &1 &6 &6 &1 &0 &546\\
14 &0 &1 &1 &3 &2 &2 &0 &3 &1 &3 &0 &8 &0 &1 &7 &3 &60\\
15 &0 &0 &1 &0 &2 &1 &0 &0 &1 &2 &0 &1 &0 &0 &3 &0 &9840\\
16 &0 &0 &0 &0 &0 &0 &0 &0 &0 &0 &0 &0 &0 &0 &0 &0 &672605\\
17 &0 &1 &2 &1 &0 &5 &0 &1 &8 &1 &0 &5 &0 &1 &2 &1 &54\\
18 &0 &0 &0 &0 &0 &0 &0 &0 &0 &0 &0 &0 &0 &0 &0 &0 &5097638\\
19 &0 &0 &1 &2 &4 &1 &0 &2 &1 &4 &0 &3 &0 &0 &5 &2 &2460\\
20 &0 &1 &1 &1 &0 &2 &0 &1 &1 &1 &0 &2 &3 &4 &1 &1 &546\\
21 &0 &0 &1 &0 &0 &1 &0 &0 &1 &0 &0 &1 &0 &0 &1 &0 &44286\\
22 &0 &0 &0 &0 &0 &0 &0 &0 &0 &0 &0 &0 &0 &0 &0 &0 &88573\\
23 &0 &1 &2 &1 &2 &3 &0 &1 &4 &3 &0 &3 &0 &1 &8 &1 &360\\
24 &0 &0 &0 &2 &2 &0 &0 &2 &0 &2 &0 &2 &0 &0 &2 &2 &174339220\\
25 &0 &0 &1 &0 &0 &1 &3 &0 &1 &0 &0 &1 &6 &6 &1 &0 &546\\
26 &0 &1 &1 &1 &0 &4 &0 &1 &4 &1 &0 &4 &0 &1 &1 &1 &54\\
27 &0 &0 &1 &0 &2 &1 &3 &0 &1 &2 &0 &1 &6 &6 &3 &0 &199290\\
28 &0 &0 &0 &0 &0 &0 &0 &0 &0 &0 &0 &0 &0 &0 &0 &0 &5719198113740\\
29 &0 &1 &2 &3 &2 &3 &0 &3 &4 &3 &0 &9 &0 &1 &8 &3 &7380\\
30 &0 &0 &0 &0 &0 &0 &0 &0 &0 &0 &0 &0 &0 &0 &0 &0 &51472783023662\\
31 &0 &0 &1 &0 &2 &1 &0 &0 &1 &2 &0 &1 &0 &0 &3 &0 &64570080\\
32 &0 &1 &1 &1 &0 &2 &0 &1 &1 &1 &0 &2 &0 &1 &1 &1 &366\\
\hline
\end{tabular}
\end{adjustbox}
\end{center}
\caption{Dimension of kernels, and length of period $\ell$ for $n=3$}
\end{table}

\begin{proposition}[Observation 1]
Sequence $\left(d_{c}(r)\right)_{r\in\N}$ is purely periodic.
\end{proposition}

\begin{proof}
By the Structure Lemma, Corollary \ref{cor:prime},
the number of different solutions is determined by 
$\dim\Ker T_{c,r}$.
  Since $(W_{c}^{r})_{r\in\N}$ is purely periodic, so is
  $(T_{c,r})_{r\in\N}$.
\end{proof}
Note that the Proposition does not mention the {\it size} of either the period
of the sequence $(W_{c}^{r})_{r\in\N}$ or that of $(T_{r})_{r\in\N}$
and hence of $\left(d_{c}(r)\right)_{r\in\N}$ yet. But it is clear that
the period of $\left(d_{c}(r)\right)_{r\in\N}$
will be a divisor of the period of $(W_{c}^{r})_{r\in\N}$. 
We will clarify the situation shortly.

Next on our agenda is our Observation 2, i.e., for
every number of rows there will be a kernel of maximal dimension.

\begin{lemma}[Observation 2]\label{lem:max}
% There is a kernel of maximal dimension.
For any given number of columns $c$ there will be be a number of rows $r$
such that $d_c(r)=c$.
\end{lemma}

\begin{proof}
  We will be using the notation as defined by the Structure Lemma.

  There exists $p\in\N$ such that $W_{c}^{p} = I$.
  Then $W_{c}^{p-1} = W_{c}^{-1} \cdot W_{c}^{p} = W_{c}^{-1}$.
  Hence,
  \[
  d_{c}(p)
  =
  \dim\Ker T_{-1}
  =
  \dim\Ker O_c
  = c.
  \]
\end{proof}
By $r_0$ we will denote be the smallest positive $r$ for which the maximal
dimension of the kernel occurs.
Before we investigate the period lengths further,
we will take a closer look at Observation 4.

\begin{theorem}[Observation 4]\label{thm:consec}
  For all $r\in\N$ the following inequality holds:
  \[d_{c}(r) + d_{c}(r+1) \leq c.\]
\end{theorem}
\begin{proof}
  Assume, to the contrary, that $d_{c}(k) + d_{c}(k+1) > c$ for some $k\in\N$.

  This means that there exist $m=d_{c}(k)$ independent press vectors
  %$q^{(1)}, q^{(2)}, \ldots, q^{(m)}$ 
  in the kernel of $T_k$ and
  $n = d_{c}(k+1)$ independent press vectors
  %$r^{(1)}, r^{(2)}, \ldots, r^{(n)}$ 
  in the kernel of $T_{k+1}$.
  Since $m+n>c$ there must be a non-trivial press vector $p$ in the intersection
  of both kernels. Then $(p\oplus o)\cdot W_c^{k} = o\oplus w$ for
  some vector $w$, and $(p\oplus o)\cdot W_c^{k+1}=o\oplus w'$ for some $w'$.
  But 
  \[ o\oplus w'=(p\oplus o)\cdot W_c^{k+1}=(p\oplus o)\cdot W_c^{k}\cdot W_c=(o\oplus w)\cdot W_c=
   w\oplus o\]
  and thus $w=o=w'$ and $p\oplus o$ is a non-trivial vector in the kernel
  of $W_c^k$, which contradicts the invertibility of $W_c$.
\end{proof}

\begin{corollary}\label{cor:max}
  If for some $r\geq 1$ we have $d_{c}(r) = c$ then
  \begin{itemize}
    \item $d_{c}(r-1) = d_{c}(r+1) = 0$, and
    \item $W_{c}^{r+1} = \left(\begin{matrix} -T_{r-1} & O_c \\
                                  O_c & -T_{r-1} \\\end{matrix}\right)$.
  \end{itemize}
	In particular: $d_c(r_0-1)=d_c(r_0+1)=0$.
\end{corollary}

\begin{proof}
The first part is immediate by Theorem \ref{thm:consec}.

For the second statement, observe that $d_c(r)=c$ can only happen
when $T_r=O_c$, so
  $$W_{c}^{r+1} = W_c^r\cdot W_c=
\left(\begin{matrix} O_c & -T_{r-1} \\ T_{r-1} & -T_{r-2}\end{matrix}\right)\cdot
\left(\begin{matrix} -E_{c} & -I_c \\ I_{c} & O_c \\\end{matrix}\right)=
\left(\begin{matrix} -T_{r-1} & O_c \\ O_c & -T_{r-1} \\\end{matrix}
\right).$$
The final statement follows from the definition of $r_0$.
\end{proof}
\begin{corollary}
The period of sequence $(d_c(r))_{r\in\N}$ is $r_0+1$.
\end{corollary}
\begin{proof}
We have already seen that the sequence is purely periodic; by definition
	$d_c(r)=\dim\ker T_r$ for $r\geq 0$. Here $T_r$ is the upper left
	$c\times c$ submatrix of $W_c^r$. We also saw that
	$(W_c^r)_{r\in\N}$ forms a purely periodic sequence in 
	Theorem \ref{thm:Wperiod}, that can in fact be
	extended to a {\it bi-infinite} purely periodic
	sequence $(W_c^r)_{r\in\Z}$, since $W_c$ is invertible by 
	Lemma \ref{lem:Winv}.
	This determines the bi-infinite sequence $(T_r)_{r\in\Z}$, for which
	we have a degree two recursion given in Lemma \ref{lem_struc}; 
	this implies that the minimal
	period for $(d_c(r))_{r\in\N}$ is reached whenever two consecutive 
	values repeat. Now $d_c(-1)=c$ as can be seen from the inverse of $W_c$
	in Lemma \ref{lem:Winv}, so $d_c(-1)=d_c(r_0)=c$ and 
	$d_c(0)=d_c(r_0+1)=0$,
	and since $r_0+1$ is the smallest positive integer for which $d_c(r)=0$,
	the period must be equal to $r_0+1$.
\end{proof}

\begin{theorem}[Observation 4]
The period of $(d_{c}(r))_{r\in\N}$ ends as soon as
	the maximal dimension appears.
\end{theorem}

%\begin{proof}
%By Lemma \ref{cor:max} the power of $W$ has a block diagonal structure. By Lemma \ref{mirror} the sequence of dimensions
%is shifted mirror image. Together with the fact that the sequence is almost palindromic from the previous theorem, we have our proof.
%\end{proof}
Note that conjugating any $2c\times 2c$ matrix $M=
\left(\begin{smallmatrix} A & B \\ C & D \\\end{smallmatrix}\right)$ by
$Z=\left(\begin{smallmatrix} O_c & I_c \\ I_c & O_c \\\end{smallmatrix}\right)$ 
corresponds to rotating the four main sub-matrices:
\begin{equation}\label{eq:rotate}
Z\cdot M\cdot Z^{-1}=\left(\begin{matrix} O_c & I_c \\ I_c & O_c \\\end{matrix}\right)
\cdot
\left(\begin{matrix} A & B \\ C & D \\\end{matrix}\right)
\cdot
\left(\begin{matrix} O_c & I_c \\ I_c & O_c \\\end{matrix}\right)
=
\left(\begin{matrix} D & C \\ B & A \\\end{matrix}\right).\end{equation}
This fact will be the linchpin in the proof of Observation 5.

But first we will see that a power of $W_{c}$ and its inverse are conjugate.
\begin{lemma}
  $W_{c}^r$ and $W_{c}^{-r}$ are conjugate, for all $r\in\Z$.
\end{lemma}

\begin{proof}
  Omitting the subscripts $c$, we conjugate $W$ by
  $Z$:
% which is its own inverse:
  \[
% \begin{aligned}
  Z \cdot W \cdot Z^{-1}
  =
	  \left(\begin{matrix}
		  O & I \\ I & O\\
	  \end{matrix}\right)
	  \cdot
	  \left(\begin{matrix}
		  -E & -I \\ I & O\\
	  \end{matrix}\right)
	  \cdot
	  \left(\begin{matrix}
		  O & I \\ I & O\\
	  \end{matrix}\right)
	  =
	  \left(\begin{matrix}
		  O & I \\ -I & -E\\
	  \end{matrix}\right)
	  = W^{-1}.
%  \left(
%  \begin{array}{cc}
%    O & I \\
%    I & O \\
%  \end{array}
%  \right)
%  \cdot
%  \left(
%  \begin{array}{cc}
%    -E & I \\
%    -I & O \\
%  \end{array}
%  \right)
%  \cdot
%  \left(
%  \begin{array}{cc}
%    O & I \\
%    I & O \\
%  \end{array}
%  \right)\\
%  & =
%  \left(
%  \begin{array}{cc}
%    -I & O \\
%    -E & I \\
%  \end{array}
%  \right)
%  \cdot
%  \left(
%  \begin{array}{cc}
%    O & I \\
%    I & O \\
%  \end{array}
%  \right) 
%   =
%  \left(
%  \begin{array}{cc}
%    O & -I \\
%    I & -E \\
%  \end{array}
%  \right) 
%   =
%  W^{-1}
%  \end{aligned}
  \]
from which the result follows immediately by taking $r$-th powers.
\end{proof}

\begin{lemma}\label{mirror}
If $W_{c}^{r} = \left(\begin{matrix} Q & O \\ O & Q \\\end{matrix}\right)$
for some $\in\N$ and some $c\times c$ matrix $Q$, then for all $i\in\N$:
  \[
  W_{c}^{r+i} = \left(\begin{matrix}QT_{i} & -QT_{i-1} \\ QT_{i-1} & -QT_{i-2} \\\end{matrix}\right)
	  \textrm{\ \ and\ \ }
  W_{c}^{r-i} = \left(\begin{matrix} -QT_{i-2} & QT_{i-1} \\ -QT_{i-1} & QT_{i} \\\end{matrix}\right).
  \]
\end{lemma}

\begin{proof}
Note that $Z\cdot W_c^{i}\cdot Z^{-1}=W_c^{-i}$ for all $i$
by the previous lemma, and that
$Z\cdot W_c^r\cdot Z^{-1}=W_c^r$ if $W_c^r=
  \left(\begin{matrix} Q & O \\ O & Q \\\end{matrix}\right)$.
  Hence
  \[Z\cdot W_{c}^{r+i}\cdot Z^{-1}=Z\cdot W_{c}^{r}\cdot Z^{-1}\cdot Z\cdot W^{i}\cdot Z^{-1}=W_{c}^{r}\cdot W_{c}^{-i}=W_{c}^{r-i}.\]
  By direct calculation and the Structure Lemma we find
  \[
 W_{c}^{r+i}
 =
  W_{c}^{r}W_{c}^{i}
  =
  \left(
\begin{matrix}
    Q & O \\
    O & Q \\
  \end{matrix}
  \right)
  \left(
  \begin{matrix}
     T_{i} &  -T_{i-1}  \\
    T_{i-1} & -T_{i-2} \\
  \end{matrix}
  \right)
  =
  \left(
  \begin{matrix}
     QT_{i} &  -QT_{i-1}  \\
    QT_{i-1} & -QT_{i-2} \\
  \end{matrix}
  \right),
  \]
  which, combined with Equation \ref{eq:rotate}, yields what we set out to prove.
\end{proof}
Recall from the Structure Lemma the notation $T_r$ for upper left $c\times c$
submatrix of $W_c^r$.
\begin{corollary}\label{cor:palin} 
If $W_c^r=\left(\begin{matrix}Q&O\\O&Q\end{matrix}\right)$ then for every $i$:
	$$T_{r-1+i}=T_{r-1-i} \textrm{\ \ and\ \ }
d_c(r-1+i)=d_c(r-1-i).$$ In particular, $d_c(r_0+i)=d_c(r_0-i)$.
\end{corollary}
\begin{proof}
	By the preceding lemma we see that the upper left $c\times c$
	submatrix of $W^{r+i}$, which is by definition $T_{r+i}$, equals
	the lower right  $c\times c$ submatrix of $W^{r-i}$, which
	equals $T_{r-i-2}$ by the Structure Lemma, and which is the upper
	left submatrix of $W^{r-i-2}$ by definition. Hence 
	$T_{r-1+(i+1)}=T_{r-1-(i+1)}$ for all $i$ for which both 
	sides are defined. The second statement follows as $d_c(k)$
	is the dimension of the kernel of $T_k$. 
	The final statement follows from Corollary \ref{cor:max} and taking $r=r_0+1$.
\end{proof}

\begin{theorem}[Observation 5]
  The the period of the sequence $\left(d_{c}(r)\right)_{r\in\N}$
  is almost palindromic: the period consist of a palindrome followed by the value $c$.
\end{theorem}

\begin{proof}
	By the last part of Corollary \ref{cor:palin} the values
	preceding $d_c(r_0)=c$ which form the final part of the first
	period, mirror the values following it.
\end{proof}
%  Let $d$ be such that $W^d$ has a upper left sub-matrix $O$.
%
%  A \emph{press vector} is a $2\times c$ vector with the last $c$ components zero, an \emph{unlit vector} is a $2\times c$ vector with the first $c$ components 0. Notice that for any press vector $v$
%
%  \[
%    v\cdot W^d
%  \]
%
%  is an unlit vector.
%
%  Choose $m,n\in\N$ such that $m + 1 + n = d$. We will show that for each press vector $v$ for which $v\cdot W^m$ is unlit, there exist a press vector $v'$ such that $v'\cdot W^n$ is unlit.
%  This shows that $\dim\Ker T_{m} \leq \dim\Ker T_{n}$. Since the argument is symmetric in $m$ and $n$ we have $\dim\Ker T_{m} = \dim\Ker T_{n}$.
%
%	Let $v$ be a press vector such that $v\cdot W^{m}$ is unlit. In particular $v\cdot W^{m} = u$ with $u := (0,0,\ldots, 0)\oplus (u_{1}, u_{2}, \ldots, u_{c})$.
%
%  Define $p := u\cdot W$. We will show that $p$ is a press vector.
%
%  \[
%    u\cdot W = \left(
%    \begin{array}{cc}
%      -E_{c} & I \\
%      -I    & O \\
%    \end{array}
%    \right)
%    \left(
%    \begin{array}{c}
%      O  \\
%      u' \\
%    \end{array}
%    \right)
%    =
%    \left(
%    \begin{array}{c}
%      u' \\
%      O  \\
%    \end{array}
%    \right)
%  \]
%
%  Since for any press vector $w$ we have that $w\cdot W^{d} = O$, in particular we have
%
%  \[
%  O = v\cdot W^{d} = v\cdot W^{n} W W^{m} = u\cdot W^{n} W = p\cdot W^{n}
%  \]
%
%  which shows that for each press vector that $W^{m}$ unlits, there is a press vector that $W^{n}$ unlits.
%\end{proof}
Finally we clear up the relation between the order of $W$ and the period length $r_0+1$ of $d_c$. 
\begin{lemma}\label{lem:half}
	Either $\ord W_c =r_0+1$ or $\ord W_c=2(r_0+1)$. 
\end{lemma}

\begin{proof}
  Let us write $p=\ord W_c$ and $q=r_0+1$ in this proof.

By the definition of $d_c$ its period length divides the order of $W_c$.

  By Corollary \ref{cor:max} we know that 
  $W^{q}=\left(\begin{smallmatrix} Q & O \\ O & Q \\\end{smallmatrix}\right)$
	  for a certain matrix $Q$, and by Lemma \ref{mirror}
	  $W_c^{2q}=W_c^0=I_{2c}$.
  Then
  \[
  I_{2c}=W_{c}^{2q}
  =
  \left(\begin{matrix} Q & O \\ O & Q \\ \end{matrix} \right)^{2}
  =
  \left(\begin{matrix} Q^{2} & O \\ O & Q^{2} \\ \end{matrix} \right)
  \]
  so $Q^{2}=I_c$ and the order $p$ of $W$ equals $q$ if $Q=I_c$
and equals $2q$ otherwise.
\end{proof}

