\section{Introduction}

Introduction to Lights Out and the chasing down the lights. See \cite{martin01}.

\begin{definition}
  For all $c,r\in\N^{+}$ we define the following symbols related to
  $r \times c$ Lights Out layout.

  An {\it $r\times c$ light pattern} is an $r\times c$ matrix
  with entries in $\GF_q$. The zero-pattern is the all-zeroes
  matrix, indicated by $O$; it corresponds to the situation
  where no lights are lit (on a $r\times c$ board).
  The set of all $r\times c$ light patterns is denoted by $\Lp_{r,c}$.

  A {\it $r\times c$ press pattern $p$} is also an $r\times c$ matrix
  with entries in $\GF_q$; it indicates the effect of changing
  the light at position $i, j$ of the board by the entry $p_{i,j}$.
  $\Pp_{r,c}$ is the set of all such press patterns.

  For every $p\in\Pp_{r,c}$ there exist a map $\phi_{p,r,c} : \Lp_{r,c}
  \rightarrow \Lp_{r,c}$ that describes the effect of pressing $p$ on
  a light pattern $\ell$.  The result will be $\ell + p$.

  Applying the map
  $P_{r,c} : \Pp_{r,c} \rightarrow \Lp_{r,c}$
  to a press pattern $p$ results in the light pattern $\ell$ that
  emerges when applying $p$ to the zero pattern. Thus
  $p \mapsto \ell=\phi_{p,r,c}(O)$; note that the
  matrices $p$ and $\ell$ are identical.
\end{definition}
