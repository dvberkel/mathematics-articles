\section{Introduction}
Lights Out is a handheld electronic puzzle game produced by Tiger Electronics in
the 1990s. It consists of a square grid of buttons that act as lights. The
object of the game is to turn off all the lights. This can be achieved by
pressing a series of buttons. Each button press has the effect of changing the
state of the light from off to on and vice versa for itself and each of it's
direct neighbors.

Lights Out, and its variants and predecessors, has a long history of
being studied by mathematicians. In \cite{anderson98}, and before that in
\cite{pelletier87}, a connection is made between Lights Out and linear algebra.
Their approach amounts to solving a linear equation

\[
Ax=-b
\]

Where $A$ is a $25\times 25$ square matrix, $b$ is the light pattern that needs
to be turned off. A solution to the equation tells you which buttons to press.

Solving this matrix equation by hand is not feasible. It does give insight into
when a solutions is possible and how many different solution exist. A practical
solution is given in \cite{martin01} where a technique knows as gathering or
\emph{chasing the lights} is introduced.

\begin{figure}
  \mbox{
    \subfigure{\epsfig{figure=image/windowing01.ps,width=.40\textwidth}}
    \hfill
    \subfigure{\epsfig{figure=image/windowing02.ps,width=.40\textwidth}}
  }
  \caption{Chasing the lights}
\end{figure}


Introduction to Lights Out and the chasing down the lights. See \cite{martin01}.

\begin{definition}
  For all $c,r\in\N^{+}$ we define the following symbols related to
  $r \times c$ Lights Out layout.

  An {\it $r\times c$ light pattern} is an $r\times c$ matrix
  with entries in $\GF_q$. The zero-pattern is the all-zeroes
  matrix, indicated by $O$; it corresponds to the situation
  where no lights are lit (on a $r\times c$ board).
  The set of all $r\times c$ light patterns is denoted by $\Lp_{r,c}$.

  A {\it $r\times c$ press pattern $p$} is also an $r\times c$ matrix
  with entries in $\GF_q$; it indicates which buttons to press. The
  button at position $i, j$ of the board will be pressed $p_{i,j}$
  times. $\Pp_{r,c}$ is the set of all such press patterns.

  For every $p\in\Pp_{r,c}$ there exist a map $\phi_{p,r,c} : \Lp_{r,c}
  \rightarrow \Lp_{r,c}$ that describes the effect of pressing $p$ on
  a light pattern $\ell$.

  The map $P_{r,c} : \Pp_{r,c} \rightarrow \Lp_{r,c}$ is defined by
  applying the press pattern $p$ to the zero pattern. I.e.
  $p \mapsto \phi_{p,r,c}(O)$.
\end{definition}
