\section{Introduction}
Lights Out is a handheld electronic puzzle game produced by Tiger Electronics in
the 1990s. It consists of a square grid of buttons that act as lights. The
object of the game is to turn off all the lights. This can be achieved by
pressing a series of buttons. Each button press has the effect of changing the
state of the light from off to on, and vice versa, for itself and each of it's
direct neighbors.

Lights Out, and its variants and predecessors, has a long history of
being studied by mathematicians. In \cite{anderson98}, and before that in
\cite{pelletier87}, a connection is made between Lights Out and linear algebra.
Their approach amounts to solving a linear equation

\[
Ap=-s
\]

Where $A$ is a $25\times 25$ square matrix, $s$ is the light pattern that needs
to be turned off. A solution to the equation tells you which buttons to press.

Solving this matrix equation by hand is not feasible. It does give insight into
when a solutions is possible and how many different solutions exist. A practical
solution is given in \cite{martin01} where a technique known as gathering or
\emph{chasing the lights} is introduced.

\begin{figure}
  \mbox{
    \subfigure[Window of first two rows\label{first_row_pressed}]{\epsfig{figure=image/windowing01.ps,width=.40\textwidth}}
    \subfigure[Next window\label{second_row_pressed}]{\epsfig{figure=image/windowing02.ps,width=.40\textwidth}}
  }
  \caption{Chasing the lights}
  \label{chasing}
\end{figure}

A moments thought will bring the realization that the order in which we press
buttons to turn the lights of is of no importance. This means that we can pick
any order. We will chose to press buttons per row from left to right and the rows
from top to bottom.

Now let's say that we are in the process of turning of the lights. In figure
\ref{chasing} we have already pressed all the buttons in the first row. There is
only one way to turn of the lights that are still lit in the first row. I.e. to
press each button in the second row that is directly underneath a lit button. No
other buttons we still need to press effects the first row.

Once the buttons in the first row are pressed, options are forced, all the way
down to the last row. This sets up a simpler linear equation. In the language of
\cite{martin01}.

\[
Bp_{1}=-chase(s)
\]

Here $B$ is a $5\times 5$ matrix. $p_{1}$ are the first five components of the
buttons to press, i.e. the first row. And $chase(s)$ is the effect of chasing
the lights to the last row.

This allows a solution that can be committed to memory, which one of the authors
has done for the standard Lights out puzzle. 

In \cite{leach17} this method chasing the lights is extended to other
rectangular board shapes. They show that analyzing a $c \times r$ Lights Out
board one needs is interested in the upper left $c \times c$ sub-matrix of
$W^{r}$ where $W$ describes the effect of one step in chasing the lights.

It is in this light that the authors made their observations.
