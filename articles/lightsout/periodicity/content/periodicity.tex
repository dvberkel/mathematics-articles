\section{Periodicity}

The process of chasing down the lights is formalized in the following
manner.

For all $c\in\N$ define $W_{c} : \GF(q)^{2c} \rightarrow \GF(q)^{2c}$
as

\[
W_{c} := \left(
\begin{array}{cc}
  E_{c} & I \\
  I    & O \\
\end{array}
\right)
\]

where $O$ is the zero matrix, $I$ the identity and $E_{c}$ is defined
as

For example

\[
E_{c} := \left(
\begin{array}{cccc}
  1 & 1 & 0 & 0 \\
  1 & 1 & 1 & 0 \\
  0 & 1 & 1 & 1 \\
  0 & 0 & 1 & 1 \\
\end{array}
\right)
\]

\begin{lemma}
  $W_{c}$ is invertable for all $c\in\N$.
\end{lemma}

\begin{proof}
  \[
  \left(
  \begin{array}{cc}
    E & I \\
    I & O \\
  \end{array}
  \right)
  \cdot
  \left(
  \begin{array}{cc}
    O & I  \\
    I & -E \\
  \end{array}
  \right)
  =
  \left(
  \begin{array}{cc}
    I & O \\
    O & I \\
  \end{array}			
  \right)
  \]
\end{proof}

\begin{theorem}
  The sequence $\left(W_{c}^{n}\right)_{n\in\N}$ is periodic.
\end{theorem}

\begin{proof}
  There are finitely many square matrices of size $2c$ over
  $\GF(q)$. So the sequence $\left(W_{c}^{n}\right)_{n\in\N}$ must
  repeat. By the preceding lemma $W_{c}$ is invertable so the
  sequence is periodic.
\end{proof}

The upper left submatrix of $W_{c}^{k}$ determine the results of
changing down the lights of $c \times k$ lights out configuration.

\begin{corollary}
  The sequence $\left(\dim\left(\Ker(P_{c,n})\right)\right)_{n\in\N}$
  is periodic.
\end{corollary}

\begin{proof}
  $\left(\dim\left(\Ker(P_{c,n})\right)\right)_{n\in\N} = \left(\dim\left(\Ker(W_{c}^{n}|_{1,1})\right)\right)_{n\in\N}$.
\end{proof}

\begin{lemma}
  $\dim\left(\Ker(P_{c,p-1})\right) = c$ where $p$ is period of the sequence. 
\end{lemma}

\begin{proof}
  Notice that $W_{c}^{p} = I$ and thus $W_{c}^{p-1} =  W_{c}^{-1}$.
  Furthermore $\dim\left(\Ker(P_{c,p-1})\right) =
  \dim\left(\Ker(W_{c}^{p-1}|_{1,1})\right) =
  \dim\left(\Ker(W_{c}^{-1}|_{1,1})\right)$ and $W_{c}^{-1}|_{1,1} = O$. 
  Hence $\dim\left(\Ker(P_{c,p-2})\right) = c$.
\end{proof}

The last result implies that every pattern of presses in the first row
can be extended into an element of the kernel.

\begin{theorem}
  The sequence $\left(\dim\left(\Ker(P_{c,n})\right)\right)_{n\in\N}$
  is almost palindromic.
\end{theorem}

\begin{proof}
  let $p$ be the period of
  $\left(\dim\left(\Ker(P_{c,n})\right)\right)_{n\in\N}$. We will show
  that for $k,k'\in\N$ such that $k+1+k'=p$
  \[
  \dim\left(\Ker(P_{c,k})\right) = \dim\left(\Ker(P_{c,k'})\right)
  \]

  For all $v\in\Ker(P_{c,k})$, v determines a element of
  $\Ker(p_{c,k'})$. To see this notice that $v$ can be extended to an
  element of $V\in\Ker(P_{c,p})$. Because $v\in\Ker(p_{c,k})$ the
  structure of $V$ is
  
  \[
  V =
  \left(
  \begin{array}{ccc}
    v & \underbrace{0}_{c\times} & v' \\
  \end{array}
  \right)^{t}
  \]
  
  Because $V\in\Ker(P_{c,p})$ also
  $v'\in\Ker(P_{c,k'})$. Hence  $\dim\left(\Ker(P_{c,k})\right) \le
  \dim\left(\Ker(P_{c,k'})\right)$ 

  Symmetry finishes the proof.
\end{proof}
