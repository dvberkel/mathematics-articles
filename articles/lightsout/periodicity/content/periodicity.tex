\section{Periodicity}

The process of chasing down the lights is formalized in the following
manner.

For all $c\in\N$ define $W_{c} : \GF(q)^{2c} \rightarrow \GF(q)^{2c}$
as

\[
W_{c} := \left(
\begin{array}{cc}
  E_{c} & I \\
  I    & O \\
\end{array}
\right)
\]

where $O$ is the zero matrix, $I$ the identity and $E_{c}$ is defined
as

For example

\[
E_{c} := \left(
\begin{array}{cccc}
  1 & 1 & 0 & 0 \\
  1 & 1 & 1 & 0 \\
  0 & 1 & 1 & 1 \\
  0 & 0 & 1 & 1 \\
\end{array}
\right)
\]

\begin{lemma}
  $W_{c}$ is invertible for all $c\in\N$.
\end{lemma}

\begin{proof}
  \[
  \left(
  \begin{array}{cc}
    E & I \\
    I & O \\
  \end{array}
  \right)
  \cdot
  \left(
  \begin{array}{cc}
    O & I  \\
    I & -E \\
  \end{array}
  \right)
  =
  \left(
  \begin{array}{cc}
    I & O \\
    O & I \\
  \end{array}			
  \right)
  \]
\end{proof}

\begin{theorem}
  The sequence $\left(W_{c}^{n}\right)_{n\in\N}$ is periodic.
\end{theorem}

\begin{proof}
  There are finitely many square matrices of size $2c$ over
  $\GF(q)$. So the sequence $\left(W_{c}^{n}\right)_{n\in\N}$ must
  repeat. By the preceding lemma $W_{c}$ is invertible so the
  sequence is periodic.
\end{proof}

\begin{lemma}[$W$-structure]
  There exist a sequence of $c \times c$ matrices
  $\left(A_{n}\right)_{n\in\N}$ such that
  \[
  W_{c}^{k+1}
  =
  \left(
  \begin{array}{cc}
    A_{k+2} & A_{k+1} \\
    A_{k+1} & A_{k}   \\
  \end{array}
  \right)  
  \]
  for all $k\in\N$.
\end{lemma}

\begin{proof}
  Define $A_{0} := O$, $A_{1} := I$ and 
  $A_{n+1} := E \cdot A_{n} + A_{n-1}$ for all $n\in\N^{+}$. So 
  $A_{2} = E \cdot I + O = E$. 

  A number $n\in\N$ is called strong if and only if
  \[
  W_{c}^{n+1}
  =
  \left(
  \begin{array}{cc}
    A_{n+2} & A_{n+1} \\
    A_{n+1} & A_{n}   \\
  \end{array}
  \right)  
  \]

  Notice that
  $W_{c}^{1} = \left(\begin{smallmatrix} E & I \\ I & O \\\end{smallmatrix}\right) = \left(\begin{smallmatrix} A_{2} & A_{1} \\ A_{1} & A_{0} \\\end{smallmatrix}\right)$
  so 0 is strong.

  Assume that $k$ is strong. We will show that $k+1$ is strong as well.
  \[
  \begin{aligned}
  W_{c}^{k+1} 
  & = W \cdot W_{c}^{k} \\
  & = 
  \left(
  \begin{array}{cc}
    E & I \\
    I & O \\
  \end{array}
  \right)
  \cdot
  \left(
  \begin{array}{cc}
    A_{k+2} & A_{k+1} \\
    A_{k+1} & A_{k}   \\
  \end{array}
  \right) \\
  & = 
  \left(
  \begin{array}{cc}
    E \cdot A_{k+2}  + A_{k+1} & E \cdot A_{k+1} + A_{k} \\
    A_{k+2}                   & A_{k+1}                \\
  \end{array}
  \right) \\
  & =
  \left(
  \begin{array}{cc}
    A_{k+3} & A_{k+2} \\
    A_{k+2} & A_{k+1} \\
  \end{array}
  \right) \\
  \end{aligned}
  \]

  By applying the principle of mathematical induction all natural
  numbers are strong, finishing the proof.
\end{proof}

\begin{corollary}
  $W_{c}^{n}$ is symmetric for all $n\in\N$.
\end{corollary}

\begin{proposition}
  The sequence $\left(\dim\left(\Ker(P_{c,n})\right)\right)_{n\in\N}$
  is periodic.
\end{proposition}

\begin{proof}
  By the periodicity of $(W_{c}^{n})_{n\in\N}$
  \[
  \left(\dim\left(\Ker(P_{c,n})\right)\right)_{n\in\N} 
  = 
  \left(\dim\left(\Ker(A_{n+1})\right)\right)_{n\in\N}
  \]

  where $W_{c}^{n}=\left(\begin{smallmatrix} A_{n+2} & A_{n+1}  \\ A_{n+1} & A_{n} \\\end{smallmatrix}\right)$
  by the $W$-structure lemma.
\end{proof}

\begin{lemma}
  $\dim\left(\Ker(P_{c,p-1})\right) = c$ where $p$ is period of the sequence. 
\end{lemma}

\begin{proof}
  We will be using the notation as defined by the $W$-structure lemma.

  Notice that $W_{c}^{p} = I$ and thus $W_{c}^{p-1} = W_{c}^{-1}$.
  Furthermore
  \[
  \dim\left(\Ker(P_{c,p-1})\right)
  =
  \dim\left(\Ker(A_{p})\right) = \dim\left(O\right)
  = c
  \]
\end{proof}

\begin{theorem}
  The sequence $\left(\dim\left(\Ker(P_{c,n})\right)\right)_{n\in\N}$
  is almost palindromic.
\end{theorem}

\begin{proof}
  let $p$ be the period of
  $\left(\dim\left(\Ker(P_{c,n})\right)\right)_{n\in\N}$. We will show
  that for $k,k'\in\N$ such that $k+1+k'=p$
  \[
  \dim\left(\Ker(P_{c,k})\right) = \dim\left(\Ker(P_{c,k'})\right)
  \]

  For all $v\in\Ker(P_{c,k})$, v determines a element of
  $\Ker(p_{c,k'})$. To see this notice that $v$ can be extended to an
  element of $V\in\Ker(P_{c,p})$. Because $v\in\Ker(p_{c,k})$ the
  structure of $V$ is
  
  \[
  V =
  \left(
  \begin{array}{ccc}
    v & \underbrace{0}_{c\times} & v' \\
  \end{array}
  \right)^{t}
  \]
  
  Because $V\in\Ker(P_{c,p})$ also
  $v'\in\Ker(P_{c,k'})$. Hence  $\dim\left(\Ker(P_{c,k})\right) \le
  \dim\left(\Ker(P_{c,k'})\right)$ 

  Symmetry finishes the proof.
\end{proof}
