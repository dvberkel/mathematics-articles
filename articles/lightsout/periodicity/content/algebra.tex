\section{Algebra}
We will formalize the process of chasing down the lights and prove all of our
observations. To begin we will introduce a matrix of some interest.

For all $c\in\N$ define the $2c\times 2c$ matrix $W_{c}$ with entries in
$\Z/n\Z$ by
\[
W_{c} := \left(
\begin{array}{cc}
  -E_{c} & I \\
  -I    & O \\
\end{array}
\right)
\]
where $O$ is the zero matrix, $I$ the identity matrix and $E_{c}$ is defined
as the $c\times c$ matrix with ones on the diagonal and the two main
sub-diagonals, and zeroes elsewhere.

For example

\[
E_{4} := \left(
\begin{array}{cccc}
  1 & 1 & 0 & 0 \\
  1 & 1 & 1 & 0 \\
  0 & 1 & 1 & 1 \\
  0 & 0 & 1 & 1 \\
\end{array}
\right)
\]

The reason we are looking at $W_{c}$ is that its powers tell us something about
the effect of chasing down the lights. In particular, if we have an $(n, c, r)$
lights out puzzle, the $c \times c$ upper left sub-matrix of $W_{c}^{r}$ is
exactly the process of gathering the lights.

The first interesting fact is that $W_{c}$ is invertible.

\begin{lemma}
  $W_{c}$ is invertible for all $c\in\N$.
\end{lemma}

\begin{proof}
  \[
  \left(
  \begin{array}{cc}
    -E & I \\
    -I & O \\
  \end{array}
  \right)
  \cdot
  \left(
  \begin{array}{cc}
    O & -I  \\
    I & -E \\
  \end{array}
  \right)
  =
  \left(
  \begin{array}{cc}
    I & O \\
    O & I \\
  \end{array}
  \right)
  \]
\end{proof}

A consequence of the invertibility of $W_{c}$ is that the sequence of its powers
is periodic. This also proves our observation \ref{observation.periodic}, but we
will have some more to say about that.

\begin{theorem}
  The sequence $\left(W_{c}^{r}\right)_{r\in\N}$ is purely periodic.
\end{theorem}

\begin{proof}
  There are only finitely many different square matrices of size $2c$ over
  $\Z/n\Z$. So the sequence $\left(W_{c}^{r}\right)_{r\in\N}$ must
  become periodic. By the preceding lemma $W_{c}$ is invertible so the
  sequence is periodic from the start.
\end{proof}

As mentioned in \cite{leach17} there is a relation between the images of chasing
down the lights and Fibonacci polynomials. We find that relation in our
structure lemma.

\begin{lemma}[structure]
  There exists a sequence of $c \times c$ matrices
  $\left(T_{r}\right)_{r\in\N}$ such that
  \[
  W_{c}^{r}
  =
  \left(
  \begin{array}{cc}
     T_{r} &  T_{r-1} \\
    -T_{r-1} & -T_{r-2}   \\
  \end{array}
  \right)
  \]
  for all $r\in\N$.
\end{lemma}

\begin{proof}
  Define $T_{0} := I$, and for convenience $T_{-1} := O$ and
  $T_{r+1} := -E \cdot T_{r} - T_{r-1}$ for all $r\in\N$. So
  $T_{1} = -E \cdot I - O = -E$. 

  A number $r\in\N$ is called strong if and only if
  \[
  W_{c}^{r}
  =
  \left(
  \begin{array}{cc}
     T_{r} &  T_{r-1}  \\
    -T_{r-1} & -T_{r-2} \\
  \end{array}
  \right)
  \]

  Notice that
  $W_{c}^{1} = \left(\begin{smallmatrix} -E & I \\ -I & O \\\end{smallmatrix}\right) = \left(\begin{smallmatrix} T_{1} & T_{0} \\ -T_{0} & -T_{-1} \\\end{smallmatrix}\right)$
  so $1$ is strong.

  Assume that $k$ is strong. We will show that $k+1$ is strong as well.
  \[
  \begin{aligned}
  W_{c}^{k+1}
  & = W_{c} \cdot W_{c}^{k} \\
  & =
  \left(
  \begin{array}{cc}
    -E & I \\
    -I & O \\
  \end{array}
  \right)
  \cdot
  \left(
  \begin{array}{cc}
     T_{k} &  T_{k-1}  \\
    -T_{k-1} & -T_{k-2} \\
  \end{array}
  \right) \\
  & =
  \left(
  \begin{array}{cc}
    -E \cdot T_{k} - T_{k-1} & -E \cdot T_{k-1} - T_{k-2} \\
    -T_{k}                  & -T_{k-1}                  \\
  \end{array}
  \right) \\
  & =
  \left(
  \begin{array}{cc}
     T_{k+1} &  T_{k} \\
    -T_{k} & -T_{k-1} \\
  \end{array}
  \right) \\
  \end{aligned}
  \]

  By mathematical induction all natural numbers are strong, finishing the proof.
\end{proof}

With the structure lemma under our belt we can prove our first observation, i.e.
the sequence of the dimension of kernels is periodic.

\begin{proposition}[Observation \ref{observation.periodic}]
  The sequence $\left(C_{(n,c,r)}\right)_{r\in\N}$ is periodic.
\end{proposition}

\begin{proof}
  By the structure lemma, for all $r\in\N$,
  \[
  C_{(n,c,r)}
  =
  \dim\Ker T_{r}
  \]

  By the periodicity of $(W_{c}^{r})_{r\in\N}$ we have the
  periodicity of $\left(C_{(n,c,r)}\right)_{r\in\N}$.
\end{proof}

It could be the case that the period of $\left(C_{(n,c,r)}\right)_{r\in\N}$
is a divisor of the period of $(W_{c}^{r})_{r\in\N}$. There is a relation which 
we will see shortly.

Next on our agenda is our observation \ref{observation.maximal}. I.e. for each
number of columns $c$ there is a kernel with that dimension.

\begin{lemma}[Observation \ref{observation.maximal}]
  There is a kernel of maximal dimension.
\end{lemma}

\begin{proof}
  We will be using the notation as defined by the structure lemma.

  Notice that there exists $p\in\N$ such that $W_{c}^{p} = I$ and
  thus $W_{c}^{p-1} = W_{c}^{-1} \cdot W_{c}^{p} = W_{c}^{-1}$.
  Furthermore
  \[
  C_{(n,c,p)}
  =
  \dim\Ker T_{-1}
  =
  \dim\Ker O
  = c
  \]
\end{proof}

Before we will dive deeper in the question if the period of both sequences
coincide, we will take a closer look at observation \ref{observation.consecutive}.

\begin{theorem}[Observation \ref{observation.consecutive}]
  For all $i\in\N$ the following inequality holds
  \[C_{(n, c, i)} + C_{(n, c, i+1)} \leq c\]
\end{theorem}

\begin{proof}
  Assume to the contrary that there is a $k\in\N$ such that $C_{(n, c, k)} + C_{(n, c, k+1)} > c$.

  Let $U := \Ker\chase_{(n, c, k)}$ and let $u_{1}, \ldots, u_{m}$ be a basis for $U$, with $m := C_{(n, c, k)}$.
  In similar fashion let $V := \Ker\chase_{(n, c, k)}$ with basis $v_{1}, \ldots, v_{n}$ and $n := C_{(n, c, k+1)}$. In particular
  the basis for $U$ and $V$ are both linear independent.

  Now, because each basis for a $\Z/n\Z$-module has the same rank, and the rank of $\Ep$ is $c$ (See \cite{roman07}), the set of combined vector is linear dependent.
  I.e. there are $s_i$ and $t_i$ not all zero such that
  \[
    s_{1} u_{1} + s_{2} u_{2} + \cdots + s_{m} u_{m} + t_{1} v_{1} + \cdots + t_{n} v_{n} = 0 
  \]

  Because the $u_{i}$ and $v_{i}$ are linear independent there exist a non-zero $p \in U \cap V$. Hence

  \[
    W^{k} p =
    \left(
    \begin{array}{c}
      O  \\
      w \\
    \end{array}
    \right)
  \]

  and

  \[
    W^{k+1} p =
    \left(
    \begin{array}{c}
      O  \\
      w' \\
    \end{array}
    \right)
  \]

  but

  \[
    \left(
    \begin{array}{c}
      O  \\
      w' \\
    \end{array}
    \right)
    =
    W^{k+1} p
    =
    W W^{k} p
    =
    W 
    \left(
    \begin{array}{c}
      O \\
      w \\
    \end{array}
    \right)
    =
    \left(
    \begin{array}{c}
      w \\
      O \\
    \end{array}
    \right)
  \]

  So $W^{k} p = O$, which contradicts the fact that $W$ is invertable.
\end{proof}

This little fact will helps us establishing the proof of the following fact

\begin{corollary}
  If for some $r\in\N$ we have $C_{n,c,r} = c$ then
  \begin{itemize}
    \item $C_{n,c,r-1} = 0$
    \item $C_{n,c,r+1} = 0$
  \end{itemize}
\end{corollary}

\begin{proof}
  If for some $r\in\N$ we have $C_{(n,c,r)} = c$, then both

  \[
    C_{(n, c, r-1)} \leq c - C_{(n,c,r)} = c - c = 0
  \]

  and

  \[
    C_{(n, c, r+1)} \leq c - C_{(n,c,r)} = c - c = 0
  \]
\end{proof}

\begin{lemma}\label{block-diagonal}
  If for some $r\in\N$ we have $C_{(n,c,r)} = c$ then
  $W_{c}^{r+1} = \left(\begin{smallmatrix} -T_{r-1} & O \\ O & -T_{r-1} \\\end{smallmatrix}\right)$
\end{lemma}

\begin{proof}
  If for some $r\in\N$ we have $C_{(n,c,r)} = c$; then

  \[
  W_{c}^{r}
  =
  \left(
  \begin{array}{cc}
     O    &  T_{r-1}  \\
    -T_{r-1} & -T_{r-2} \\
  \end{array}
  \right)
  \]

  By direct multiplication we note that
  $W_{c}^{r+1} = \left(\begin{smallmatrix} -T_{r-1} & O \\ O & -T_{r-1} \\\end{smallmatrix}\right)$,
\end{proof}

In proving our lemma \ref{block-diagonall} we have learned that the
structure of the corresponding matrix power is particular simple. This fact will
be instrumental in the relation between the period of
$\left(W_{c}^{r}\right)_{r\in\N}$ and that of
$\left(C_{(n,c,r)}\right)_{r\in\N}$.

But first we will see that $W_{c}$ and it's inverse are conjugates.

\begin{lemma}
  $W_{c}$ and $W_{c}^{-1}$ are conjugates.
\end{lemma}

\begin{proof}
  We will conjugate $W=W_{c}$ by
  $D := \left(\begin{smallmatrix} O & I \\ I & O \\\end{smallmatrix}\right)$,
  which is its own inverse,
  \[
  \begin{aligned}
  D \cdot W \cdot D^{-1}
  & =
  \left(
  \begin{array}{cc}
    O & I \\
    I & O \\
  \end{array}
  \right)
  \cdot
  \left(
  \begin{array}{cc}
    -E & I \\
    -I & O \\
  \end{array}
  \right)
  \cdot
  \left(
  \begin{array}{cc}
    O & I \\
    I & O \\
  \end{array}
  \right) \\
  & =
  \left(
  \begin{array}{cc}
    -I & O \\
    -E & I \\
  \end{array}
  \right)
  \cdot
  \left(
  \begin{array}{cc}
    O & I \\
    I & O \\
  \end{array}
  \right) \\
  & =
  \left(
  \begin{array}{cc}
    O & -I \\
    I & -E \\
  \end{array}
  \right) \\
  & =
  W^{-1}
  \end{aligned}
  \]
\end{proof}

\begin{corollary}
  $W_{c}^{-r} =
  \left(\begin{smallmatrix} O & I \\ I & O \\\end{smallmatrix}\right)
  W_{c}^{r}
  \left(\begin{smallmatrix} O & I \\ I & O \\\end{smallmatrix}\right)$
\end{corollary}

\begin{proof}
  Clear.
\end{proof}

Note that conjugating any matrix with
$\left(\begin{smallmatrix} O & I \\ I & O \\\end{smallmatrix}\right)$ 
corresponds to rotating the four cardinal sub-matrices through 180 degrees.

This fact will be the linchpin in the proof of observation
\ref{observation.palindromic}.

\begin{lemma}\label{mirror}
  If for some $r\in\N$ we have $W_{c}^{r} = \left(\begin{smallmatrix} Q & O \\ O & Q \\\end{smallmatrix}\right)$ then for all $i\in\N$.

  \[
  W_{c}^{r+i} = \left(\begin{array}{cc} QT_{i} & QT_{i-1} \\ -QT_{i-1} & -QT_{i-2} \\\end{array}\right)
  \]

  and

  \[
  W_{c}^{r-i} = \left(\begin{array}{cc} -QT_{i-2} & -QT_{i-1} \\ QT_{i-1} & QT_{i-2} \\\end{array}\right)
  \]
\end{lemma}

\begin{proof}
  By direct calculation and the structure lemma we find that

  \[
  W_{c}^{r+i}
  =
  W_{c}^{r}W_{c}^{i}
  =
  \left(
  \begin{array}{cc}
    Q & O \\
    O & Q \\
  \end{array}
  \right)
  \left(
  \begin{array}{cc}
     T_{i} &  T_{i-1}  \\
    -T_{i-1} & -T_{i-2} \\
  \end{array}
  \right)
  =
  \left(
  \begin{array}{cc}
     QT_{i} &  QT_{i-1}  \\
    -QT_{i-1} & -QT_{i-2} \\
  \end{array}
  \right)
  \]

  Furthermore, with $C=\left(\begin{smallmatrix} O & I \\ I & O \\\end{smallmatrix}\right)$
  we have, by the preceding lemma,
  $CW_{c}^{r-i}C^{-1}=CW_{c}^{r}C^{-1}CW^{-i}C^{-1}=W_{c}^{r}W_{c}^{i}=W_{c}^{r+i}$
  which we set out to prove.
\end{proof}

We now come to our promise about the period of the dimension of kernels the
kernels and the period of $W$. let $q$ be the smallest number rows that has a
maximal kernel dimension and let $d = q+1$. The period of $W$ will be $p$.

\begin{lemma}
  Either $p=d$ or $p=2d$.
\end{lemma}

\begin{proof}
  If $p=d$ we are finished, so assume it is not. We will show that $p=2d$ in
  that case.

  By the preceding lemma we that the lower right $c\times c$ sub-matrix of
  $W_{c}^{d-i}$ equals the upper left $c\times c$ sub-matrix of $W_{c}^{d+i}$
  for all $i$. In particular for $i=d$. The lower right sub-matrix of
  $W_{c}^{d-d} = W_{c}^{0} = I$ is the $c\times c$ identity matrix.

  Furthermore, since $W^{d}=\left(\begin{smallmatrix} Q & O \\ O & Q \\\end{smallmatrix}\right)$
  for certain matrix $Q$, we have

  \[
  W_{c}^{2d}
  =
  \left(
  \begin{array}{cc}
    Q & O \\
    O & Q \\
  \end{array}
  \right)^{2}
  =
  \left(
  \begin{array}{cc}
    Q^{2} & O \\
    O & Q^{2} \\
  \end{array}
  \right)
  \]

  Hence, $Q^{2}$ is equal to the $c\times c$ identity matrix, therefore $W^{2d}$
  is the identity matrix, and the period of $(W_{c}^{r})_{r\in\N}$ is $2d$.
\end{proof}

\begin{theorem}[Observation \ref{observation.palindromic}]
  The sequence $\left(C_{(n,c,r)}\right)_{r\in\N}$
  is almost palindromic.
\end{theorem}

\begin{proof}
  let $d$ be such that $W^d$ has a upperleft sub matrix $O$.

  A \emph{press vector} is a $2\times c$ vector with the last $c$ components zero, an \emph{unlit vector} is a $2\times c$ vector with the first $c$ components 0. Notice that for any press vector $v$

  \[
    W^d v
  \]

  is an unlit vector.

  Choose $m,n\in\N$ such that $m + 1 + n = d$. We will show that for each press vector $v$ for which $W^m v$ is unlit, there exist a press vector $v'$ such that $W^n v'$ is unlit.
  This shows that $\dim\Ker T_{m} \leq \dim\Ker T_{n}$. Since the argument is symmetric in $m$ and $n$ we have $\dim\Ker T_{m} = \dim\Ker T_{n}$.

  Let $v$ be a press vector such that $W^{m} v$ is unlit. In particular $W^{m} v = u$ with $u := (0,0,\ldots, 0, u_{1}, u_{2}, \ldots, u_{c})^t$.

  Define $p := W u$. We will show that $p$ is a press vector.

  \[
    W u = \left(
    \begin{array}{cc}
      -E_{c} & I \\
      -I    & O \\
    \end{array}
    \right)
    \left(
    \begin{array}{c}
      O  \\
      u' \\
    \end{array}
    \right)
    =
    \left(
    \begin{array}{c}
      u' \\
      O  \\
    \end{array}
    \right)
  \]k

  Since for any press vector $w$ we have that $W^{d} w = O$, in particular we have

  \[
  O = W^{d} v = W^{n} W W^{m} v = W^{n} W u = W^{n} p
  \]

  which shows that for each press vector that $W^{m}$ unlits, there is a press vector that $W^{n}$ unlits.
\end{proof}

Now everything is in play to prove \ref{observation.period}

\begin{theorem}[Observation \ref{observation.period}]
  The period of $(C_{(n, c, r)})_{r\in\N}$ starts after the maximal dimension.
\end{theorem}

\begin{proof}
By lemma \ref{block-diagonal} the power of $W$ has a block diagonal structure. By lemma \ref{mirror} the sequence of dimensions
is shifted mirror image. Together with the fact that the sequence is almost palindromic from the previous theorem, we have our proof.
\end{proof}