\section{Algebra}

The process of chasing down the lights is formalized in the following
manner.

For all $r\in\N$ define the $2r\times 2r$ matrix
$W_{r}$ with entries in $\GF(q)$ by
\[
W_{r} := \left(
\begin{array}{cc}
  -E_{r} & I \\
  -I    & O \\
\end{array}
\right)
\]
where $O$ is the zero matrix, $I$ the identity and $E_{r}$ is defined
as the $r\times r$ matrix with ones on the diagonal and the two main
sub-diagonals, and zeroes elsewhere.

For example

\[
E_{4} := \left(
\begin{array}{cccc}
  1 & 1 & 0 & 0 \\
  1 & 1 & 1 & 0 \\
  0 & 1 & 1 & 1 \\
  0 & 0 & 1 & 1 \\
\end{array}
\right)
\]

\begin{lemma}
  $W_{r}$ is invertible for all $r\in\N$.
\end{lemma}

\begin{proof}
  \[
  \left(
  \begin{array}{cc}
    -E & I \\
    -I & O \\
  \end{array}
  \right)
  \cdot
  \left(
  \begin{array}{cc}
    O & -I  \\
    I & -E \\
  \end{array}
  \right)
  =
  \left(
  \begin{array}{cc}
    I & O \\
    O & I \\
  \end{array}
  \right)
  \]
\end{proof}


\begin{lemma}
  $\det(W_{r}^{n}) = 1$ for all $n\in\N$.
\end{lemma}

\begin{proof}
  Note that $\det(W_{r}^{0}) = \det(I) = 1$ and
  \[
  \det(W_{r}) = \det
  \left(
  \begin{array}{cc}
    -E & I \\
    -I & O \\
  \end{array}
  \right)
  =
  \det(-E) \cdot \det(O) - \det(-I) \cdot \det(I)
  =
  1
  \]
  The lemma follows by multiplicativity of the determinant and
  by induction on $n$.
\end{proof}

\begin{theorem}
  The sequence $\left(W_{r}^{n}\right)_{n\in\N}$ is purely periodic.
\end{theorem}

\begin{proof}
  There are only finitely many different square matrices of size $2r$ over
  $\GF(q)$. So the sequence $\left(W_{r}^{n}\right)_{n\in\N}$ must
  become periodic. By the preceding lemma $W_{r}$ is invertible so the
  sequence is periodic from the start.
\end{proof}

\begin{lemma}[$W$-structure]
  There exists a sequence of $r \times r$ matrices
  $\left(A_{n}\right)_{n\in\N}$ such that
  \[
  W_{r}^{k+1}
  =
  \left(
  \begin{array}{cc}
     A_{k+2} &  A_{k+1} \\
    -A_{k+1} & -A_{k}   \\
  \end{array}
  \right)
  \]
  for all $k\in\N$.
\end{lemma}

\begin{proof}
  Define $A_{0} := O$, $A_{1} := I$ and
  $A_{n+1} := -E \cdot A_{n} - A_{n-1}$ for all $n\in\N^{+}$. So
  $A_{2} = -E \cdot I + O = -E$.

  A number $n\in\N$ is called strong if and only if
  \[
  W_{r}^{n+1}
  =
  \left(
  \begin{array}{cc}
     A_{n+2} &  A_{n+1} \\
    -A_{n+1} & -A_{n}   \\
  \end{array}
  \right)
  \]

  Notice that
  $W_{r}^{1} = \left(\begin{smallmatrix} -E & I \\ -I & O \\\end{smallmatrix}\right) = \left(\begin{smallmatrix} A_{2} & A_{1} \\ -A_{1} & -A_{0} \\\end{smallmatrix}\right)$
  so $0$ is strong.

  Assume that $k$ is strong. We will show that $k+1$ is strong as well.
  \[
  \begin{aligned}
  W_{r}^{k+1}
  & = W \cdot W_{r}^{k} \\
  & =
  \left(
  \begin{array}{cc}
    -E & I \\
    -I & O \\
  \end{array}
  \right)
  \cdot
  \left(
  \begin{array}{cc}
     A_{k+2} &  A_{k+1} \\
    -A_{k+1} & -A_{k}   \\
  \end{array}
  \right) \\
  & =
  \left(
  \begin{array}{cc}
    -E \cdot A_{k+2} - A_{k+1} & -E \cdot A_{k+1} - A_{k} \\
    -A_{k+2}                  & -A_{k+1}                 \\
  \end{array}
  \right) \\
  & =
  \left(
  \begin{array}{cc}
     A_{k+3} &  A_{k+2} \\
    -A_{k+2} & -A_{k+1} \\
  \end{array}
  \right) \\
  \end{aligned}
  \]

  By mathematical induction all natural
  numbers are strong, finishing the proof.
\end{proof}

\begin{proposition}
  The sequence $\left(\dim \Ker P_{r,n}\right)_{n\in\N}$
  is periodic.
\end{proposition}

\begin{proof}
  By the $W$-structure lemma, for all $n\in\N$,
  \[
  \dim\Ker P_{r,n}
  =
  \dim\Ker A_{n+1}
  \]

  where $W_{r}^{n}=\left(\begin{smallmatrix} A_{n+1} & A_{n}  \\ -A_{n} & -A_{n-1} \\\end{smallmatrix}\right)$.
  By the periodicity of $(W_{r}^{n})_{n\in\N}$ we have the
  periodicity of
  $\left(\dim\Ker(P_{r,n}\right)_{n\in\N}$.
\end{proof}

\begin{lemma}
  $\dim\Ker P_{r,p-1} = r$ where $p$ is period of the sequence.
\end{lemma}

\begin{proof}
  We will be using the notation as defined by the $W$-structure lemma.

  Notice that there exists $k\in\N$ such that $W_{r}^{kp} = I$ and
  thus $W_{r}^{kp-1} = W_{r}^{-1} \cdot W_{r}^{kp} = W_{r}^{-1}$.
  Furthermore
  \[
  \dim\Ker P_{r,p-1}
  =
  \dim\Ker P_{r,kp-1}
  =
  \dim\Ker A_{kp} = \dim O
  = r
  \]
\end{proof}

\begin{lemma}
  If for some $n\in\N$ we have $\dim\Ker P_{r,n} = r$ then
  \begin{itemize}
    \item $\dim\Ker P_{r,n-1} = 0$
    \item $\dim\Ker P_{r,n+1} = 0$
  \end{itemize}
\end{lemma}

\begin{proof}
  Let $n\in\N$ be such that $\dim\Ker P_{r,n} = r$; then
  \[
  W_{r}^{n}
  =
  \left(
  \begin{array}{cc}
     O    &  A_{n}  \\
    -A_{n} & -A_{n-1} \\
  \end{array}
  \right)
  \]
  and $1 = \det W_{r}^{n} = (\det A_{n})^{2}$, so $A_{n}$ is
  invertible.

  Note that
  $W_{r}^{n-1} = \left(\begin{smallmatrix} A_{n} & A_{n-1} \\-A_{n-1} & -A_{n-2} \\\end{smallmatrix}\right)$
  and
  $W_{r}^{n+1} = \left(\begin{smallmatrix} A_{n} & O \\ O & -A_{n} \\\end{smallmatrix}\right)$,
  hence both
  \begin{itemize}
    \item $\dim\Ker P_{r,n-1} = \dim\Ker A_{n} = 0$,
    \item $\dim\Ker P_{r,n+1} = \dim\Ker A_{n} = 0$.
  \end{itemize}
\end{proof}

\begin{lemma}
  $W_{r}$ and $W_{r}^{-1}$ are conjugates.
\end{lemma}

\begin{proof}
  We will conjugate $W=W_{r}$ by
  $C := \left(\begin{smallmatrix} O & I \\ I & O \\\end{smallmatrix}\right)$,
  which is its own inverse,
  \[
  \begin{aligned}
  C \cdot W \cdot C^{-1}
  & =
  \left(
  \begin{array}{cc}
    O & I \\
    I & O \\
  \end{array}
  \right)
  \cdot
  \left(
  \begin{array}{cc}
    -E & I \\
    -I & O \\
  \end{array}
  \right)
  \cdot
  \left(
  \begin{array}{cc}
    O & I \\
    I & O \\
  \end{array}
  \right) \\
  & =
  \left(
  \begin{array}{cc}
    -I & O \\
    -E & I \\
  \end{array}
  \right)
  \cdot
  \left(
  \begin{array}{cc}
    O & I \\
    I & O \\
  \end{array}
  \right) \\
  & =
  \left(
  \begin{array}{cc}
    0 & -I \\
    I & -E \\
  \end{array}
  \right) \\
  & =
  W^{-1}
  \end{aligned}
  \]
\end{proof}

\begin{corollary}
  $W_r^{-n} =
  \left(\begin{smallmatrix} O & I \\ I & O \\\end{smallmatrix}\right)
  W_r
  \left(\begin{smallmatrix} O & I \\ I & O \\\end{smallmatrix}\right)$
\end{corollary}

\begin{proof}
  Clear.
\end{proof}

\begin{remark}
  The effect of conjugating $W$ by
  $\left(\begin{smallmatrix} O & I \\ I & O \\\end{smallmatrix}\right)$
  is a rotation of $180^{\degree}$.
\end{remark}

\begin{theorem}
  The sequence $\left(\dim\Ker P_{r,n}\right)_{n\in\N}$
  is almost palindromic.
\end{theorem}

\begin{proof}
  We will be studing the doubly-infinite sequence
  $(W_{r}^{z})_{z\in\Z}$.

  By the $W$-structure lemma we have, for every $n\in\N$,
  \[
  W_{r}^{n+1}
  =
  \left(
  \begin{array}{cc}
     A_{n+2} &  A_{n+1} \\
    -A_{n+1} & -A_{n}   \\
  \end{array}
  \right)
  \]

  and by the preceding corollary $W_{r}^{-(n+1)}$ equals
  \[
  \left(
  \begin{array}{cc}
    O & I \\
    I & O \\
  \end{array}
  \right)
  \cdot
  \left(
  \begin{array}{cc}
     A_{n+2} &  A_{n+1} \\
    -A_{n+1} & -A_{n}   \\
  \end{array}
  \right)
  \cdot
  \left(
  \begin{array}{cc}
    O & I \\
    I & O \\
  \end{array}
  \right)
  =
  \left(
  \begin{array}{cc}
    -A_{n}  & -A_{n+1} \\
     A_{n+1} & A_{n+2} \\
  \end{array}
  \right);
  \]

  thus for all $n\in\N$
  \[
  W_{r}^{-(n+3)}
  =
  \left(
  \begin{array}{cc}
    -A_{n+2}  & -A_{n+3} \\
     A_{n+3} & A_{n+4} \\
  \end{array}
  \right).
  \]

  Hence
  \[
  \dim\Ker P_{r,n} =
  \dim\Ker A_{n+1} =
  \dim\Ker(-A_{n+1}) =
  \dim\Ker P_{r,-(n+3)}.
  \]

  This proves the theorem.
\end{proof}
