\begin{abstract}
The {\em Lights Out} puzzle game has been extensively
analyzed before, using linear algebra. In this paper
we discuss and prove interesting observations about
dimensions of null spaces determining the number of
solutions. In particular, we find periodic patterns
in these dimensions, when the size of the game board
as well as the number of colours of the buttons, varies.
% A lot had been said on the {\em Lights Out} puzzle game and its variants. A
% clear connection with linear algebra exists and that connection can be used to
% answer various puzzle related questions, such as when a solution exists and how
% many solutions there are.

%In this article we will discuss interesting observations about the
% dimensions of null spaces occurring when analyzing rectangular Lights Out. We
% will prove that this sequence is periodic and almost palindromic.
\end{abstract}
