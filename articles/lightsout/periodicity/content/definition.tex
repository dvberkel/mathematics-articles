\section{Definition}
In this section we will define some terms that we will use throughout the article.

In this article we are exploring rectangular Lights Out. A $(n, c, r)$ Lights Out puzzle
is a matrix $M$ with $c\in\N$ columns and $r\in\N$ rows with entries over $\Z/n\Z$. The space
of all $(n, c, r)$ puzzles is called $\Lp_{(n,c,r)}$.

For a each $\Lp_{(n,c,r)}$ and for each $(i, j)$ with $1\leq i \leq r$ and $1\leq j \leq c$
there is a basic press function $p_{(i,j)}: \Lp_{(n,c,r)} \rightarrow \Lp_{(n,c,r)}$ mapping

\[
    \left(p_{(i,j)}(M)\right)_{(u,v)} :=
    \left\{
    \begin{array}{l@{\quad:\quad}l}
        M_{(u,v)} + 1 & \operatorname{d}\left((i,j), (u,v)\right) \leq 1 \\
        M_{(u,v)}     & \text{otherwise}
    \end{array}
    \right.
\]

where $\operatorname{d}$ is the \emph{Manhattan distance}. The set of all basic presses is called $B$.

A press sequence is a finite sequence of basic presses. The set of all press sequences is called $P$.
$P$ together with concatenation of sequences makes a monoid with the empty sequence as identity element.

The effect $E$ of a press sequence is an mapping on $\Lp_{(n,c,r)}$ that extends basic presses.
I.e. the effect of the empty sequence is the identity map and for a press sequence $\left(q_{t}\right)_{t\in\overline{m}}$
with $\overline{m}:=\{0, 1, \ldots, m-1\}$

\[
    E\left(\left(q_{t})_{t\in\overline{m}}\right)\right)
    =
    E\left(\left(q_{t+1})_{t\in\overline{m-1}}\right)\right) \circ q_{0}
\]

For each press sequence $q:=\left(q_{t}\right)_{t\in\overline{m}}$ we define a count function $N: P \rightarrow \N^B$ that counts the number of times that a basic press is present.

\[
    N_{q}(p_{(i,j)}) = \sum_{t\in\overline{m}} 1_{(i,j)}(q_{t})
\]
where $1_{(i,j)}(q)$ is $1$ when $q=p_{(i,j)}$ and zero otherwise. We will write the application of $N$ to press sequence $q$ as $N_{q}$.

With this count we will define a fingerprint function $I: P \rightarrow (\Z/n\Z)^{B}$ that count how many times a basic press
is present in the sequence, modulo $n$. So for $\overline{m}:=\{0, 1, \ldots, m-1\}$ we have

\[
    I\left(q\right)
    =
    p_{(i,j)} \mapsto \overline{N_{q}(p_{(i,j)})}
\]

Notice that $I(u\circ v) = I(u) + I(v)$ and $I(\epsilon) = O$. Again, we will write $I_{u}$ for the application $I(u)$.

Now we will define a relation $\sim$ over $P$. $u \sim v$ if and only if $I(u) = I(v)$. With some thought one
can see that $\sim$ is an equivalence relation. The equivalence class of $u\in P$ will be denoted by $[u]$
and wil be called a \emph{press pattern}.
The set of all press patterns will be denoted by $\Pp$.

Note that for $[s],[t]\in\Pp$ with $[s]=[t]$ we have

\[
    E(s) M = E(t) M
\]

since 

\[
    \left(E(s) M\right)_{(u,v)}
    = M_{(u,v)} + \sum_{d((i,j),(u,v))\leq 1}I_{s}(p_{(i,j)})
    = M_{(u,v)} + \sum_{d((i,j),(u,v))\leq 1}I_{t}(p_{(i,j)})
    = \left(E(t) M\right)_{(u,v)}
 \]

We will define the following binary operation on $\Pp$: $[u] + [v] = [u\circ v]$. Notice that for
$u_{0}, u_{1}, v_{0}, v_{1}\in\Pp$ with $[u_{0}] = [u_{1}]$ and $[v_{0}] = [v_{1}]$ we have.

\[
    [u_{0}\circ v_{0}] =
    \{w\in P|
    I(w)=I(u_{0}\circ v_{0}) = I(u_{0})+I(v_{0}) = I(u_{1})+I(v_{1}) = I(u_{1}\circ v_{1})
    \} =
    [u_{1}\circ v_{1}]
\]

which show that the addition is well defined. Since $I(u) + I(v) = I(v) + I(u)$
$\left(\Pp, +\right)$ is an abelian group.

If we define scalar multiplication with $r\in\Z/n/Z$ as

\[
    r[u] = [\underbrace{u\circ u\circ\ldots u}_{r\text{times}}]
\]

We turn $\Pp$ into a free $\Z/n/Z$-module.

We will construct a map $\Ep$ from $\Pp \rightarrow \Lp$ with $[u] \mapsto E(u) O$. 
Notice that $\Ep(u\circ v) = \Ep(u)(\Ep(v))$ which makes $\Ep$ a linear transformation.

Solving a $s\in\Lp$ amounts to solving for $u\in\Pp$ the linear equation

\[
\Ep(u) = -s
\]

Next we consider an order $\leq$ on $B$, the basic presses. For $p_{(i,j)}, p_{(u,v)}\in B$
we have $p_{(i,j)}\leq p_{(u,v)}$ if and only if $j \lt v$ or when $j = v$ then $i \leq j$.
This ordering amounts to ordering the basic pressed per row from top to bottom and for each row
from left to right.

A press sequence $q:=(q_{t})_{t\in\overline{m}}\in P$ is called a \emph{standard} press sequence when
$q_{i}\leq q_{j}$ whenever $i \leq j$. For each press pattern $P\in\Pp$ there is a unique standard 
press sequence $p\in P$ such that $P = [p]$.
A press sequence is \emph{fertile} if it only contains basic presses with a row index of 0.

We will turn to chasing the light. For each light pattern $L\in\Lp$ we define a press pattern
$chase(L)$. We create a sequence of standard press patterns. $z_{0} := \epsilon$, the empty
press pattern. $z_{n+1} := z_{n} \circ p_{(i,j+1)}$ where $(i,j)$ is the smallest non-zero entry
in $L+\Ep(z_{n})$. $chase(L)$ is the press pattern when the press sequence does not grow anymore.

Chasing amounts to finding a fertile press sequence $p\in P$ such that

\[
chase([p]) = -chase(L)
\]

$chase_{(n,c,r)}$ is a linear transformation.

\begin{definition}
    For a $(n,c,r)$ Light Out puzzle we call $C_{(n,c,r)} := \dimension \Ker chase_{(n,c,r)}$
\end{definition}

When $n$ and $c$ are clear from context we will write $C_{r}$ for $C_{(n,c,r)}$.