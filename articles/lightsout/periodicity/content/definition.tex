\section{Definition}
In this section we will define some terms that we will use throughout the article.

In this article we are exploring rectangular Lights Out. A $(n, c, r)$ Lights Out puzzle
is a matrix $M$ with $c\in\N$ columns and $r\in\N$ rows with entries over $\Z/n\Z$. The space
of all $(n, c, r)$ puzzles is called $\Lp_{(n,c,r)}$.

For a each $\Lp_{(n,c,r)}$ and for each $(i, j)$ with $1\leq i \leq r$ and $1\leq j \leq c$
there is a basic press function $p_{(i,j)}: \Lp_{(n,c,r)} \rightarrow \Lp_{(n,c,r)}$ mapping

\[
    \left(p_{(i,j)}(M)\right)_{(u,v)} :=
    \left\{
    \begin{array}{l@{\quad:\quad}l}
        M_{(u,v)} + 1 & \operatorname{d}\left((i,j), (u,v)\right) \leq 1 \\
        M_{(u,v)} & \text{otherwise}
    \end{array}
    \right.
\]

where $\operatorname{d}$ is the \emph{Manhattan distance}. The set of all basic presses is called $B$.

A press sequence is a finite sequence of basic presses. The set of all press sequences is called $P$.
$P$ together with concatenation of sequences makes a monoid with the empty sequence as identity element.

The effect $E$ of a press sequence is an mapping on $\Lp_{(n,c,r)}$ that extends basic presses.
I.e. the effect of the empty sequence is the identity map and for a press sequence $\left(q_{t})_{t\in\overline{m}}$
with $\overline{m}:=\{0, 1, \ldots, m-1\}$ 

\[
    E\left(\left(q_{t})_{t\in\overline{m}}\right)\right)
    =
    E\left(\left(q_{t+1})_{t\in\overline{m-1}}\right)\right) \circ q_{0}
\]

We will define a fingerprint function $I: P \rightarrow (\Z/n\Z)^{B}$ that count how many times a basic press
is present in the sequence, modulo $n$. So for $\overline{m}:={0, 1, \ldots, m-1}$ we have

\[
    I\left((q_t)_{t\in\overline{m}}\right)
    =
    p_{(i,j)} \mapsto \overline{\sum_{t\in\overline{m}}1_{(i,j)}(q_{t})}
\]

where $1_{(i,j)}(q)$ is $1$ when $q=p_{(i,j)}$ and zero otherwise. Notice that $I(u\circ v) = I(u) + I(v)$ and $I(\epsilon) = O$.

Now we will define a relation $\sim$ over $P$. $u \sim v$ if and only if $I(u) = I(v)$. With some thought one
can see that $\sim$ is an equivalence relation. The equivalence class of $u\in P$ will be denoted by $[u]$
and wil be called a \emph{press pattern}.
The set of all press patterns will be denoted by $\Pp$.

We will define the following binary operation on $\Pp$: $[u] + [v] = [u\circ v]$. Notice that for
$u_{0}, u_{1}, v_{0}, v_{1}\in\Pp$ with $[u_{0}] = [u_{1}]$ and $[v_{0}] = [v_{1}]$ we have.

\[
    [u_{0}\circ v_{0}] = 
    \{w\in P|
    I(w)=I(u_{0}\circ v_{0}) = I(u_{0})+I(v_{0}) = I(u_{1})+I(v_{1}) = I(u_{1}\circ v_{1})
    \} =
    [u_{1}\circ v_{1}]
\]

which show that the addition is well defined. Since $I(u) + I(v) = I(v) + I(u)$
$\left(\Pp, +\right)$ is an abelian group.

If we define scalar multiplication with $r\in\Z/n/Z$ as

\[
    r[u] = [\underbrace{u\circ u\circ\ldots u}_{r\text{times}}]
\]

We turn $\Pp$ into a free $\Z/n/Z$-module.

We will construct a linear map