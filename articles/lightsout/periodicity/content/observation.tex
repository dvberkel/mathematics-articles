\section{Observations}
In discovering the theorems of \cite{martin01} and \cite{leach17} for
themselves, the authors studied the table \ref{kernels}, and its variants,
extensively. To see more extensive tables see \url{http://dvberkel.github.io/mathematics-articles/lights_out/periodicity.html}.

\begin{table}
  \begin{center}
  \begin{tabular}{|c|cccccccccccc|}
    \hline
    & \phantom{0}0 & \phantom{0}1 & \phantom{0}2 & \phantom{0}3 & \phantom{0}4 & \phantom{0}5 & \phantom{0}6 & \phantom{0}7 & \phantom{0}8 & \phantom{0}9 & 10 & 11 \\
    \hline
    \hline
    1 & 0 & 0 & 1 & 0 & 0 & 1 & 0 & 0 & 1 & 0 & 0 & 1 \\
    2 & 0 & 1 & 0 & 2 & 0 & 1 & 0 & 2 & 0 & 1 & 0 & 2 \\
    3 & 0 & 0 & 2 & 0 & 0 & 3 & 0 & 0 & 2 & 0 & 0 & 3 \\
    4 & 0 & 0 & 0 & 0 & 4 & 0 & 0 & 0 & 0 & 4 & 0 & 0 \\
    \hline
  \end{tabular}
  \end{center}
  \caption{Dimension of Kernels}\label{kernels}
\end{table}

During our studies we made the following observations. For each row

\begin{enumerate}
\item \label{observation.periodic} The sequence is periodic.
\item \label{observation.maximal} There is a kernel of maximal dimension.
\item \label{observation.period} The period starts after the maximal dimension.
\item \label{observation.surround} Before and after the maximal dimension
  the dimension is $0$.
\item \label{observation.palindromic} The sequence is almost palindromic.
\end{enumerate}
