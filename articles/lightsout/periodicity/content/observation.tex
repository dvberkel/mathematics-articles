\section{Observations}
In discovering the theorems of \cite{martin01} and \cite{leach17} for
themselves, the authors studied the table \ref{kernels}, and its variants,
extensively. To see more entries of the table goto the website for this
article\footnote{\url{http://dvberkel.github.io/mathematics-articles/lights_out/periodicity.html}}.

\begin{table}
  \begin{center}
    \begin{tabular}{|c|cccccccccccc|}
      \hline
        & \phantom{0}0 & \phantom{0}1 & \phantom{0}2 & \phantom{0}3 & \phantom{0}4 & \phantom{0}5 & \phantom{0}6 & \phantom{0}7 & \phantom{0}8 & \phantom{0}9 & 10 & 11 \\
      \hline
      \hline
      1 & 0            & 0            & 1            & 0            & 0            & 1            & 0            & 0            & 1            & 0            & 0  & 1  \\
      2 & 0            & 1            & 0            & 2            & 0            & 1            & 0            & 2            & 0            & 1            & 0  & 2  \\
      3 & 0            & 0            & 2            & 0            & 0            & 3            & 0            & 0            & 2            & 0            & 0  & 3  \\
      4 & 0            & 0            & 0            & 0            & 4            & 0            & 0            & 0            & 0            & 4            & 0  & 0  \\
      \hline
    \end{tabular}
  \end{center}
  \caption{Dimension of Kernels}\label{kernels}
\end{table}

During our studies we made the following observations. For each row

\begin{enumerate}
  \item \label{observation.periodic} The sequence is purely periodic.
  \item \label{observation.maximal} There is a kernel of maximal dimension.
  \item \label{observation.period} The period starts after the maximal dimension.
  \item \label{observation.consecutive} The sum of two consecutive dimensions is less than or equal to the maximal dimension.
  \item \label{observation.palindromic} The sequence is almost palindromic.
\end{enumerate}

We will explain each observation with the aid of the above table.
Observation \ref{observation.periodic} pertains that for each number of columns, the sequence of dimensions of kernels
of $c\times r$ Lights Out puzzles repeats itself eventually, and when it repeats it does so from the start.

A kernel of maximal dimension is when the kernel is the entire press space. In other words, in the sequence for the number of columns $c$,
there is an entry of precisely $c$. Furthermore, the sequence repeats itself when this happens.

For observation \ref{observation.palindromic}, take a look at the third row of the above table. The sequence up until the value 3, i.e. $0, 0, 2, 0, 0$ is palindromic.
It is the same read from left to right as from right to left.

The fact that in a row consecutive numbers sum to less than the maximal dimension is observation \ref{observation.consecutive}.



