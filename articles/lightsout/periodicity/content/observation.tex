\section{Observations}
In discovering the theorems of \cite{martin01} and \cite{leach17} for
themselves, the authors studied the table \ref{kernels}, and its variants, extensively.

\begin{table}
  \caption{Dimension of Kernels}\label{kernels}
  \begin{tabular}{|c|cccccccccccc|}
    \hline
    & \phantom{0}0 & \phantom{0}1 & \phantom{0}2 & \phantom{0}3 & \phantom{0}4 & \phantom{0}5 & \phantom{0}6 & \phantom{0}7 & \phantom{0}8 & \phantom{0}9 & 10 & 11 \\
    \hline
    \hline
    1 & 0 & 0 & 1 & 0 & 0 & 1 & 0 & 0 & 1 & 0 & 0 & 1 \\
    2 & 0 & 1 & 0 & 2 & 0 & 1 & 0 & 2 & 0 & 1 & 0 & 2 \\
    3 & 0 & 0 & 2 & 0 & 0 & 3 & 0 & 0 & 2 & 0 & 0 & 3 \\
    4 & 0 & 0 & 0 & 0 & 4 & 0 & 0 & 0 & 0 & 4 & 0 & 0 \\
    \hline
  \end{tabular}
\end{table}

The following observations can be made.

\begin{itemize}
\item The sequence is periodic.
\item The period start when dimension is $c$.
\item Before and after the dimension is $0$.
\item The sequence is almost palindromic.
\end{itemize}

