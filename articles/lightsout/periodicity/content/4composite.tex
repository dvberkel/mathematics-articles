\section{Composite colours}\label{sec:comp}
In this section we will assume that the number of colours $n$
will be a composite number. Trouble is then caused by the fact
that $\znz$ is a commutative ring that contains zero divisors,
so it will not be field. That means that the linear algebra we
want to do is not taking place in a vector space, but rather
in the module $(\znz)^c$.

The results (and proofs) from Section \ref{sec:algebra} still hold:
the matrix $W_c\in\Mat_{r,c}(\znz)$ is invertible, and its
powers form a purely periodic sequence, the four blocks of which
are given by the simple recurrence given by the Structure Lemma.

The complications arise when we try to explicitly compute
the numbers of solvable configurations (and the number of
different solutions in each case) as given by Corollary \ref{cor:numbers}.
This is due to the extra care the notions of `independence'
and `basis' require in this case: it is, for example, not true that
in any set of dependent vectors one one of them will be a linear combination
of the others (in particular, 2 dependent vectors are not
necessarily multiples of each other),
and neither is it true that vectors in $(\znz)^c$ can only span
subspaces of $n^k$ vectors, for some $k$ with $0\leq k\leq c$.

What is still true, however, is the following alternative for
the Dimension Theorem, which implies that for a $k\times k$
matrix $M$ over a field the sum of the dimension of the kernel
of $M$ and its (row) rank are $k$:
the number of elements $K_M$ in $(\znz)^c$ that are in the kernel
of $M$ and the number of elements $R_M$ $(\znz)^c$ that are in the 
span of the rows of $M$ are related by $K_M\cdot R_M=n^c$.
This holds (as in the case where $\znz$ is field) because
the equivalence relation $x \sim y\iff xM=yM$ partitions
$(\znz)^c$ into cosets of size $M$ of elements
with the same image.

As a consequence, the observations from the Introduction do not
generally hold in this case. Nonetheless, the following still holds.

\begin{theorem}
For a fixed number of colours $n$ and a fixed number of columns $c$,
the number of solvable boards in $\Lp(n, c, r)$ forms a purely periodic
sequence (as a function of the number of rows $r$), and so does the
number of different solutions for each of these solvable boards.
The common period of these sequences is a divisor of the period of the sequence
of powers $W_c^r$ of $W_c\in\Mat_{r,c}(\znz)$.
\end{theorem}

\begin{proof}
This is an immediate consequence of Corollary \ref{cor:numbers},
Theorem \ref{thm:Wperiod}, and Lemma \ref{lem_struc} and the remarks
above.
\end{proof}

\begin{table}\label{tab:n4}
  \begin{center}
    \begin{adjustbox}{max width=\textwidth}
    \begin{tabular}{|c|cccccccccccccccc|c|}
      \hline
$r\ \backslash\ c$:&0 &1 &2 &3 &4 &5 &6 &7 &8 &9 &10 &11 &12 &13 &14  &$\ell$
\\
      \hline
$1$ &$1$ &$1$ &$2^2$ &$1$ &$1$ &$2^2$ &$1$ &$1$ &$2^2$ &$1$ &$1$ &$2^2$ &$1$ &$1$ &$2^2$ &$3$ \\
%$2$ &$1$ &$4$ &$1$ &$8$ &$1$ &$4$ &$1$ &$16$ &$1$ &$4$ &$1$ &$8$ &$1$ &$4$ &$1$ &$8$ \\
$2$ &$1$ &$2^2$ &$1$ &$2^3$ &$1$ &$2^2$ &$1$ &$2^4$ &$1$ &$2^2$ &$1$ &$2^3$ &$1$ &$2^2$ &$1$ &$8$ \\
%$3$ &$1$ &$1$ &$8$ &$1$ &$1$ &$64$ &$1$ &$1$ &$8$ &$1$ &$1$ &$64$ &$1$ &$1$ &$8$ &$6$ \\
$3$ &$1$ &$1$ &$2^3$ &$1$ &$1$ &$2^6$ &$1$ &$1$ &$2^3$ &$1$ &$1$ &$2^6$ &$1$ &$1$ &$2^3$ &$6$ \\
%$4$ &$1$ &$1$ &$1$ &$1$ &$64$ &$1$ &$1$ &$1$ &$1$ &$256$ &$1$ &$1$ &$1$ &$1$ &$64$ &$10$ \\
$4$ &$1$ &$1$ &$1$ &$1$ &$2^6$ &$1$ &$1$ &$1$ &$1$ &$2^8$ &$1$ &$1$ &$1$ &$1$ &$2^6$ &$10$ \\
%$5$ &$1$ &$4$ &$4$ &$64$ &$1$ &$16$ &$1$ &$128$ &$4$ &$4$ &$1$ &$256$ &$1$ &$4$ &$4$ &$48$ \\
$5$ &$1$ &$2^2$ &$2^2$ &$2^6$ &$1$ &$2^4$ &$1$ &$2^7$ &$2^2$ &$2^2$ &$1$ &$2^8$ &$1$ &$2^2$ &$2^2$ &$48$ \\
%$6$ &$1$ &$1$ &$1$ &$1$ &$1$ &$1$ &$1$ &$1$ &$512$ &$1$ &$1$ &$1$ &$1$ &$1$ &$1$ &$18$ \\
$6$ &$1$ &$1$ &$1$ &$1$ &$1$ &$1$ &$1$ &$1$ &$2^9$ &$1$ &$1$ &$1$ &$1$ &$1$ &$1$ &$18$ \\
%$7$ &$1$ &$1$ &$16$ &$1$ &$1$ &$128$ &$1$ &$1$ &$16$ &$1$ &$1$ &$8192$ &$1$ &$1$ &$16$ &$24$ \\
$7$ &$1$ &$1$ &$2^4$ &$1$ &$1$ &$2^7$ &$1$ &$1$ &$2^4$ &$1$ &$1$ &$2^{13}$ &$1$ &$1$ &$2^4$ &$24$ \\
%$8$ &$1$ &$4$ &$1$ &$8$ &$1$ &$4$ &$512$ &$16$ &$1$ &$4$ &$1$ &$8$ &$1$ &$16384$ &$1$ &$56$ \\
$8$ &$1$ &$2^2$ &$1$ &$2^3$ &$1$ &$2^2$ &$2^9$ &$2^4$ &$1$ &$2^2$ &$1$ &$2^3$ &$1$ &$2^{14}$ &$1$ &$56$ \\
%$9$ &$1$ &$1$ &$4$ &$1$ &$256$ &$4$ &$1$ &$1$ &$4$ &$4096$ &$1$ &$4$ &$1$ &$1$ &$1024$ &$60$ \\
$9$ &$1$ &$1$ &$2^2$ &$1$ &$2^8$ &$2^2$ &$1$ &$1$ &$2^2$ &$2^{12}$ &$1$ &$2^2$ &$1$ &$1$ &$2^{10}$ &$60$ \\
%$10$ &$1$ &$1$ &$1$ &$1$ &$1$ &$1$ &$1$ &$1$ &$1$ &$1$ &$1$ &$1$ &$1$ &$1$ &$1$ &$62$ \\
$10$ &$1$ &$1$ &$1$ &$1$ &$1$ &$1$ &$1$ &$1$ &$1$ &$1$ &$1$ &$1$ &$1$ &$1$ &$1$ &$62$ \\
%$11$ &$1$ &$4$ &$8$ &$64$ &$1$ &$256$ &$1$ &$8192$ &$8$ &$4$ &$1$ &$4096$ &$1$ &$4$ &$8$ &$96$ \\
$11$ &$1$ &$2^2$ &$2^3$ &$2^6$ &$1$ &$2^8$ &$1$ &$2^{13}$ &$2^3$ &$2^2$ &$1$ &$2^{12}$ &$1$ &$2^2$ &$2^3$ &$96$ \\
%$12$ &$1$ &$1$ &$1$ &$1$ &$1$ &$1$ &$1$ &$1$ &$1$ &$1$ &$1$ &$1$ &$1$ &$1$ &$1$ &$126$ \\
$12$ &$1$ &$1$ &$1$ &$1$ &$1$ &$1$ &$1$ &$1$ &$1$ &$1$ &$1$ &$1$ &$1$ &$1$ &$1$ &$126$ \\
%$13$ &$1$ &$1$ &$4$ &$1$ &$1$ &$4$ &$1$ &$1$ &$16384$ &$1$ &$1$ &$4$ &$1$ &$1$ &$4$ &$36$ \\
$13$ &$1$ &$1$ &$2^2$ &$1$ &$1$ &$2^2$ &$1$ &$1$ &$2^{14}$ &$1$ &$1$ &$2^2$ &$1$ &$1$ &$2^2$ &$36$ \\
%$14$ &$1$ &$4$ &$1$ &$8$ &$64$ &$4$ &$1$ &$16$ &$1$ &$1024$ &$1$ &$8$ &$1$ &$4$ &$64$ &$680$ \\
$14$ &$1$ &$2^2$ &$1$ &$2^3$ &$2^6$ &$2^2$ &$1$ &$2^4$ &$1$ &$2^{10}$ &$1$ &$2^3$ &$1$ &$2^2$ &$2^6$ &$680$ \\
%$15$ &$1$ &$1$ &$16$ &$1$ &$1$ &$256$ &$1$ &$1$ &$16$ &$1$ &$1$ &$32768$ &$1$ &$1$ &$16$ &$48$ \\
$15$ &$1$ &$1$ &$2^4$ &$1$ &$1$ &$2^8$ &$1$ &$1$ &$2^4$ &$1$ &$1$ &$2^{15}$ &$1$ &$1$ &$2^4$ &$48$ \\
%$16$ &$1$ &$1$ &$1$ &$1$ &$1$ &$1$ &$1$ &$1$ &$1$ &$1$ &$1$ &$1$ &$1$ &$1$ &$4096$ &$510$ \\
$16$ &$1$ &$1$ &$1$ &$1$ &$1$ &$1$ &$1$ &$1$ &$1$ &$1$ &$1$ &$1$ &$1$ &$1$ &$2^{12}$ &$510$ \\
%$17$ &$1$ &$4$ &$4$ &$64$ &$1$ &$16$ &$4096$ &$128$ &$4$ &$4$ &$1$ &$256$ &$1$ &$1048576$ &$4$ &$336$ \\
$17$ &$1$ &$2^2$ &$2^2$ &$2^6$ &$1$ &$2^4$ &$2^{12}$ &$2^7$ &$2^2$ &$2^2$ &$1$ &$2^8$ &$1$ &$2^{20}$ &$2^2$ &$336$ \\
%$18$ &$1$ &$1$ &$1$ &$1$ &$1$ &$1$ &$1$ &$1$ &$1$ &$1$ &$1$ &$1$ &$1$ &$1$ &$1$ &$1026$ \\
$18$ &$1$ &$1$ &$1$ &$1$ &$1$ &$1$ &$1$ &$1$ &$1$ &$1$ &$1$ &$1$ &$1$ &$1$ &$1$ &$1026$ \\
%$19$ &$1$ &$1$ &$8$ &$1$ &$256$ &$64$ &$1$ &$1$ &$8$ &$65536$ &$1$ &$64$ &$1$ &$1$ &$2048$ &$120$ \\
$19$ &$1$ &$1$ &$2^3$ &$1$ &$2^8$ &$2^6$ &$1$ &$1$ &$2^3$ &$2^{16}$ &$1$ &$2^6$ &$1$ &$1$ &$2^{11}$ &$120$ \\
%$20$ &$1$ &$4$ &$1$ &$8$ &$1$ &$4$ &$1$ &$16$ &$512$ &$4$ &$1$ &$8$ &$1$ &$4$ &$1$ &$4680$ \\
$20$ &$1$ &$2^2$ &$1$ &$2^3$ &$1$ &$2^2$ &$1$ &$2^4$ &$2^9$ &$2^2$ &$1$ &$2^3$ &$1$ &$2^2$ &$1$ &$4680$ \\
%$21$ &$1$ &$1$ &$4$ &$1$ &$1$ &$4$ &$1$ &$1$ &$4$ &$1$ &$1$ &$4$ &$1$ &$1$ &$4$ &$372$ \\
$21$ &$1$ &$1$ &$2^2$ &$1$ &$1$ &$2^2$ &$1$ &$1$ &$2^2$ &$1$ &$1$ &$2^2$ &$1$ &$1$ &$2^2$ &$372$ \\
%$22$ &$1$ &$1$ &$1$ &$1$ &$1$ &$1$ &$1$ &$1$ &$1$ &$1$ &$1$ &$1$ &$1$ &$1$ &$1$ &$4094$ \\
$22$ &$1$ &$1$ &$1$ &$1$ &$1$ &$1$ &$1$ &$1$ &$1$ &$1$ &$1$ &$1$ &$1$ &$1$ &$1$ &$4094$ \\
%$23$ &$1$ &$4$ &$16$ &$64$ &$1$ &$512$ &$1$ &$16384$ &$16$ &$4$ &$1$ &$524288$ &$1$ &$4$ &$16$ &$192$ \\
$23$ &$1$ &$2^2$ &$2^4$ &$2^6$ &$1$ &$2^9$ &$1$ &$2^{14}$ &$2^4$ &$2^2$ &$1$ &$2^{19}$ &$1$ &$2^2$ &$2^4$ &$192$ \\
%$24$ &$1$ &$1$ &$1$ &$1$ &$64$ &$1$ &$1$ &$1$ &$1$ &$256$ &$1$ &$1$ &$1$ &$1$ &$64$ &$2050$ \\
$24$ &$1$ &$1$ &$1$ &$1$ &$2^6$ &$1$ &$1$ &$1$ &$1$ &$2^8$ &$1$ &$1$ &$1$ &$1$ &$2^6$ &$2050$ \\
%$25$ &$1$ &$1$ &$4$ &$1$ &$1$ &$4$ &$1$ &$1$ &$4$ &$1$ &$1$ &$4$ &$1$ &$1$ &$4$ &$252$ \\
$25$ &$1$ &$1$ &$2^2$ &$1$ &$1$ &$2^2$ &$1$ &$1$ &$2^2$ &$1$ &$1$ &$2^2$ &$1$ &$1$ &$2^2$ &$252$ \\
%$26$ &$1$ &$4$ &$1$ &$8$ &$1$ &$4$ &$512$ &$16$ &$1$ &$4$ &$1$ &$8$ &$1$ &$16384$ &$1$ &$4088$ \\
$26$ &$1$ &$2^2$ &$1$ &$2^3$ &$1$ &$2^2$ &$2^9$ &$2^4$ &$1$ &$2^2$ &$1$ &$2^3$ &$1$ &$2^{14}$ &$1$ &$4088$ \\
%$27$ &$1$ &$1$ &$8$ &$1$ &$1$ &$64$ &$1$ &$1$ &$32768$ &$1$ &$1$ &$64$ &$1$ &$1$ &$8$ &$72$ \\
$27$ &$1$ &$1$ &$2^3$ &$1$ &$1$ &$2^6$ &$1$ &$1$ &$2^{15}$ &$1$ &$1$ &$2^6$ &$1$ &$1$ &$2^3$ &$72$ \\
%$28$ &$1$ &$1$ &$1$ &$1$ &$1$ &$1$ &$1$ &$1$ &$1$ &$1$ &$1$ &$1$ &$1$ &$1$ &$1$ &$6554$ \\
$28$ &$1$ &$1$ &$1$ &$1$ &$1$ &$1$ &$1$ &$1$ &$1$ &$1$ &$1$ &$1$ &$1$ &$1$ &$1$ &$6554$ \\
%$29$ &$1$ &$4$ &$4$ &$64$ &$256$ &$16$ &$1$ &$128$ &$4$ &$16384$ &$1$ &$256$ &$1$ &$4$ &$1024$ &$4080$ \\
$29$ &$1$ &$2^2$ &$2^2$ &$2^6$ &$2^8$ &$2^4$ &$1$ &$2^7$ &$2^2$ &$2^{14}$ &$1$ &$2^8$ &$1$ &$2^2$ &$2^{10}$ &$4080$ \\
%$30$ &$1$ &$1$ &$1$ &$1$ &$1$ &$1$ &$1$ &$1$ &$1$ &$1$ &$32768$ &$1$ &$1$ &$1$ &$1$ &$682$ \\
$30$ &$1$ &$1$ &$1$ &$1$ &$1$ &$1$ &$1$ &$1$ &$1$ &$1$ &$2^{15}$ &$1$ &$1$ &$1$ &$1$ &$682$ \\
%$31$ &$1$ &$1$ &$16$ &$1$ &$1$ &$256$ &$1$ &$1$ &$16$ &$1$ &$1$ &$65536$ &$1$ &$1$ &$16$ &$96$ \\
$31$ &$1$ &$1$ &$2^4$ &$1$ &$1$ &$2^8$ &$1$ &$1$ &$2^4$ &$1$ &$1$ &$2^{16}$ &$1$ &$1$ &$2^4$ &$96$ \\
%$32$ &$1$ &$4$ &$1$ &$8$ &$1$ &$4$ &$1$ &$16$ &$1$ &$4$ &$1$ &$8$ &$1$ &$4$ &$1$ &$8184$ \\
$32$ &$1$ &$2^2$ &$1$ &$2^3$ &$1$ &$2^2$ &$1$ &$2^4$ &$1$ &$2^2$ &$1$ &$2^3$ &$1$ &$2^2$ &$1$ &$8184$ \\

      \hline
      \hline
    \end{tabular}
    \end{adjustbox}
  \end{center}
  \caption{Size of kernels and period $\ell$ for $n=4$}
\end{table}

{\tiny\rm

\begin{table}\label{tab:n6}
  \begin{center}
    \begin{adjustbox}{max width=\textwidth}
    \begin{tabular}{|c|cccccccccccc|c|}
      \hline
$r\ \backslash\ c$:&0 &1 &2 &3 &4 &5 &6 &7 &8 &9 &10 % &11 &12 &13 &14 &15 
&$\ell$
\\
      \hline
      \hline
	    0 & 1 & 1 & 1 & 1 & 1 & 1 & 1 & 1 &1 &1 &1 &1\\
%1 &1 &1 &6 &1 &1 &6 &1 &1 &6 &1 &1 &3\\
1 &$1$ &$1$ &$2\cdot 3$ &$1$ &$1$ &$2\cdot 3$ &$1$ &$1$ &$2\cdot 3$ &$1$ &$1$ &3\\
%2 &1 &6 &3 &12 &1 &18 &1 &12 &3 &6 &1 &12\\
2 &$1$ &$2\cdot 3$ &$3$ &$2^{2}\cdot 3$ &$1$ &$2\cdot 3^{2}$ &$1$ &$2^{2}\cdot 3$ &$3$ &$2\cdot 3$ &$1$ &12\\
%3 &1 &1 &12 &1 &9 &24 &1 &1 &12 &9 &1 &30\\
3 &$1$ &$1$ &$2^{2}\cdot 3$ &$1$ &$3^{2}$ &$2^{3}\cdot 3$ &$1$ &$1$ &$2^{2}\cdot 3$ &$3^{2}$ &$1$ &30\\
%4 &1 &1 &1 &9 &144 &1 &1 &9 &1 &144 &1 &20\\
4 &$1$ &$1$ &$1$ &$3^{2}$ &$2^{4}\cdot 3^{2}$ &$1$ &$1$ &$3^{2}$ &$1$ &$2^{4}\cdot 3^{2}$ &$1$ &20\\
%5 &1 &6 &18 &24 &1 &108 &1 &48 &162 &6 &1 &72\\
5 &$1$ &$2\cdot 3$ &$2\cdot 3^{2}$ &$2^{3}\cdot 3$ &$1$ &$2^{2}\cdot 3^{3}$ &$1$ &$2^{4}\cdot 3$ &$2\cdot 3^{4}$ &$2\cdot 3$ &$1$ &72\\
%6 &1 &1 &1 &1 &1 &1 &1 &1 &64 &1 &1 &1638\\
6 &$1$ &$1$ &$1$ &$1$ &$1$ &$1$ &$1$ &$1$ &$2^{6}$ &$1$ &$1$ &1638\\
%7 &1 &1 &12 &1 &9 &48 &1 &1 &12 &9 &1 &120\\
7 &$1$ &$1$ &$2^{2}\cdot 3$ &$1$ &$3^{2}$ &$2^{4}\cdot 3$ &$1$ &$1$ &$2^{2}\cdot 3$ &$3^{2}$ &$1$ &120\\
%8 &1 &6 &3 &12 &1 &162 &64 &12 &81 &6 &1 &252\\
8 &$1$ &$2\cdot 3$ &$3$ &$2^{2}\cdot 3$ &$1$ &$2\cdot 3^{4}$ &$2^{6}$ &$2^{2}\cdot 3$ &$3^{4}$ &$2\cdot 3$ &$1$ &252\\
%9 &1 &1 &6 &9 &144 &6 &1 &9 &6 &2304 &1 &2460\\
9 &$1$ &$1$ &$2\cdot 3$ &$3^{2}$ &$2^{4}\cdot 3^{2}$ &$2\cdot 3$ &$1$ &$3^{2}$ &$2\cdot 3$ &$2^{8}\cdot 3^{2}$ &$1$ &2460\\
%10 &1 &1 &1 &1 &1 &1 &1 &1 &1 &1 &1 &3782\\
10 &$1$ &$1$ &$1$ &$1$ &$1$ &$1$ &$1$ &$1$ &$1$ &$1$ &$1$ &3782\\
%11 &1 &6 &36 &24 &9 &432 &1 &384 &324 &54 &1 &720\\
11 &$1$ &$2\cdot 3$ &$2^{2}\cdot 3^{2}$ &$2^{3}\cdot 3$ &$3^{2}$ &$2^{4}\cdot 3^{3}$ &$1$ &$2^{7}\cdot 3$ &$2^{2}\cdot 3^{4}$ &$2\cdot 3^{3}$ &$1$ &720\\
%12 &1 &1 &1 &1 &1 &1 &27 &1 &1 &1 &1 &1638\\
12 &$1$ &$1$ &$1$ &$1$ &$1$ &$1$ &$3^{3}$ &$1$ &$1$ &$1$ &$1$ &1638\\
%13 &1 &1 &6 &1 &1 &6 &27 &1 &384 &1 &1 &1638\\
13 &$1$ &$1$ &$2\cdot 3$ &$1$ &$1$ &$2\cdot 3$ &$3^{3}$ &$1$ &$2^{7}\cdot 3$ &$1$ &$1$ &1638\\
%14 &1 &6 &3 &108 &144 &18 &1 &108 &3 &864 &1 &1020\\
14 &$1$ &$2\cdot 3$ &$3$ &$2^{2}\cdot 3^{3}$ &$2^{4}\cdot 3^{2}$ &$2\cdot 3^{2}$ &$1$ &$2^{2}\cdot 3^{3}$ &$3$ &$2^{5}\cdot 3^{3}$ &$1$ &1020\\
%15 &1 &1 &12 &1 &9 &48 &1 &1 &12 &9 &1 &9840\\
15 &$1$ &$1$ &$2^{2}\cdot 3$ &$1$ &$3^{2}$ &$2^{4}\cdot 3$ &$1$ &$1$ &$2^{2}\cdot 3$ &$3^{2}$ &$1$ &9840\\
%16 &1 &1 &1 &1 &1 &1 &1 &1 &1 &1 &1 &2017815\\
16 &$1$ &$1$ &$1$ &$1$ &$1$ &$1$ &$1$ &$1$ &$1$ &$1$ &$1$ &2017815\\
%17 &1 &6 &18 &24 &1 &972 &64 &48 &13122 &6 &1 &1512\\
17 &$1$ &$2\cdot 3$ &$2\cdot 3^{2}$ &$2^{3}\cdot 3$ &$1$ &$2^{2}\cdot 3^{5}$ &$2^{6}$ &$2^{4}\cdot 3$ &$2\cdot 3^{8}$ &$2\cdot 3$ &$1$ &1512\\
%18 &1 &1 &1 &1 &1 &1 &1 &1 &1 &1 &1 &2615088294\\
18 &$1$ &$1$ &$1$ &$1$ &$1$ &$1$ &$1$ &$1$ &$1$ &$1$ &$1$ &2615088294\\
%19 &1 &1 &12 &9 &1296 &24 &1 &9 &12 &20736 &1 &2460\\
19 &$1$ &$1$ &$2^{2}\cdot 3$ &$3^{2}$ &$2^{4}\cdot 3^{4}$ &$2^{3}\cdot 3$ &$1$ &$3^{2}$ &$2^{2}\cdot 3$ &$2^{8}\cdot 3^{4}$ &$1$ &2460\\
%20 &1 &6 &3 &12 &1 &18 &1 &12 &192 &6 &1 &16380\\
20 &$1$ &$2\cdot 3$ &$3$ &$2^{2}\cdot 3$ &$1$ &$2\cdot 3^{2}$ &$1$ &$2^{2}\cdot 3$ &$2^{6}\cdot 3$ &$2\cdot 3$ &$1$ &16380\\
%21 &1 &1 &6 &1 &1 &6 &1 &1 &6 &1 &1 &1372866\\
21 &$1$ &$1$ &$2\cdot 3$ &$1$ &$1$ &$2\cdot 3$ &$1$ &$1$ &$2\cdot 3$ &$1$ &$1$ &1372866\\
%22 &1 &1 &1 &1 &1 &1 &1 &1 &1 &1 &1 &7882997\\
22 &$1$ &$1$ &$1$ &$1$ &$1$ &$1$ &$1$ &$1$ &$1$ &$1$ &$1$ &7882997\\
%23 &1 &6 &36 &24 &9 &864 &1 &384 &324 &54 &1 &1440\\
23 &$1$ &$2\cdot 3$ &$2^{2}\cdot 3^{2}$ &$2^{3}\cdot 3$ &$3^{2}$ &$2^{5}\cdot 3^{3}$ &$1$ &$2^{7}\cdot 3$ &$2^{2}\cdot 3^{4}$ &$2\cdot 3^{3}$ &$1$ &1440\\
%24 &1 &1 &1 &9 &144 &1 &1 &9 &1 &144 &1 &35739540100\\
24 &$1$ &$1$ &$1$ &$3^{2}$ &$2^{4}\cdot 3^{2}$ &$1$ &$1$ &$3^{2}$ &$1$ &$2^{4}\cdot 3^{2}$ &$1$ &35739540100\\
%25 &1 &1 &6 &1 &1 &6 &27 &1 &6 &1 &1 &1638\\
25 &$1$ &$1$ &$2\cdot 3$ &$1$ &$1$ &$2\cdot 3$ &$3^{3}$ &$1$ &$2\cdot 3$ &$1$ &$1$ &1638\\
%26 &1 &6 &3 &12 &1 &162 &64 &12 &81 &6 &1 &55188\\
26 &$1$ &$2\cdot 3$ &$3$ &$2^{2}\cdot 3$ &$1$ &$2\cdot 3^{4}$ &$2^{6}$ &$2^{2}\cdot 3$ &$3^{4}$ &$2\cdot 3$ &$1$ &55188\\
%27 &1 &1 &12 &1 &9 &24 &27 &1 &768 &9 &1 &1195740\\
27 &$1$ &$1$ &$2^{2}\cdot 3$ &$1$ &$3^{2}$ &$2^{3}\cdot 3$ &$3^{3}$ &$1$ &$2^{8}\cdot 3$ &$3^{2}$ &$1$ &1195740\\
%28 &1 &1 &1 &1 &1 &1 &1 &1 &1 &1 &1 &646269386852620\\
28 &$1$ &$1$ &$1$ &$1$ &$1$ &$1$ &$1$ &$1$ &$1$ &$1$ &$1$ &646269386852620\\
%29 &1 &6 &18 &216 &144 &108 &1 &432 &162 &13824 &1 &250920\\
29 &$1$ &$2\cdot 3$ &$2\cdot 3^{2}$ &$2^{3}\cdot 3^{3}$ &$2^{4}\cdot 3^{2}$ &$2^{2}\cdot 3^{3}$ &$1$ &$2^{4}\cdot 3^{3}$ &$2\cdot 3^{4}$ &$2^{9}\cdot 3^{3}$ &$1$ &250920\\
%30 &1 &1 &1 &1 &1 &1 &1 &1 &1 &1 &1024 &51472783023662\\
30 &$1$ &$1$ &$1$ &$1$ &$1$ &$1$ &$1$ &$1$ &$1$ &$1$ &$2^{10}$&51472783023662\\
%31 &1 &1 &12 &1 &9 &48 &1 &1 &12 &9 &1 &64570080\\
31 &$1$ &$1$ &$2^{2}\cdot 3$ &$1$ &$3^{2}$ &$2^{4}\cdot 3$ &$1$ &$1$ &$2^{2}\cdot 3$ &$3^{2}$ &$1$ &64570080\\
%32 &1 &6 &3 &12 &1 &18 &1 &12 &3 &6 &1 &249612\\
32 &$1$ &$2\cdot 3$ &$3$ &$2^{2}\cdot 3$ &$1$ &$2\cdot 3^{2}$ &$1$ &$2^{2}\cdot 3$ &$3$ &$2\cdot 3$ &$1$ &249612\\
      \hline
    \end{tabular}
    \end{adjustbox}
  \end{center}
  \caption{Size of kernels and period $\ell$ for $n=6$}
\end{table}

}
\begin{example}
In Tables \ref{tab:n4} and \ref{tab:n6} we list a small part of the
sequences over $\Z/4\Z$ and $\Z/6\Z$. Let us consider the
entry for $c=5$ in the latter table in more detail.

The matrix $W_5$ has order $72$ in this case (as the rightmost entry
in row $r=5$ in Table \ref{tab:n6} indicates). For an $r\times 5$
board, the information we are looking for is provided by the
$5\times 5$ matrix $T_r$, which is the upper left submatrix of $W_5^r$.
If we computer the numbers of elements of the kernel of $T_r$ as a
function of $r$, for $0\leq r\leq 71$, we find the following
sequence (the initial segment
of which corresponds with the column $c=5$ of the table):
$1$, $6$, $18$, $24$, $1$, $108$, $1$, $48$, $162$, $6$, $1$, $432$, $1$, $6$, $18$, $48$, $1$, $972$, $1$, $24$, $18$, $
6$, $1$, $864$, $1$, $6$, $162$, $24$, $1$, $108$, $1$, $48$, $18$, $6$, $1$, $3888$, $1$, $6$, $18$, $48$, $1$, $108$, $
1$, $24$, $162$, $6$, $1$, $864$, $1$, $6$, $18$, $24$, $1$, $972$, $1$, $48$, $18$, $6$, $1$, $432$, $1$, $6$, $162$, $
48$, $1$, $108$, $1$, $24$, $18$, $6$, $1$, $7776$.
Note that$, $indeed$, $for $r=71$ we find $6^5$ elements in the kernel
for the first time: $T_{71}$ is the zero-matrix.

We take a closer look at the case $r=5$, so the $5\times 5$ board.
The full matrix $W_5^5$, of which $T_5$ is the upper left block)
equals
$$T_5 := \left(
\begin{array}{cccccccccc}
4&5&5&2&1&4&0&0&4&1\\
5&3&1&0&2&0&4&4&1&4\\
5&1&4&1&5&0&4&5&4&0\\
2&0&1&3&5&4&1&4&4&0\\
1&2&5&5&4&1&4&0&0&4\\
2&0&0&2&5&2&3&3&1&0\\
0&2&2&5&2&3&5&4&3&1\\
0&2&1&2&0&3&4&5&4&3\\
2&5&2&2&0&1&3&4&5&3\\
5&2&0&0&2&0&1&3&3&2\\
\end{array}
\right)$$
and a small computation shows that the kernel of $T_5$
is generated by the following 3 independent vectors over $\Z/6\Z$:
$$(1, 0, 1, 0, 3), (0, 1, 1, 5, 4), (0, 0, 2, 0, 2).$$
Note that the third of these only spans a submodule of size $3$,
whence the kernel consists of $6^2\cdot 3$ instead of $6^3$
elements. This vector correpsonds to the following
pressing pattern in the kernel:
$$\left(
\begin{array}{cccccccccc}
0&0&2&0&2\\
0&4&4&2&4\\
2&4&0&2&4\\
0&2&2&4&2\\
2&4&4&2&2\\
\end{array}
\right)$$
The total number of light patterns that can be created from
a totally unlit board (or equivalently, the number of different
light patterns that can be turned off completely) equals 
$\frac{6^{25}}{6^2\cdot 3}=263243407684534272$.
\end{example}
\section{Periods}
Finally, in this section, we explain how the periods in our table were
computed. Neither computing the powers of $W_c$ nor the recursion for
$T_{c,r}$ will sufficiently efficient to be able to determine
the periods we want; however, using Theorem \ref{thm:Wperiod} and Lemma
\ref{lem:half} we will be able to determine the length of the period
(but not the values in it) if we can compute the order of the matrix $W_c$.
It turns out that $W_c$ is a symplectic matrix, and the order of
the group of all $2c\times 2c$ symplectic matrices over the finite field
$\FF_p$ can be found in many textbooks; for symplectic matrices over
$\znz$ generally, it took some effort to find the following
formula in \cite{newman72}, p.~136.
\begin{theorem}\label{thm:newman}
The order of the group of $2c\times 2c$ symplectic matrices over $\znz$
equals
$$G(c,n)=n^{2c^2+c}\cdot\prod_{p\vert c\atop {p\ {\rm prime}}}\cdot\prod_{k=1}^c \left(1-\frac{1}{p^{2k}}\right).$$
\end{theorem}
Clearly, the order of $W_c$ over $\znz$ is a divisor of the group
order given by the theorem. The group order $G(c,n)$  grows quickly with $c$,
and the order of $W_c$ is usually much smaller. Fortunately, the
group order is highly composite, as a product of reasonably small
primes and this enabled us to compute the order of $W_c$ from the
factorization of $G(c, n)$ using a standard technique, finding the true
power of any prime dividing $G(c,n)$ in the order of $W_c$ by
the fast exponentiation of $W_c$.
\begin{example}
As an example, look at $n=6$ and $c=16$; then $G(c,n)$ is a 411 decimal
digit number with prime factorization
$$2^{319} \cdot 3^{278} \cdot 5^{18} \cdot 7^{10} \cdot 11^9
 \cdot 13^7 \cdot 17^6 \cdot 19^2 \cdot 23^2 \cdot 29^2 \cdot 31^4 \cdot 37 \cdot 41^5 \cdot 43^2 \cdot 61^3 \cdot 67 \cdot 73^3 \cdot 89 \cdot$$
$$113\cdot 127^2 \cdot 151 \cdot 193^2 \cdot 241 \cdot 257^2 \cdot 271 \cdot 331 \cdot 547^2 \cdot 661 \cdot 683 \cdot 757 \cdot 1093^2 \cdot 1181 \cdot$$
$$ 2731 \cdot 3851 \cdot 4561 \cdot 6481 \cdot 8191 \cdot 16493 \cdot 65537 \cdot 398581 \cdot 797161 \cdot 21523361,$$
while the order of $W_{16}$ equals
$$5230176588=2^2 \cdot 3^3 \cdot 7 \cdot 13 \cdot 19 \cdot 37 \cdot 757.$$
In this case it is just doable to compute the numbers that make up
the full period (which turns out to be half the order of $W_{16}$).

$q=6$, $c=18$ niet meer de perioden
\end{example}
