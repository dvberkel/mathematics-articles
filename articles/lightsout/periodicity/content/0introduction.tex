\section*{Introduction}\label{sec:intro}
Lights Out is a handheld electronic puzzle game produced by Tiger Electronics in
the 1990s. It consists of a square grid of 25 buttons that also act as lights.
Each light has two states: on and off.  Pressing a button has the effect of
changing the state of the light of the button itself as well as that of each of
its four possible direct neighbours to the left, right, above, and below.
The object of the game is to turn off all the lights (from some initial
configuration) by pressing a series of buttons.

The main goal of this paper is to analyze the number of
solutions of the game and certain generalizations.
The generalizations concern the size of the board (we consider
rectangular boards of any size $r\times c$), and the number of colours:
besides the off-state we will allow not just one on-state,
but any positive number of different on-states (the different
colours). Pressing a button repeatedly will change the state
of the button (and its neighbours) in a fixed, cyclic order.
We denote the number of states by $n\geq 2$; the case $n=2$
is that of the original game.
The number of solutions we study refers to both the number of
initial states of a board that can be solved, and to the number
of different solutions for such cases.

By previous work it was clear (as we will explain in
Section \ref{sec:algebra}) that the numbers we are looking for
are the numbers of elements of certain linear spaces determined
by a matrix associated with the game. The results in this paper
were inspired by certain observations we made when inspecting
tables like the one shown here as Table \ref{kernels}.
The table encodes part of all the information we search for
in the case of rectangular boards of up to 15 columns and
32 rows with lights that can either be on or off, as in the
original Lights Out game (with 5 columns and 5 rows).

Each entry $d=d(r, c)$ signifies the dimension of a vector
space over $\FF_2$ (the 2 coming from the number of states $n$)
containing a number of vectors ($2^d$) corresponding to
the number of different solutions for any solvable initial board
of size $r\times c$.
Moreover, the number of different states for an $r\times c$ board
that are solvable, equals $2^{r\cdot c-d}$, as we will soon see.
Thus, the number $d(5,5)=2$ in the table indicates that for
the original Lights Out game $2^{5\cdot 5-2}=2^{23}$ (out of the $2^{25}$)
possible initial configurations are solvable,
and for each of these there will be $2^2=4$ different ways
to solve it.
Note that zeroes in the table occur precisely when 
every configuration of lights on the $r\times c$ board can be
turned off, in a unique way.

The main new results in this paper concern the regular
patterns in this table and similar versions for larger $n$,
where the situation is slightly trickier if $n$ is a composite
integer. Our initial observations for Table \ref{kernels}
(extended in both directions) can be summarized as follows.

For $c\geq 0$, consider the column $d_c(r)$ of non-negative
integers $d(r, c)$ for boards with a fixed number of columns $c$
and $r=0,1,2,3,\ldots$ rows.
By symmetry, the $c$-th row coincides
with the $c$-th column, so the properties we list here also
apply to the rows. We find:
%\footnote{To see more entries of the table go to the website for this paper,
%\url{http://dvberkel.github.io/mathematics-articles/lights_out/periodicity.html}}.

\begin{table}
  \begin{center}
    \begin{tabular}{|c|cccccccccccccccc|c|}
      \hline
$r\ \backslash\ c$:&0 &1 &2 &3 &4 &5 &6 &7 &8 &9 &10 &11 &12 &13 &14 &15 &$\ell$\\
      \hline
      \hline
0&0  &0 &0 &0 &0 &0 &0 &0 &0 &0 &0 &0 &0 &0 &0 &0 &1\\
1&0  &0 &1 &0 &0 &1 &0 &0 &1 &0 &0 &1 &0 &0 &1 &0 &3\\
2&0  &1 &0 &2 &0 &1 &0 &2 &0 &1 &0 &2 &0 &1 &0 &2 &4\\
3&0  &0 &2 &0 &0 &3 &0 &0 &2 &0 &0 &3 &0 &0 &2 &0 &6\\
4&0  &0 &0 &0 &4 &0 &0 &0 &0 &4 &0 &0 &0 &0 &4 &0 &5\\
5&0  &1 &1 &3 &0 &2 &0 &4 &1 &1 &0 &4 &0 &1 &1 &4 &24\\
6&0  &0 &0 &0 &0 &0 &0 &0 &6 &0 &0 &0 &0 &0 &0 &0 &9\\
7&0  &0 &2 &0 &0 &4 &0 &0 &2 &0 &0 &7 &0 &0 &2 &0 &12\\
8&0  &1 &0 &2 &0 &1 &6 &2 &0 &1 &0 &2 &0 &7 &0 &2 &28\\
9&0  &0 &1 &0 &4 &1 &0 &0 &1 &8 &0 &1 &0 &0 &5 &0 &30\\
10&0  &0 &0 &0 &0 &0 &0 &0 &0 &0 &0 &0 &0 &0 &0 &0 &31\\
11&0  &1 &2 &3 &0 &4 &0 &7 &2 &1 &0 &6 &0 &1 &2 &8 &48\\
12&0  &0 &0 &0 &0 &0 &0 &0 &0 &0 &0 &0 &0 &0 &0 &0 &63\\
13&0  &0 &1 &0 &0 &1 &0 &0 &7 &0 &0 &1 &0 &0 &1 &0 &18\\
14&0  &1 &0 &2 &4 &1 &0 &2 &0 &5 &0 &2 &0 &1 &4 &2 &340\\
15&0  &0 &2 &0 &0 &4 &0 &0 &2 &0 &0 &8 &0 &0 &2 &0 &24\\
16&0  &0 &0 &0 &0 &0 &0 &0 &0 &0 &0 &0 &0 &0 &8 &0 &255\\
17&0  &1 &1 &3 &0 &2 &6 &4 &1 &1 &0 &4 &0 &13 &1 &4 &168\\
18&0  &0 &0 &0 &0 &0 &0 &0 &0 &0 &0 &0 &0 &0 &0 &0 &513\\
19&0  &0 &2 &0 &4 &3 &0 &0 &2 &8 &0 &3 &0 &0 &6 &0 &60\\
20&0  &1 &0 &2 &0 &1 &0 &2 &6 &1 &0 &2 &0 &1 &0 &2 &2340\\
21&0  &0 &1 &0 &0 &1 &0 &0 &1 &0 &0 &1 &0 &0 &1 &0 &186\\
22&0  &0 &0 &0 &0 &0 &0 &0 &0 &0 &0 &0 &0 &0 &0 &0 &2047\\
23&0  &1 &2 &3 &0 &5 &0 &7 &2 &1 &0 &10 &0 &1 &2 &15 &96\\
24&0  &0 &0 &0 &4 &0 &0 &0 &0 &4 &0 &0 &0 &0 &4 &0 &1025\\
25&0  &0 &1 &0 &0 &1 &0 &0 &1 &0 &0 &1 &0 &0 &1 &0 &126\\
26 &0 &1 &0 &2 &0 &1 &6 &2 &0 &1 &0 &2 &0 &7 &0 &2 &2044\\
27 &0 &0 &2 &0 &0 &3 &0 &0 &8 &0 &0 &3 &0 &0 &2 &0 &36\\
28 &0 &0 &0 &0 &0 &0 &0 &0 &0 &0 &0 &0 &0 &0 &0 &0 &3277\\
29 &0 &1 &1 &3 &4 &2 &0 &4 &1 &9 &0 &4 &0 &1 &5 &4 &2040\\
30 &0 &0 &0 &0 &0 &0 &0 &0 &0 &0 &10 &0 &0 &0 &0 &0 &341\\ 
31 &0 &0 &2 &0 &0 &4 &0 &0 &2 &0 &0 &8 &0 &0 &2 &0 &48\\
32 &0 &1 &0 &2 &0 &1 &0 &2 &0 &1 &0 &2 &0 &1 &0 &2 &4092\\
      \hline
    \end{tabular}
  \end{center}
  \caption{Dimension of kernels, and length of period $\ell$}\label{kernels}
  \label{tab:n2}
\end{table}

\begin{itemize}
\item{\bf Observation 1.}
The sequence $d_c$ is purely periodic. %, for every $c\geq 0$.
\item{\bf Observation 2.}
There exists a number of rows $r$ where $d_c$ is maximal:
$\exists r\geq 1:\ d_c(r)=c$; by $r_0$ we will denote the
smallest such positive $r$.
\item{\bf Observation 3.}
The period length $\ell$ of $d_c$ equals $r_0+1$.
\item{\bf Observation 4.}
The sum of two consecutive kernel dimensions 
is less than or equal to the maximal dimension:
$$\forall r\geq0:\ d_c(r)+d_c(r+1)\leq c.$$
\item{\bf Observation 5.}
The period of $d_c$ is almost palindromic:
$$d_c(0), d_c(1), \ldots, d_c(r_0-1)$$
is a palindrome of length $r_0$, and $d_c(r_0)=c$ completes the period.
\end{itemize}
\noindent
Looking, for example, at
the column of the table with $c=5$, which contains
the standard $5\times 5$ lay out, it begins like this:
$$0,1,1,3,0,2,0,4,1,1,0,4,0,1,1,4,0,2,0,3,1,1,0,5,\ldots$$
and repeats from the beginning after the first `$5$' occurs, which
happens for $r_0=23$. The length of the period, $r_0+1$,
is listed in the final column of the table.
The sequence up until the value 5 is palindromic 
(it reads the same from left to right as from right to left).

The fact that in a column consecutive entries sum to less than the
maximal dimension is Observation 4. It implies
that the penultimate value in the period (the last value of the
palindrome) is always 0, as must be the first.

The variants of Lights Out have a considerable history of
being studied by mathematicians. In \cite{anderson98}, and before that in
\cite{pelletier87}, methods from linear algebra are used to solve Lights
Out systematically.
The first approach is to number the buttons from 1 to 25 and to identify
the state of the board by a row {\it state vector} $s$ of zeroes and ones
(with $s_i=1$ just for those lights that are lit). This state vector can be
interpreted as an element $s\in\FF_2^{25}$.
Any series of buttons to be pressed will also be coded as the (row) 
{\it press vector} $p\in\FF_2^{25}$,
where $p_i=1$ if the $i$-the button is to be pressed, and 0 otherwise.
The effect of pressing button $i$ can then be encoded as a row vector
$a_i\in\FF_2^{25}$, having $1$ precisely at the positions of button $i$
and its four (or fewer) direct neighbours: pressing button $i$ then
results in adding the {\it effect vector} $a_i$ to the state vector. Thus the state
arrived at from the 0 state (with no lights on) using the press
pattern $p$ will be $p\cdot A$, where $A$ is the $25\times 25$ 
symmetric matrix over $\FF_2$ having $a_i$ as its $i$-th row.
Since $v=-v$ for any vector over $\FF_2$, it will be clear
that the press pattern $p$ needed to turn all lights out
from a given initial state vector $s$ can be found as a
solution to the vector-matrix equation $p\cdot A=-s=s$.

Solving this equation by hand is hardly feasible, but the equation does give
insight into the solvability question: it turns out that the rank
of $A$ is 23, so the kernel has dimension 2. Hence, for example,
$2^{23}$ out of the $2^{25}$ possible state configurations are solvable,
and for each of these there will be $2^2=4$ different solutions $p$.

A more practical solution is given in \cite{martin01}, where a technique
known as gathering or \emph{chasing the lights} is introduced.
Since the order in which buttons are pressed is of no importance,
we may choose to do so row by row of the board (starting from the top), from
left to right. To turn the lights in a given configuration $s$ off,
one might start by turning off the lights in the first row by
pressing those buttons in the second row directly below lit lights
of the first row. 

In Figure \ref{chasing}, for example, some lights are initially
lit (indicated by gray).  There is only one way to turn off the
lit lights of the first row by pressing buttons
in the second row,
namely, to press each button directly below
a lit button (the ones in row 2 marked with $\times$).

\begin{figure}
  \mbox{
    \subfigure[Window of first two rows\label{first_row_pressed}]{\epsfig{figure=image/windowing01.ps,width=.40\textwidth}}
    \subfigure[Next window\label{second_row_pressed}]{\epsfig{figure=image/windowing02.ps,width=.40\textwidth}}
  }
  \caption{Chasing the lights}
  \label{chasing}
\end{figure}

Then turn off any remaining lights in the second
row by pressing the proper buttons of the third row, and so on. 
In the end only some buttons in the fifth (or bottom) row will
be on. We have chased the lit buttons in the first row down to the
last row.


To turn off any remaining lights in the last row, we attempt to
create a light pattern $t$ on the empty board by only pressing buttons
in the first row, with the property that chasing that pattern results
in exactly the same light pattern in the last row. Creating and chasing
$t$  after chasing $s$ would then result in duplicating the bottom
row, which means it will also become unlit.

The task for any person who (like the first author) would
like to be able to turn a given light pattern off (when
possible) is to memorize the (small table of) results of chasing the independent
patterns that can be created on the first row, and to combine those
in his or her head to the bottom row pattern obtained by chasing
the initial configuration.

In \cite{leach17} this method chasing the lights is extended to general
rectangular board shapes.  It is shown there, as we will explain further
in Section
\ref{sec:algebra}, that for analyzing an $r \times c$ Lights Out
board one is interested in the upper left $c \times c$ sub-matrix of
$W^{r}$, where $W$ describes the effect of one step in chasing the lights.
As a consequence, both the number of configurations ($2^{rs-d}$)
for which a solution to the Lights Out problem exists and the number
of different solutions once it is solvable ($2^d$), are determined
by the dimension $d$ of a linear subspace of some $\FF_2$-vector space
(the kernel of the submatrix of $W^r$ referred to above); it is this number
that is given in Table \ref{tab:n2}.

In Section \ref{sec:def}  we will describe chasing and the matrices involved
in detail, because this is the context in which we made
the above observations about the `kernel dimensions for
Lights Out on a rectangular board'. As we saw above for the $5\times 5$
board, we are interested in the dimension of 
the null-space (or kernel) of some matrix.

In Section \ref{sec:prime} we will prove the observations we made,
not just for 2 colours, but any {\it prime} number of colours.
In Section \ref{sec:comp} we deal with the complications arising
when the number of colours $n$ is a {\it composite} number.


