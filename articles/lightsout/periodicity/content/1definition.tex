\section{Matrices}\label{sec:def}
%\subsection*{Definitions}
We will be exploring {\it rectangular Lights Out with
$n$ colours} $\Lp(r, c, n)$: the game will consist of an $r\times c$ rectangular
layout (so $r$ rows and $c$ columns) of buttons that can each be
in one of $n$ different states. These states will often be referred to
as {\it colours}, and will be identified with elements
of $\Z/n\Z$. 
The zero-state of a button means that its light is turned off.
Pressing any of the buttons will change the state
of the button itself as well as that of any of its neighbouring buttons
by adding 1 modulo $n$. 
Neighbouring buttons are buttons
immediately next to, above it or below it; so there are at most 4
of them (and fewer on the edges).
The standard Lights Out game described in the Introduction
corresponds to $\Lp(5,5,2)$.

\begin{remark}
There are two other generalization we like to mention that will not be
dicussed further here. In some variants of the game the effect of pressing some
button is different: the states of adjacent buttons may be affected by adding
other values, or the neighbouring relation may be given by a more general
graph than this rectangular one. In another variant of the game one is
only allowed to press buttons that are lit, i.e., that are in a state
different from zero. The latter requirement drastically alters the
game, as the order in which buttons are pressed will be of importance then!
\end{remark}
%
An instance of the $\Lp(r, c, n)$ Lights Out puzzle now consists
of an $r\times c$ matrix $M$ with entries from $\Z/n\Z$, and the goal
is to reach the state where all lights are off, by a sequence of button
presses.

The effect on $M$ of pressing the single button at position $i, j$
(with $1\leq i \leq r$ and $1\leq j \leq c$) is given by
$$\pi_{i,j}: M\mapsto M+P_{i, j},$$
and is called a {\it basic press}.
Here the matrix $P_{i,j}$ has the same size as $M$, and is
defined by
\[
    P_{i,j}{(u,v)} :=
    \left\{
    \begin{array}{l@{\quad:\quad}l}
        1 & \text{if} \operatorname{d}\left((i,j), (u,v)\right) \leq 1 \\
        0 & \text{otherwise}
    \end{array}
    \right.
\]
where $\operatorname{d}$ is the \emph{Manhattan distance} between 
positions $(i,j)$ and $(u,v)$.
Put differently, when pushing the button at position $(i, j)$ the effect will
be to add matrix $P_{i,j}$ to $M$, where $P_{i,j}$ is the $r\times c$ matrix
with $1\in\Z/n\Z$ at position $i, j$ and its immediate neighbours.

A {\it press pattern} could be thought of as a succession of 
basic presses; however, it will be clear that order in which the
basic presses are executed in the press pattern is irelevant, and since 
repeating a particular basic press $n$ times will have no effect,
we may identify such a press pattern also by an element 
$\Pi\in\Mat_{r,c}(\Z/n\Z)$:
matrix element $\Pi_{i,j}\in\Z/n\Z$ simply indicates how often the button
at position $i, j$ will be pressed. The {\it effect} of $\Pi$ on the initial
puzzle $M$ will be:
$$\Pi:M\mapsto M+\sum_{j=1}^r\sum_{i=1}^c \Pi_{i, j}P_{i, j}.$$
We will call $E=E(\Pi)=\sum_{j=1}^r\sum_{i=1}^c \Pi_{i, j}P_{i, j}$ the
{\it effect-matrix} of press pattern $\Pi$, and thus
$\Pi(M)=M+E$.

Note that we now have three different interpretations of elements
of $\Mat_{r,c}(\Z/n\Z)$: a light pattern $L$ gives the states (colours)
of the buttons in the display; a press pattern $\Pi$ indicates which buttons
will be pressed (and how often), and an effect matrix $E$ records
the change to $L$ if applying $\Pi$:
$\Pi(L)=L+E$.

The composition of two press patterns 
corresponds to the sum of the matrices $\Pi+\Pi'$, and this makes 
the set of all press patterns $\cal P$ into a monoid,
in which the identity element corresponds to the zero matrix.
It is also clear from the definition that $E(\Pi+\Pi')=E(\Pi)+E(\Pi')$.
\subsection*{Chasing}
We will now more formally describe the operation of {\it chasing}
a light pattern $L$ on any rectangular board, with any number
of colours. It consists of two steps.

In the first step a press pattern $\Pi$ is found,
with corresponding effect matrix $E$, such that applying
$\Pi$ to $L$ results in a light pattern $\chase(L)$ that has the property
that $\chase(L)_{i,j}=0$ for entries with $i<r$; in other words, 
for matrix $L+E$ only entries on the bottom row can be non-zero.
The matrix $\Pi$ is constructed 
row-by-row: the first row will be zero. The second row of the press pattern
$\Pi$ is chosen in such a way that the lights on the first row of $L$
will all be turned off; that is, $\Pi_{2,j}=n-L_{1,j}$ for $1\leq j\leq c$,
meaning that we press the button below any button in the first row exactly so many
times that the button in the first row will be turned off. The first row
of the effect matrix $E$ will thus precisely be $(n-L_{1,1}, n-L_{1,2},\ldots,
n-L_{1,c})$, as required. Now the complications start, as the second row
of the press pattern will also affect the second and third rows of $E$.
The third row will become a copy of the first row, but to compute the
second row of $E$ we need to determine the effect of any row of button presses
$(p_1, p_2, \ldots, p_c)$. Pressing the first button $p_1$ times adds $p_1$
to both the first and the second light in that row (hence to the row of $E$),
pressing the second button $p_2$ times adds $p_2$ to the first, second and third
light, and so on. In summary, the effect on a row of applying the press pattern
$(p_1, p_2, \ldots, p_c)$ to that row
is given by the matrix multiplication
$$(p_1, p_2, \ldots, p_c)\cdot
\begin{pmatrix}
1&1&0&0&\cdots&0&0&0&0\\
1&1&1&0&\cdots&0&0&0&0\\
0&1&1&1&\cdots&0&0&0&0\\
\vdots&\vdots&\vdots&\vdots&\cdots&\vdots&\vdots&\vdots\\
0&0&0&0&\cdots&1&1&1&0\\
0&0&0&0&\cdots&0&1&1&1\\
0&0&0&0&\cdots&0&0&1&1
\end{pmatrix}$$
where we will call the $c\times c$ matrix on the right $E_c$: it
has $1$s only on, directly above, and directly below the main diagonal.
We use matrix multiplication on the right, because vectors will then be
written on the left as row vectors, with obvious typographical advantages.
Once we have changed the second and third rows of $E$, we replace $L$
by $L+E$, a new light pattern where all lights in the first row are
turned off. Next we determine the third row of $\Pi$, in such a way that
that also the lights in the second row of the new light pattern will
be turned off, adapt the effect matrix $E$ accordingly (rows 2, 3, and 4
will be modified) and change the light pattern also. This is
repeated, until by changing the bottom row of $\Pi$ also the penultimate
row of lights in $L$ is turned off completely. The resulting
light pattern $\chase(L)$ is thus uniquely determined, and it is
completely described by the bottom row $b=(b_1, b_2, \cdots, b_c)$, as
all other entries are zero.

For the second step of solving a given Lights Out problem
$L$, a press pattern solution $S$ to the matrix
equation $S\cdot E_c=-\chase(L)$ is needed:
the effect of first applying the press pattern
$\Pi$ found above and then the press pattern $S$ then turns all the lights off.
Two questions clearly remain: does such solution $S$ exist, and how
to find it in case the answer is affirmative?

From a theoretical point of view
both questions are answered simultaneously by observing that
any light pattern for which only bottom row lights are turned on
can be obtained from a board where all lights are initially turned off
by just chasing a light pattern created from some
press pattern involving only buttons on the first row.
Since all operations involved are linear, the only bottom row patterns
that can be obtained are the $\Z/n\Z$-linear combinations of the
bottom rows $a^{(1)}, a^{(2)}, \ldots, a^{(c)}$, where $a^{(i)}$ is the result of
chasing the light pattern that is the effect of only pressing the $i$-th
button on the first row once. A solution $S$ only exists when
when $-\chase(L)=-(b_1, b_2, \ldots, b_c)$ is a linear combination
of the vectors $a^{(i)}$:
$$-(b_1, b_2, \ldots, b_c)=\lambda_1a^{(1)}+\lambda_2a^{(2)}+\cdots+\lambda_ca^{(c)}.$$
For the practical puzzler, the solution can be found by memorizing how the $\lambda_{i}$ depend on the $b_{i}$,
a calculation that can be performed from the top of one's head, and then the solution is found
by chasing the result of applying the press pattern 
$(\lambda_1, \lambda_2, \ldots, \lambda_c)$ to the first row.

For example, the table below can be used to determine the $\lambda_{i}$ for the standard Lights Out. Since the kernel is of dimension 2, one only has to look at the first three buttons

\begin{table}
  \begin{center}
    \begin{tabular}{|c|c|}
      \hline
      \hline
$\lambda_1$ & $b_1 + b_2$\\
$\lambda_2$ & $b_1 + b_2 + b_3$\\
$\lambda_3$ & $b_2 + b_3$\\
        \hline
    \end{tabular}
  \end{center}
  \caption{A solution determined from the chased light pattern in standard lights out.}
  \label{tab:solution}
\end{table}

On the other hand, this also implies that the only press patterns
that do not change the state of the board, are those that combine
a press pattern on the first row by a press pattern on all subsequent
rows that correspond to chasing the lights to a bottom row without any
lights on.

