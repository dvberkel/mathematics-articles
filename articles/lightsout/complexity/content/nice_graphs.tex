\section{Nice Graphs}
We call a graph $G:=(V, E)$ \emph{nice} if and only if it has the following properties

\begin{enumerate}
    \item \emph{reflexive} there is an self-edge for each vertex of $G$
    \item \emph{symmetric} if $(u,v)\in E$ then also $(v,u)\in E$
    \item \emph{equal weight} Each edge has the same weight.
\end{enumerate}

Without the loss of generality, for nice graphs we can set the weights to 1.

\begin{proof}[TODO talk about equivalent problems]
    There is an \emph{equivalent} problem with weights 1 for each nice graph.
\end{proof}

\begin{theorem}
    For a nice graph $G$ we have

    a solution for the restricted Lights Out problem on G exists if and only if a solution to the standard Lights Out problem on G exists.
\end{theorem}

\begin{proof}[TODO expand proof]
    from restricted to standard trivial.

    From trivial to restricted. Each standard solution can be lifted to restricted. Induction on solution length and the observation that there is a lit button nearby. 
\end{proof}

\begin{corollary}
    The complexity for the existance and solution restricted Lights Out problem on nice graphs is the same for standard Lights Out problem on such graph.
\end{corollary}