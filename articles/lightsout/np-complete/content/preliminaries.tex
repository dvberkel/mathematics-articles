\section{Preliminaries}\label{section:preliminaries}

In this section we will make all the necessary layouts to facilitate the
proof of the fact that Lit Generalised Lights Out is NP-complete.

Our layouts will be over $\Z/3\Z$. This is not a restriction. The proof
(with slight alterations) is valid for any $\Z/q\Z$. See section
\ref{section:general} for the general case.

We will define layouts of a general type. A \emph{construct} is a layout
with certain vertices labelled as \emph{inputs} and other vertices
labelled as \emph{outputs}.

Complex constructs can be build out of simpler constructs by
identifying an output of a construct with an input of an other
construct.

The states of inputs and outputs will be interpeted in the following
way. If an input(/output) is in state $1$ it is said to be $\f$ and if
an input(/output) is in state $2$ it is said to be $\t$.

We also define a function $\mu : \{\f,\t\} \rightarrow \Z/3\Z$ with
$\mu(\f) = 1$ and $\mu(\t) = 2$.

\subsection{Kill-switch Layout}

An  $n$-kill-switch is a layout on $n$ vertices in a predetermined
state. Every vertex is connected to every vertex with multiplicity $2$.
Every vertex is in state $1$. (For an example of an $2$-kill-switch see
figure \ref{figure:killswitch}.) 
\begin{figure}
	\begin{center}
		\includegraphics{image/definitions.1}
	\end{center}
	\caption{A $2$-kill-switch}\label{figure:killswitch}
\end{figure}

In an $n$-kill-switch every button can be pressed. Furthermore every
button has the same effect i.e. after pressing a button every button is
in the off state.
 
We can use a $n$-kill-switch to distinguish between a choice of
$n$-options. How this can be used is shown in the following subsections.

\subsection{Assigners, Splitters and Terminators.}

In order to simulate a logical formula we must have the ability to
assign logical values to variables, split them to facilitate the various
logical operators and transmit these values to these logical operators.

We will now present various constructs which do these tasks. At the
center of all these layouts is the \emph{$2$-kill-switch}.

\subsubsection{Assigners}

We will first define an \emph{assigner}. Its purpose is to assign a (truth) 
value to a variable.

An assigner is a construct on the set $\{k_{0},k_{1},o\}$. Vertices $k_{0}$
and $k_{1}$ will form a $2$-kill-switch. Vertex $o$ will serve as
output. There are no inputs. Depending on which of $k_{0}$ and $k_{1}$ is 
pressed the output  will be in the $\t$ state or the $\f$ state.

Below is the adjacency matrix of an assigner.
\[
	\left(
	\begin{array}{ccc}
		2 & 2 & 0 \\
		2 & 2 & 0 \\
		1 & 2 & 0 \\
	\end{array}
	\right)
\]

for an example of an assigner see figure \ref{figure:constructsA}

\subsubsection{Splitters}

Next we will define an \emph{$n$-splitter}. Its purpose is to split a signal
into $n$ channels.

An $n$-splitter is a construct on the set 
$\{k_{0},k_{1},i,o_{0},\ldots,o_{n-1}\}$. The vertices
$k_{0}$ and $k_{1}$ will by used to create a $2$-kill-switch. Vertex $i$
serves as input and vertex $o_{0}$ through vertex $o_{n-1}$ are the
outputs.

The adjacency matrix of a $n$-splitter is given below.
\[
	\left(
	\begin{array}{ccccc}
		2 & 2 & 0 & \cdots & 0 \\
		2 & 2 & 0 & \cdots & 0 \\
		2 & 1 & 0 & \cdots & 0 \\
		1 & 2 & 0 & \cdots & 0 \\
		\vdots & \vdots & \vdots & \ddots & \vdots \\
		1 & 2 & 0 & \cdots & 0 \\
	\end{array}
	\right)
\]

By pressing one of the kill-switch vertices the state of $i$ is transferred to
all of the outputs. For example, assume $i$ is in the $\t$ (2) state. By
pressing $k_{1}$, $i$, $k_{0}$ and $k_{2}$ will be in the off state. Furthermore
all $o_{0}$ through $o_{n}$ wil now be in $\t$ state.

See figure \ref{figure:constructsA} for an example of a $n$-splitter.
\begin{figure}
	\begin{center}
		\subfigure[assigner]{\includegraphics{image/definitions.2}}
		\subfigure[$n$-splitter]{\includegraphics{image/definitions.3}}
	\end{center}
	\caption{Assigners and splitters}\label{figure:constructsA}
\end{figure}

A $1$-splitter will be called a \emph{wire}. It is also possible to 
interchanging the multiplicities of the edges to an output. This has the effect
of inverting the input signal. A wire which has inverts the input is called
\emph{inverter}.
\begin{figure}
	\begin{center}
		\subfigure[wire]{\includegraphics{image/definitions.5}}
		\subfigure[inverter]{\includegraphics{image/definitions.6}}
	\end{center}
	\caption{Wires and inverters}\label{figure:constructsB}
\end{figure}

\subsubsection{Terminator}

Now we will define a terminator. It plays a keystone role in the argument. Its
purpose is to determine if a certain truth value is attained.

The construct is very simple. It constist only of an input which is in state 
$\f$ (1). The only way to switch the input into the off state is by connecting
is with a output which will produce the $\t$ state.

\subsection{Logical Connectives}

A standard way to continue is to show that it is possible to construct
every logical connective by exhibiting a universal logical connective
like \emph{nand} or \emph{nor}.

We will take another approach and show an universal construction for any
logical connective, surpassing the problem of expressing a given
connective as a product of a universal one.

For that programme let $\oplus$ be any logical connective. In table
\ref{table:oplus} a general definition is given.
\begin{table}
	\begin{center}
		\begin{tabular}{c|c|c}
			$A$ & $B$ & $A \oplus B$ \\
			\hline
			\f & \f & $r_{0}$ \\
			\f & \t & $r_{1}$ \\
			\t & \f & $r_{2}$ \\
			\t & \t & $r_{3}$ \\
		\end{tabular}
	\end{center}
	\caption{A general logical connective $\oplus$.}\label{table:oplus} 
\end{table}

We will create a construct which acts as the logical connective
$\oplus$. The construct uses the vertices 
$\{k_{0}, k_{1}, k_{2}, k_{3}, i_{1}, i_{2}, o \}$. $k_{0}$ through $k_{3}$ will
form an $4$-kill-switch. Below is the adjacency matrix of the construct. 
\[
	\left(
	\begin{array}{ccccccc}
		2 & 2 & 2 & 2 & 0 & 0 & 0 \\
		2 & 2 & 2 & 2 & 0 & 0 & 0 \\
		2 & 2 & 2 & 2 & 0 & 0 & 0 \\
		2 & 2 & 2 & 2 & 0 & 0 & 0 \\
		1 & 1 & 2 & 2 & 0 & 0 & 0 \\
		1 & 2 & 1 & 2 & 0 & 0 & 0 \\
		\mu(r_{0}) & \mu(r_{1}) & \mu(r_{2}) & \mu(r_{3}) & 0 & 0 & 0 \\
	\end{array}
	\right)
\]

Notice that every kill-switch vertex is connected to both of the inputs and the
output. The only difference are the multiplicities of the edges. They are chosen
to reflect the logical operator.

For example, assume that $i_{0}$ is in the $\t$ (2) state and that $i_{1}$ is in
the $\f$ (1) state. Pressing $k_{2}$ will switch off both of the inputs and all 
of the kill-switch vertices. Furthermore the output will be in the $\mu(r_{2})$,
just as the logical operater dictates. 

For an example of a logical connector see figure
\ref{figure:connective}. 
\begin{figure}
	\begin{center}
		\includegraphics{image/definitions.4}
	\end{center}
	\caption{A layout of a general logical connective.($4$-kill-switch is not shown.)}\label{figure:connective}
\end{figure}
