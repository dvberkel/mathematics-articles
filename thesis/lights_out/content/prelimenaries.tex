\chapter{Preliminaries}

This chapter gathers results needed in earlier chapters. It is assumed the
reader has at least seen the definition of a ring.

\section{Graphs}

In general there is no consensus among mathematicians what a graph is. Everybody
adopts her own definition. Because we are interested in a particular kind of
graph we will choose are definitions apropriatly.

\begin{definition}
	A \emph{directed multigraph} $\Gamma:=(V,E,i,t)$ is a four-tuple where
	$V$ and $E$ are disjoint sets, $i$ and $t$ are mappingss from $E$ to
	$V$.
	
	Elements of $V$ are called \emph{vertices} and elements of $E$ are
	called \emph{directed edges}. An edge $e\in E$ has a \emph{initial vertex} $i(e)$
	and \emph{terminal vertex} $t(e)$. $e$ joins $i(e)$ and $t(e)$ and is
	directed from $i(e)$ and to $t(e)$.
\end{definition}

\begin{remark}		
	It is possible that there are multiple edges between the same vertices. 
	
	It is also possible that an edge has an initial	vertex which is the same
	as the terminal vertex. Such edges are called \emph{loops}.
\end{remark}

We would like to shorten our notations where ever possible. The following remark
makes a step in that direction.

\begin{remark}
	We will use the following notation for a directed multigraph $\Gamma :=
	(V,E,i,t)$. $v\in\Gamma$ will mean that $v\in V$, $e\in E(\Gamma)$ will
	mean $e\in E$. $i_{\Gamma} := i$ and $t_{\Gamma} := t$ or if the context
	permits it we will drop the subscript altogether. 
\end{remark}

A general notion in graphs is that of adjacency. We will define these concepts
in the following definition.

\begin{definition}
	For every directed multigraph $\Gamma$ and vertices $u,v\in\Gamma$ we
	say that $u$ is adjacent to $v$ if there exist an edge $e\in E(\Gamma)$
	such that $i(e)=v$ and $t(e)=u$.
	
	The neighbourhood of $v\in\Gamma$ is the set $N(v):=\{ u\in\Gamma \bar u
	\text{ is adjacent to } v\}$.
	
	We define the multiplicity $m_{u,v}$ as the number of edges $e\in
	E(\Gamma)$ with $i(e)=u$ and $t(e)=v$.
	
	If we have chosen an ordering of the vertices of $\Gamma$ we define the
	adjacency matrix $A_{\Gamma}$ to be the matrix with $m_{u,v}$ on row $v$
	and column $u$.
\end{definition}

\begin{example}\label{example:multigraph}
	Let $\Gamma := (\{u,v,w\},\{1,2,3,4,5,6,7\},i,t)$ where $i$ and $t$ are
	defined by tables \ref{table:initial} and \ref{table:terminal}
	respectively.
	\begin{table}
		\begin{center}
			\begin{tabular}{|c|ccccccc|}
				\hline
				$e$ & 1 & 2 & 3 & 4 & 5 & 6 & 7 \\
				\hline
				$i(e)$ & $u$ & $u$ & $v$ & $u$ & $w$ & $v$ & $v$ \\
				\hline
			\end{tabular}
		\end{center}
		\caption{Definition of $i_{\Gamma}$ of example
		\ref{example:multigraph}.}\label{table:initial}
	\end{table}
	\begin{table}
		\begin{center}
			\begin{tabular}{|c|ccccccc|}
				\hline
				$e$ & 1 & 2 & 3 & 4 & 5 & 6 & 7 \\
				\hline
				$t(e)$ & $v$ & $v$ & $u$ & $w$ & $u$ & $w$ & $v$ \\
				\hline
			\end{tabular}
		\end{center}
		\caption{Definition of $t_{\Gamma}$ of example
		\ref{example:multigraph}.}\label{table:terminal}
	\end{table}
	(See also figure \ref{figure:multigraph})
	\begin{figure}
		\begin{center}
			\includegraphics{image/graph.6}
		\end{center}
		\caption{The graph $\Gamma$ of example
		\ref{example:multigraph}}\label{figure:multigraph} 
	\end{figure}
	
	In $\Gamma$ $u$ is adjacent to $v$ and $w$, $v$ is adjacent to $u$ and
	itself and $w$ is adjacent to $u$ and $v$. This information determines
	the neighbourhoods of all the vertices.
	
	The following matrix is the adjacency matrix $A_{\Gamma}$.
	\[
		\left(
		\begin{array}{ccc}
			0 & 1 & 1 \\
			2 & 1 & 0 \\
			1 & 1 & 0 \\
		\end{array}
		\right)
	\]
	
	Edge $7$ is a loop.
\end{example}

Example \ref{example:multigraph} is an example where a picture can be made.
These pictures can be unwieldy if many edges have the same endpoints. In order
to adress this consider the following convention.

\begin{remark}
	If in a picture of a directed graph an edge between vertices $u$ and $v$
	is seen without an arrow, it must be interpreted as two directed edges.
	One from $u$ to $v$ and another from $v$ to $u$.
	
	If an edge from $u$ to $v$ is labelled with a natural number $n$ it
	tells us that the multiplicity $m_{u,v} = n$. 
\end{remark}

There are various operations we preform on graphs. They will have our interest
in following chapters so we shall define them here.

\begin{definition}
	Let $\Gamma$ be a directed multigraph. We will define an
	\emph{associate} $\Gamma'$ of $\Gamma$ as follows. $\Gamma'$ is a
	directed multigraph on the vertices and edges of $\Gamma$ with for all
	$e\in E_{\Gamma'}$
	\[
		i_{\Gamma'}(e) = t_{\Gamma}(e)
		\Leftrightarrow
		t_{\Gamma'}(e) = i_{\Gamma}(e)
	\]
	
	This corresponds to reversing all directions of the edges.
\end{definition}

It is thinkable that the operation just described executed by different
operators produce different results. That this is not the case is stated in the
following lemma.

\begin{lemma}
	Every directed multigraph $\Gamma$ has a unique associate.
\end{lemma}

\begin{proof}
	Let $\Gamma_{1}$ and $\Gamma_{2}$ be associates of $\Gamma$. Then for
	all $e\in\Gamma$: $i_{\Gamma_{1}}(e) = t_{\Gamma}(e) =
	i_{\Gamma_{2}}(e)$ and $t_{\Gamma_{1}}(e) = i_{\Gamma}(e) =
	t_{\Gamma_{2}}(e)$.
	
	This shows that $\Gamma_{1} = \Gamma_{2}$.
\end{proof}

The above lemma justifies the following definition.

\begin{definition}
	We will denote the associate of $\Gamma$ by $\Gamma^{t}$.
\end{definition}

\begin{example}\label{example:associate}
	Let $\Gamma$ be the directed multigraph on $\{u,v,w,x,y\}$ defined by
	the following adjacency matrix (See also figure \ref{figure:associate}.)
	\[
		\left(
		\begin{array}{ccccc}
			0 & 0 & 0 & 0 & 0 \\
			1 & 0 & 0 & 0 & 0 \\
			0 & 1 & 1 & 0 & 0 \\
			0 & 1 & 0 & 0 & 0 \\
			0 & 1 & 0 & 0 & 0 \\
		\end{array}
		\right)
	\]
	
	$\Gamma^{t}$ is obtained from the graph of $\Gamma$ by reversing the
	direction of every edge.
	\begin{figure}
		\begin{center}
			\hfill
			\subfigure[$\Gamma$]{\includegraphics{image/graph.4}}
			\hfill
			\subfigure[$\Gamma^{t}$]{\includegraphics{image/graph.5}}
			\hspace*{\fill}
		\end{center}
		\caption{The graphs $\Gamma$ and $\Gamma^{t}$ of example
		\ref{example:associate}}\label{figure:associate}
	\end{figure}
	The adjacency matrix of $\Gamma^{t}$ is the transpose of the adjacency
	matrix of $\Gamma$.
\end{example}

\section{Modules}

A module is an extension of the concept of vectorspace.

\begin{definition}
	Let $R$ be a ring. An $R$-module $M$ is an abelian group $(M,+)$ with an
	operation $\cdot$ from $R\times M$ to $M$ such that for all $m,n\in M$
	and $r,s\in R$ 
	\begin{definitionlist}[M]
		\item $r\cdot(m+n) = r\cdot m + r\cdot n$. \label{module:grouphomomorphism}
		\item $(r+s)\cdot m = r\cdot m + s\cdot m$. \label{module:ringhomomorphism1}
		\item $r\cdot(s\cdot m) = (rs)\cdot m$. \label{module:ringhomomorphism2}
		\item $1\cdot m = m$. \label{module:ringhomomorphism3}
	\end{definitionlist}
\end{definition}

\begin{remark}
	We will often suppress the use of $\cdot$.
\end{remark}

\begin{example}\label{example:ringmodule}
	Let $R$ be a ring. Then $R$ is also an $R$-module.
\end{example}

\begin{example}
	Let $F$ be a field and $V$ an $F$-vector space. Then $V$ is also an
	$F$-module.
\end{example} 

The single most important idea in vector spaces is that of a base. We will
define a similiar notion for modules.

\begin{definition}
	A free $R$-module $M$ is a $R$-module with a finite subset
	$\{b_{1},\ldots,b_{n}\}\subset M$ such that 
	\begin{definitionlist}[B]
		\item\label{basis:generating} For every $m\in M$ there exist 
		$r_{1},\ldots,r_{n}\in R$ such that $m=\sum_{i=1}^{n}
		r_{n}b_{n}$.
		
		\item\label{basis:independent} If $\sum_{i=1}^{n} r_{n}b_{n} =
		0$ with $b_{i}\not=b_{j}$ if $i\not=j$ then
		$r_{1}=\ldots=r_{n}=0$. 
	\end{definitionlist}
	A set with property \ref{basis:generating} is called a generating set. A
	set with property \ref{basis:independent} is called a linearly
	independent set. So a free $R$-module $M$ is a module with linearly
	independent, generating set. 	
\end{definition}

\subsection{Submodules}

\begin{definition}
	Let $M$ be an $R$-module. A subset $N$ of $M$ is a $R$-submodule of $M$
	if $N$ is a module with the operations of $M$ restricted to $N$. We will
	denote this by $N<M$.
\end{definition}

In order to recognize submodules the following test is appropriate.

\begin{proposition}[Submodule Test]
	Let $M$ be an $R$-module. The following are equivalent.
	\begin{enumerate}
		\item\label{submodule:equivalence:A} $N$ is an $R$-submodule of
		$M$. 
		
		\item\label{submodule:equivalence:B} For all $x,y\in N$ and for
		all $r\in R$: $x+y\in N$, $-x\in N$ and $rx\in N$.
		
		\item\label{submodule:equivalence:C} For all $x,y\in N$ and for
		all $r\in R$: $x-y\in N$ and $rx\in N$.
	\end{enumerate}	
\end{proposition}

\begin{proof}
	We will show the following chain of implications
	\ref{submodule:equivalence:A} $\Rightarrow$
	\ref{submodule:equivalence:B} $\Rightarrow$
	\ref{submodule:equivalence:C} $\Rightarrow$
	\ref{submodule:equivalence:A}.
	
	\begin{namedlist}[\ref{submodule:equivalence:A} $\Rightarrow$ \ref{submodule:equivalence:B}]
		\item[\ref{submodule:equivalence:A} $\Rightarrow$ \ref{submodule:equivalence:B}]
		If $N$ is a $R$-submodule of $M$ then in particular $N$ is an
		abelian group and for all $r\in R$ for all $x\in N$: $rx\in N$.
				
		\item[\ref{submodule:equivalence:B} $\Rightarrow$ \ref{submodule:equivalence:C}]
		Notice that for all $x,y\in N$: $x-y = x + (-y) \in N$
		
		\item[\ref{submodule:equivalence:C} $\Rightarrow$ \ref{submodule:equivalence:A}]
		By the subgroup test $N$ is a subgroup of $(M,+)$. Because $N$
		is also closed under scalar multiplication and scalar
		multiplication is well behaved, $N$ is an $R$-submodule.
	\end{namedlist}
\end{proof}

\begin{example}\label{example:submodule}
	By example \ref{example:ringmodule} $\Z$ is a $\Z$-module. Let
	$q\in\N$. We will prove that $q\Z := \{qz \bar z\in \Z\}$ is a
	$\Z$-submodule of $\Z$. 
	
	Let $qx,qy\in q\Z$ then $qx-qy = q(x-y) \in q\Z$ because $x-y\in\Z$.
	Furthermore for every $r\in\Z$ $r(qx) = q(rx) \in q\Z$ because $rx \in
	\Z$.
	
	By the submodule test $q\Z$ is a $\Z$-submodule of $\Z$.
\end{example}

\subsection{Quotient Module}

Taking quotients is a very strong tool in mathematics. We develop the theory for
modules here.

\begin{definition}
	Let $M$ be an $R$-module and $N$ a $R$-submodule of $M$. We define a
	relation $\sim_{N}$ on $M$. For $x,y\in M$
	\[
		x \sim_{N} y \Leftrightarrow x-y \in N
	\]
\end{definition}

\begin{proposition}
	$\sim_{N}$ is an equivalence relation for all $R$-modules $M$ and
	$R$-submodules $N$.
\end{proposition}

\begin{proof}
	We will show that $\sim_{N}$ is reflexive, symmetric and transitive.
	\begin{namedlist}[Transitivity]
		\item[Reflexivity] For all $x\in M$ $x-x = 0 \in N$. This shows
		that $\sim_{N}$ is reflexive.
		
		\item[Symmetry] If $x\sim_{N} y$ for $x,y\in M$ then $x-y\in N$
		but then also $-(x-y) = y-x \in N$ thus $y\sim_{N} x$ which shows
		that $\sim_{N}$ is symmetric.
		
		\item[Transitivity] If $x\sim_{N} y$ and $y\sim_{N} z$ for
		$x,y,z\in M$ then $x-y\in N$ and $y-z\in N$. But then also
		$(x-y) + (y-z) = x-z \in N$ so $x\sim_{N} z$. Hence $\sim_{N}$
		is transitive.
	\end{namedlist}
	
	This shows that $\sim_{N}$ is an equivalence relation.
\end{proof}

Naming the equivalence classes is appropriate now.

\begin{definition}
	The set of equivalence classes with respect to $\sim_{N}$ will be
	denoted by $M/N$ for all $R$-modules $M$ and submodules $N$.
	
	An element in $M/N$ will be denoted by $x+N$ for an $x\in M$.
\end{definition}

A great aspect of taking quotients is that you will end up with a new module. We
lay the foundations of this in the following definition, lemma and proposition.

\begin{definition}
	We define operations $\oplus$ from $M/N \times M/N$ to $M/N$ and $\odot$
	from $R\times M/N$ to $M/N$. For all $x+N,y+N\in M/N$ and $r\in R$
	\begin{definitionlist}[Q]
		\item $(x+N) \oplus (y+N) = (x+y) + N$.
		
		\item $r\odot(x+N) = rx + N$.
	\end{definitionlist}
\end{definition}

\begin{lemma}
	Operations $\oplus$ and $\odot$ on $M/N$ are well defined.
\end{lemma}

\begin{proof}
	Let $x+N = y+N$ and $s+N = t+N$ then $x-y\in N$ and $s-t\in N$ and so
	$(x-y) + (s-t) = (x+s) - (y+t) \in N$ hence $(x+N) \oplus (s+N) = (y+N)
	\oplus (t+N)$.
	
	Furthermore $r(x-y)=rx - ry\in N$ so $r\odot(x+N) = r\odot(y+N)$.
\end{proof}

\begin{proposition}
	$M/N$ with $\oplus$ and $\odot$ make an $R$-module for all $R$-modules
	$M$ and $R$-submodules $N$.
\end{proposition}

\begin{proof}
	Notice that because $(M,+)$ is an abelian group $(N,+) \triangleleft
	(M,+)$ so $(M/N,\oplus)$ is an abelian group.
	
	Furthermore for all $x+N,y+N\in M/N$ and $r\in R$
	\begin{namedlist}[\ref{module:ringhomomorphism3}]
		\item[\ref{module:grouphomomorphism}] $r\odot((x+N) \oplus
		(y+N)) = r\odot((x+y)+N) = r(x+y)+N = (rx+ry)+N = (rx+N) \oplus
		(ry+N) = r\odot(x+N)\oplus r\odot(y+N)$.
		
		\item[\ref{module:ringhomomorphism1}] $(r+s)\odot(x+N) =
		((r+s)x)+N = (rx + sx)+N = ((rx)+N)\oplus((sx)+N) = r\odot(x+N)
		\oplus s\odot(x+N)$.

		\item[\ref{module:ringhomomorphism2}] $r\odot(s\odot (x+N)) =
		r\odot((sx)+N) = r(sx)+N = ((rs)x)+N = (rs)\odot(x+N)$.
		
		\item[\ref{module:ringhomomorphism3}] $1\odot(x+N) = (1x)+N =
		x+N$. 
	\end{namedlist}
	
	This shows that $M/N$ is an $R$-module.
\end{proof}

\begin{remark}
	We will often write $+$ for $\oplus$ and suppress $\odot$ in calculations
	in $M/N$.
\end{remark}

\begin{example}
	By example \ref{example:submodule} we know that $q\Z$ is a submodule of
	$\Z$ for every $q\in\N$. So we can form the quotient module $\Z/q\Z$. As
	per the construction this is a $\Z$ module. 
	
	With a sleightly different view we can see $\Z/q\Z$ as a $\Z/q\Z$
	module, because $\Z/q\Z$ is also a ring.
\end{example}

\subsection{Homomorphisms}

\begin{definition}
	Let $M,N$ be $R$-modules and $\phi:M\rightarrow N$ be a mapping. $\phi$
	is called a homomorphism if for all $x,y\in M$ and $r\in R$: $\phi(x+y) =
	\phi(x) + \phi(y)$ and $\phi(rx) = r\phi(x)$.
	
	An endomorphism is a homomorphism from a $R$-module to itself. An
	isomorphism is a bijective homomorphism. An automorphism is a bijective
	endomorphism.
\end{definition}

\begin{example}\label{example:homomorphism1}
	Define $\phi : \Z \rightarrow \Z/q\Z : x \mapsto (x \mod q)$. We will
	denote the residue class of $x$ by $\overline{x}$. (So $x = \overline{x}
	\mod q$.) We will show that $\phi$ is an homomorphism.
	
	$\phi$ is an homomorphism because for all $x,y\in\Z$ $\phi(x+y) =
	\overline{x+y} = \overline{x} + \overline{y} = \phi(x) + \phi(y)$.
	Furthermore for every $r\in\Z$ $\phi(rx) = \overline{rx} =
	\overline{r}\overline{x} = r\phi(x)$ because $\overline{r}\overline{x} -
	r\overline{x} = (\overline{r}-r)\overline{x} = 0\overline{x} = 0 \mod
	q$.
	
	This shows that $\phi$ is a homomorphism.
\end{example}

\begin{definition}
	Let $\phi$ be a homomorphism between $R$-modules $M$ and $N$. The kernel
	of $\phi$ is defined as $\Ker(\phi) := \{m\in M \bar \phi(m)= 0\}$. The
	image of $\phi$ is defined as $\Im(\phi) := \{\phi(m) \bar m\in M\}$.
\end{definition}

\begin{lemma}
	Let $\phi$ be a homomorphism between $R$-modules $M$ and $N$. Then
	$\Ker(\phi)$ is a submodule of $M$ and $\Im(\phi)$ is a submodule of
	$N$.
\end{lemma}

\begin{proof}
	Let $x,y\in\Ker(\phi)$ and $r\in R$ then $\phi(x-y) = \phi(x) - \phi(y)
	= 0 - 0 = 0$ and $\phi(rx) = r\phi(x) = r0 = 0$. So $\Ker(\phi)$ is a
	submodule of $M$.
	
	Let $s,t\in\Im(\phi)$ and $x,y\in M$ such that $\phi(x)=s$ and
	$\phi(y)=t$. Then $\phi(x-y) = \phi(x) - \phi(y) = s-t$ so
	$s-t\in\Im(\phi)$. Also $\phi(rx) = r\phi(x) = rs$ so $rx\in\Im(\phi)$
	for all $r\in R$. Hence $\Im(\phi)$ is a submodule of $N$.
\end{proof}

\begin{example}\label{example:homomorphism2}
	We will determine the kernel and image of the homomorphism of $\phi$
	from example \ref{example:homomorphism1}.
	
	$\Ker(\phi) = \{x \bar \phi(x) = \overline{x} = 0 \mod q\} = q\Z$.
	
	$\Im(\phi) = \Z/q\Z$.
\end{example}

\begin{proposition}[Isomorphism Theorem]
	Let $M,N$ be $R$-modules and $\phi$ a homomorphism from $M$ to $N$. Then
	$M/\Ker(\phi)$ is isomorphic to	$\Im(\phi)$.
\end{proposition}

\begin{proof}
	Define $\overline{\phi} : M/\Ker(\phi) \rightarrow \Im(\phi) :
	x+\Ker(\phi) \mapsto \phi(x)$. We will show that $\overline{\phi}$ is
	well defined and that $\overline{\phi}$ is an isomorphism.
	
	Let $x+\Ker(\phi)=y+\Ker(\phi)$ then $x-y\in\Ker(\phi)$ so $0 =
	\phi(x-y) = \phi(x) - \phi(y)$ hence $\phi(x)=\phi(y)$. This shows that
	$\overline{\phi}$ is well defined.
	
	Let $x+\Ker(\phi),y+\Ker(\phi)\in M/\Ker(\phi)$ and $r\in R$. Then
	$\overline{\phi}( (x+\Ker(\phi)) + (y+\Ker(\phi)) ) =
	\overline{\phi}((x+y)+\Ker(\phi)) = \phi(x+y) = \phi(x) + \phi(y) =
	\overline{\phi}(x+\Ker(\phi)) + \overline{\phi}(y+\Ker(\phi))$. Also
	$\overline{\phi}(r(x+N)) = \overline{\phi}((rx)+N) = \phi(rx) = r\phi(x)
	= r\overline{\phi}(x+N)$. This shows that $\overline{\phi}$ is a
	homomorphism.
	
	Furthermore if $\overline{\phi}(x+\Ker(\phi)) =
	\overline{\phi}(y+\Ker(\phi))$ then $\phi(x) = \phi(y)$ hence
	$\Ker(\phi)\ni 0 = \phi(x) - \phi(y) = \phi(x-y)$ thus $x+\Ker(\phi) = y
	+\Ker(\phi)$. This shows that $\overline{\phi}$ is injective.
	
	Now let $s\in\Im(\phi)$ and $x\in M$ such that $\phi(x)=s$ then
	$\overline{\phi}(x+\Ker(\phi)) = \phi(x) = s$. This shows that
	$\overline{\phi}$ is surjective.
	
	All together this shows that $M/\Ker(\phi)$ is isomorphic to $\Im(\phi)$.
\end{proof}

\begin{example}
	from examples \ref{example:homomorphism1} and
	\ref{example:homomorphism2} it is clear that $\Z/q\Z$ is isomorphic to
	$\Z/q\Z$.	
\end{example}

\subsection{Free Modules over Commutative Rings}

\begin{theorem}
	Let $M$ be a free $R$-module where $R$ is a commutative ring. Any two
	bases of $M$ have the same cardinality. 
\end{theorem}

for a proof see \cite{roman07}.

\begin{proposition}\label{theorem:MisRn}
	$M\isomorphic R^{n}$ for certain $n\in N$ for every free $R$-module $M$
	where $R$ is a commutative ring.
\end{proposition}

\begin{proof}
	Let $B:=\{b_{1},\ldots,b_{n}\}$ be a basis for $M$. For all $M\ni m :=
	\sum_{i:=1}^{n}m_{i}b_{i}$ define $\phi: M \rightarrow R^{n} : m \mapsto
	\sum_{i:=1}^{n}m_{i}e_{i}$ where the $e_{i}$ is the $i$th standard base
	element. We will show that $\phi$ is a isomorphism.
	
	Let $x,y\in M$ with $x=\sum_{i:=1}^{n}x_{i}b_{i}$ and
	$y=\sum_{i:=1}^{n}y_{i}b_{i}$ and $r\in R$. Then $\phi(x+y) = \phi(
	\sum_{i:=1}^{n} x_{i}b_{i} + \sum_{i:=1}^{n} x_{i}b_{i} ) = \phi(
	\sum_{i:=1}^{n} (x_{i}+y_{i})b_{i} ) = \sum_{i:=1}^{n}(x_{i}+y_{i})e_{i}
	= \sum_{i:=1}^{n}x_{i}e_{i} + \sum_{i:=1}^{n}y_{i}e_{i} = \phi(x) +
	\phi(y)$. Also $\phi(rx) = \phi(\sum_{i:=1}^{n}(rx_{i})b_{i}) =
	\sum_{i:=1}^{n}(rx_{i})e_{i} = r\sum_{i:=1}^{n}(x_{i})e_{i} = r\phi(x)$.
\end{proof}

\begin{proposition}[Rank-Nulllity Theorem]\label{proposition:ranknullity}
	Let $R$ be a PID and $f:M\rightarrow N$ be an $R$-module homomorphism of
	finite dimensional free $R$-modules. Then
	\[
		\dim(M) = \dim( \Ker( f ) ) + \dim( \Im( f ) )
	\]
\end{proposition}

for a proof see \cite{adkins92}.

\begin{remark}
	PID stands for \emph{Principal Ideal Domain}.
\end{remark}

\begin{proof}
	Find a basis for $\Ker(\phi)$ extend to a basis for $V$. $\Im(\phi)
	\isomorphic V/\Ker(\phi)$ so $\dim(\Im(\phi)) = \dim(V) -
	\dim(\Ker(\phi))$.  
\end{proof}

\subsection{Modules of Functions}

\begin{definition}
	Let $X$ be a set and $R$ be a ring. Define the set $\mathcal{M}(X,R)$ of
	functions from $X$ to $R$ of finite support.
	
	For all $f,g\in\mathcal{M}(X,R)$, for all $r\in R$ we define operations
	$\boxplus$ from $\mathcal{M}(X,R)\times\mathcal{M}(X,R)$ to
	$\mathcal{M}(X,R)$ and $\boxdot$ from $R\times\mathcal{M}(X,R)$ to
	$\mathcal{M}(X,R)$. 
	
	Forall $x\in X$:
	\begin{definitionlist}[$\mathcal{M}$]
		\item $(f\boxplus g)(x) := f(x) + g(x)$.
		\item $(r\boxdot f)(x) := r f(x)$. 
	\end{definitionlist}
\end{definition}

\begin{lemma}
	$\mathcal{M}(X,R)$ with operations $\boxplus$ and $\boxdot$ is a
	$R$-module for all sets $X$ and all rings $R$. 
\end{lemma}

\begin{proof}
	Let $X$ be a set and $R$ be a ring. Showing that $\mathcal{M}(X,R)$ is a
	$R$-module amounts to showing that $(\mathcal{M}(X,R),\boxplus)$ is a
	abelian group and that $\boxdot$ behaves nicely with respect to
	$\boxplus$. We first will show that $(\mathcal{M}(X,R),\boxplus)$ is an
	abelian group.
	
	Define $\mathcal{M} := \mathcal{M}(X,R)$ and let $f,g,h\in\mathcal{M}$.
	
	Notice that $\mathcal{M}$ is closed under $\boxplus$. Let $U,V\subset X$
	be the (finite) supports of $f,g$ respectively. It is clear that the
	support of $f\boxplus g$ is contained in $U\cup V$ and is therefore
	necessarily finite.  
	
	Define $O:X\rightarrow R : x\mapsto 0$. Then forall $x\in X$: $f(x) =
	f(x) + 0 = f(x) + O(x) = (f\boxplus O)(x)$ so $O$ acts as the identity
	for $+$.
	
	Furtermore let $\overline{f}:X\rightarrow R: x\mapsto -f(x)$ then
	$\overline{f}\in \mathcal{M}$ and forall $x\in X$:
	$(f\boxplus\overline{f})(x) = f(x) + \overline{f}(x) = 
	f(x) + -f(x) = 0 = O(x)$. Elements in $\mathcal{M}$ have inverses.
	
	Also $(f\boxplus (g\boxplus h))(x) = f(x) + (g\boxplus h)(x) = f(x) +
	g(x) + h(x) = (f\boxplus g)(x) + h(x) = ((f\boxplus g)\boxplus h)(x)$
	thus $+$ is associative. 
	
	Finally $(f\boxplus g)(x) = f(x) + g(x) = g(x) + f(x) = (g\boxplus f)(x)$.
	
	This shows that $(\mathcal{M},\boxplus)$ is a abelian group.
	
	We will continue are proof by showing that $\boxdot$ behaves nicely with
	respect to $\boxplus$. Notice that forall $r\in R$: $r\boxdot
	f\in\mathcal{M}$ because the support of $r\boxdot f$ is contained in the
	support of $f$. So $\mathcal{M}$ is closed under $\boxdot$.
	
	Now we will show the following facts: for all $r,s\in R$ and for all
	$x\in X$
	\begin{namedlist}[\ref{module:ringhomomorphism3}]
		\item[\ref{module:grouphomomorphism}] $(r\boxdot(f\boxplus g))(x) =
		r((f\boxplus g)(x)) = r(f(x)+g(x)) = rf(x) + rg(x) = (r\boxdot
		f)(x) \boxplus (r\boxdot g)(x)$.
		
		\item[\ref{module:ringhomomorphism1}] $((r+s)\boxdot f)(x) =
		(r+s)f(x) = rf(x) + sf(x) = (r\boxdot f)(x) + (s\boxdot f)(x)$.

		\item[\ref{module:ringhomomorphism2}] $((r\boxdot(s\boxdot f))(x) =
		r((s\boxdot f)(x)) = r(sf(x)) = (rs)f(x) = ((rs)\boxdot f)(x)$.
		
		\item[\ref{module:ringhomomorphism3}] $(1\boxdot f)(x) = 1f(x) =
		f(x)$. 
	\end{namedlist}
	
	This shows that $\mathcal{M}$ is a $R$-module.
\end{proof}

\begin{remark}
	We will often write $+$ for $\boxplus$ and supress the use of $\boxdot$
	in calculations in $\mathcal{M}(X,R)$.
\end{remark}

\begin{lemma}
	$\mathcal{M}(X,R)$ is a free $R$-module for all sets $X$ and all rings
	$R$.	
\end{lemma}

\begin{proof}
	Let $\mathcal{M}:=\mathcal{M}(X,R)$.
	
	For every $y\in X$ define $\overline{y}\in\mathcal{M}$  for all $x\in X$
	as
	\[
		\overline{y}(x):=\left\{
		\begin{array}{rl}
			1 & \text{if } x \text{ is equal to } y \\
			0 & \text{otherwise} \\		
		\end{array}
		\right.
	\]
	and let $B:=\{\overline{x}\bar\forall x\in X\}$. We will show that $B$
	is a linearly independent generating set for $\mathcal{M}$.
	
	Take $f\in\mathcal{M}$ and let $U$ be the (finite) support of $f$.
	Notice that $f=\sum_{u\in U} f(u)\overline{u}$. This shows that $B$ is a
	generating set for $\mathcal{M}$.
	
	Now let $S$ be a finite subset of $X$ with $r_{s}\in R$ forall $s\in S$
	such that $\sum_{s\in S} r_{s}\overline{s} = O$. Forall $t\in S$: $r_{t}
	= r_{t} 1 = r_{t} \overline{t}(t) = (\sum_{s\in S} r_{s}\overline{s})(t)
	= O(t) = 0$ which shows that $B$ is lineairly independent set for
	$\mathcal{M}$. 
\end{proof}

\begin{corollary}
	Let $X$ be a finite set. $\mathcal{M}(X,\Z/q\Z)\isomorphic
	(\Z/q\Z)^{\#X}$.
\end{corollary}

\begin{proof}
	This is a consequence of theorem \ref{theorem:MisRn}.
\end{proof}

\subsection{Matrices over Modules}

\begin{proposition}{Smith Normal Form}
	Let $R$ be a principal ideal domain and let $A\in M_{m,n}(R)$. Then
	there is a $U\in GL(m,R)$ and a $V\in GL(n,R)$ such that
	\[
		UAV=\left(
		\begin{array}{cc}
			D_r & 0 \\
			0   & 0 \\
		\end{array}
		\right)
	\]
	where $r=rank(A)$ and $D_r = diag(s_{1},\ldots,s_{r})$ with
	$s_{i}\not=0$ for all $1\le i \le r$ and $s_{i} | s_{i+1}$ for all $1\le
	i \le r-1$. Furthermore, if $R$ is an Euclidean domain, ten the matrices
	U and V can be taken te be a product of elementary matrices.
\end{proposition}

for a proof see \cite{adkins92}

\begin{corollary}\label{corollary:dimcolisdimrow}
	$\dim(\Col(A)) = \dim(\Row(A))$ for every matrix $A$ over a free
	$R$-module where $R$ is a division ring.
\end{corollary}

\begin{example}
	Let $S\in M_{3,4}(\Z/9\Z)$ be given by
	\[
		S = \left(
		\begin{array}{cccc}
			1 & 2 & 3 & 6 \\
			2 & 1 & 0 & 6 \\
			5 & 4 & 6 & 3 \\
		\end{array}
		\right)
	\]
	Will be brought to smith normal form by $U\in M_{3,3}(\Z/9\Z)$ and $V\in
	M_{4,4}(\Z/9\Z)$ where
	\[
		U = \left(
		\begin{array}{ccc}
			0 & 5 & 0 \\
			8 & 5 & 0 \\
			2 & 1 & 1 \\
		\end{array}
		\right)
		\quad
		V = \left(
		\begin{array}{cccc}
			1 & 4 & 4 & 6 \\
			0 & 1 & 1 & 0 \\
			0 & 0 & 1 & 8 \\
			0 & 0 & 0 & 1 \\
		\end{array}
		\right)
	\]
	and
	\[
		USV = \left(
		\begin{array}{cccc}
			1 & 0 & 0 & 0 \\
			0 & 3 & 0 & 0 \\
			0 & 0 & 3 & 0 \\
		\end{array}
		\right)
	\]
\end{example}

